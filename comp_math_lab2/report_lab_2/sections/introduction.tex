\documentclass[../main.tex]{subfiles}
\graphicspath{{\subfix{../images/}}}
\begin{document}

\section{Вариант 7}

\subsection{Постановка задачи}

Для функции, заданной таблично, найти значение производной в указанной точке с максимально возможной точностью с помощью интерполяции.


\begin{table}[h]
    \centering
    \begin{tabular}{|l|l|l|l|l|l|l|}
    \hline
    $f''(0.3)=?$ & $x$ & $x_1 = 0$ & $x_2 = 0.1$ & $x_3 = 0.2$  & $x_4 = 0.3$ & $x_5 = 0.4$  \\ \hline
    & $f(x)$ & $5$ & $2.5$ & $3$ & $-2.5$ & $-0.2$ \\ \hline
    \end{tabular}
\end{table}

\subsection{Полином Лагранжа}

\begin{align*}
    P(x) & = \frac{6250}{3}(x-0.1)(x-0.2)(x-0.3)(x-0.4) + & \\
      & + \frac{-12500}{3}(x-0)(x-0.2)(x-0.3)(x-0.4) + & \\
      & + \frac{7500}{1}(x-0)(x-0.1)(x-0.3)(x-0.4) + & \\
      & + \frac{12500}{3}(x-0)(x-0.1)(x-0.2)(x-0.4) + & \\
      & + \frac{-250}{3}(x-0)(x-0.1)(x-0.2)(x-0.3) = & \\
      & = 9500 x^4 - 7200 x^3 + 1645 x^2 - 127 x + 5 & \\
    P'(x) & = 38000 x^3 - 21600 x^2 + 3290 x - 127 & \\
    P''(x) & = 114000 x^2 - 43200 x + 3290 x & \\
    P''(0.3) & = 590 &
\end{align*}

Полином Ньютона имеет такою же форму. Разделенные разности:

\begin{align*}
    f(x_0; x_1) & = -25 \\
    f(x_0; x_1; x_2) & = 150 \\
    f(x_0; x_1; x_2; x_3) & = -1500 \\
    f(x_0; x_1; x_2; x_3; x_4) & = 9500
\end{align*}

\end{document}
