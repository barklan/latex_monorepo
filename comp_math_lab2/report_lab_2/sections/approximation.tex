\documentclass[../main.tex]{subfiles}
\graphicspath{{\subfix{../images/}}}
\begin{document}


\subsection{Метод половинного деления}

Будем уточнять все корти с точностью $\epsilon = 10^{-4}$.
\bigskip

% \definecolor{bg}{rgb}{0.98,0.98,0.98}
% Use optional arg bgcolor=bg
\inputminted[mathescape,frame=single,framesep=2mm,fontsize=\scriptsize,
label=bisection.go (\href{https://github.com/barklan/computational_mathematics}{GitHub})
]{go}{
../numerical/bisection.go
}

\textbf{Для первой функции}

$$ f(x) = 3x + 4x^3 - 12 x^2 - 5 $$

потребовалось 14 итераций до нужной точности на отрезке $[2.5, 3.5]$ с найденным заначением $x = 2.89019775390625$.

\textbf{Для второй функции}

$$  f(x) = 2 \tan (x) - \frac{x}{2} + 1 = 0 $$

потребовалось 15 итераций на отрезке $[-1, 1]$ c найденным значением $x = -0.57135009765625$.

\newpage
\subsection{Метод секущих}
\bigskip

\inputminted[mathescape,frame=single,framesep=2mm,fontsize=\scriptsize,
label=secant.go
]{go}{
../numerical/secant.go
}

Для достижения нужной точности потребовалось 30 итераций для обеих функций.

\newpage
\subsection{Метод простой итерации}

\bigskip
\inputminted[mathescape,frame=single,framesep=2mm,fontsize=\scriptsize,
label=fixed\_point\_iteration.go
]{go}{
../numerical/fixed_point_iteration.go
}

Для метода простой итерации приведем выражение вида $f(x) = 0$ к виду $x = g(x)$.

$$ x = \sqrt[3]{3x^2 - 0.75x + 1.25} $$

\bigskip
\begin{tiny}
\begin{verbatim}
Iteration-1, x1 = 2.626794 and fn(x1) = -7.420175
Iteration-3, x1 = 2.772188 and fn(x1) = -3.686390
Iteration-5, x1 = 2.837943 and fn(x1) = -1.706944
Iteration-7, x1 = 2.867166 and fn(x1) = -0.766414
Iteration-9, x1 = 2.880054 and fn(x1) = -0.339508
Iteration-11, x1 = 2.885718 and fn(x1) = -0.149510
Iteration-13, x1 = 2.888204 and fn(x1) = -0.065670
Iteration-15, x1 = 2.889294 and fn(x1) = -0.028812
Iteration-17, x1 = 2.889772 and fn(x1) = -0.012634
Iteration-19, x1 = 2.889982 and fn(x1) = -0.005539
Iteration-21, x1 = 2.890074 and fn(x1) = -0.002428
Iteration-23, x1 = 2.890114 and fn(x1) = -0.001064
Iteration-25, x1 = 2.890132 and fn(x1) = -0.000467
Iteration-27, x1 = 2.890139 and fn(x1) = -0.000205
Iteration-29, x1 = 2.890143 and fn(x1) = -0.000090
\end{verbatim}
\end{tiny}

Для заданной точности потребовалось 29 итераций.

\subsection{Метод Ньютона}

\bigskip
\inputminted[mathescape,frame=single,framesep=2mm,fontsize=\scriptsize,
label=newtons\_method.go
]{go}{
../numerical/newtons_method.go
}

Для первого уравнения потребовалось 5 итераций:

\begin{tiny}
\begin{verbatim}
    Newton's method:
    Iteration-1, x1 = 0.000000 and fn(x1) = -5.000000
    Iteration-2, x1 = 3.055556 and fn(x1) = 6.241427
    Iteration-3, x1 = 2.905894 and fn(x1) = 0.539087
    Iteration-4, x1 = 2.890309 and fn(x1) = 0.005540
    Iteration-5, x1 = 2.890145 and fn(x1) = 0.000001
\end{verbatim}
\end{tiny}

\newpage
Для второго - 7:

\begin{tiny}
\begin{verbatim}
    Iteration-1, x1 = 0.000000 and fn(x1) = 1.000000
    Iteration-2, x1 = -0.764296 and fn(x1) = -0.535176
    Iteration-3, x1 = -0.624857 and fn(x1) = -0.130106
    Iteration-4, x1 = -0.582066 and fn(x1) = -0.025216
    Iteration-5, x1 = -0.573268 and fn(x1) = -0.004543
    Iteration-6, x1 = -0.571664 and fn(x1) = -0.000806
    Iteration-7, x1 = -0.571379 and fn(x1) = -0.000143
\end{verbatim}
\end{tiny}


\end{document}
