%!TEX program = xelatex
\documentclass[a4paper,14pt]{extarticle}
\usepackage{fontspec}
\setmainfont[Ligatures={TeX}]{Liberation Serif}
\usepackage{graphicx}
\graphicspath{{./images/}}
\usepackage{float}
\usepackage{geometry}
\geometry{top=2cm}
\geometry{bottom=2cm}
\geometry{left=3cm}
\geometry{right=1cm}
\usepackage{setspace}
\onehalfspacing
\usepackage{amsthm,amssymb, amsmath}
\usepackage{booktabs}
\usepackage{svg}
\usepackage{multirow}
\usepackage{siunitx}
\usepackage{enumitem}
\usepackage{csquotes}
\usepackage{graphicx}
\usepackage{microtype}
\usepackage{pgf}
\usepackage{subfig}
\usepackage{wrapfig}
\usepackage[bookmarks]{hyperref}
\renewcommand{\contentsname}{Содержание}
\renewcommand{\tablename}{Таблица}
\renewcommand{\figurename}{Схема}
\usepackage[parfill]{parskip}

\begin{document}

\thispagestyle{empty}
\begin{center}
    МОСКОВСКИЙ ФИЗИКО-ТЕХНИЧЕСКИЙ ИНСТИТУТ\\
    (ГОСУДАРСТВЕННЫЙ УНИВЕРСИТЕТ)\\
    Физтех-школа аэрокосмических технологий

  \vspace{5cm}
    РАСЧЕТНО-ГРАФИЧЕСКАЯ РАБОТА\\
    \vspace{0.4cm}
    по дисциплине\\
    \vspace{0.4cm}
    «Метрология, стандартизация и сертификация»\\
    \vspace{0.4cm}
    (Вариант 4)
  \vspace{1.5cm}
\end{center}
\begin{flushright}
    Выполнил:\\
    студент 937 группы\\
    Бузин Глеб Борисович

  \vspace{1cm}
    Проверил:\\
    Моисеев Леонид Фёдорович
\end{flushright}
\vfill
\begin{center}
  Москва\\
  2022
\end{center}

\newpage

\tableofcontents

\newpage

\section{Постановка задачи}

Необходимо найти ЭДС источника постоянного тока (с использованием закона Ома), при
которой в нагрузке ($r_x$) с минимальной погрешностью выделяется требуемая мощность $P$.

\subsection{Исходные данные}

\begin{table}[ht]
  \centering
  \begin{tabular}{|c|c|c|c|c|}
   \hline
  № варианта & \parbox[t]{3cm}{№ типа\\вольтметра} & \parbox[t]{3cm}{№ типа\\амперметра} & \parbox[t]{4cm}{Выделяемая\\мощность P, Вт} & $R_{вн}$ \\
   \hline
   4  & 5 & 2 & 0.5 & 10 \\
   \hline
  \end{tabular}
  \caption{Параметры варианта}
  \label{tableref1}
\end{table}

\begin{table}[ht]
  \centering
  \begin{tabular}{|c|c|c|}
   \hline
     & № типа вольтметра & 5 \\
   \hline
   1 & Диапазон измерений, В & 0-50 \\
   \hline
   2 & Внутреннее сопротивление, кОм & 250 \\
   \hline
   3 & Класс точности & 0.1/0.2 \\
   \hline
  \end{tabular}
  \caption{Параметры вольтметра}
  \label{tableref2}
\end{table}

\begin{table}[ht]
  \centering
  \begin{tabular}{|c|c|c|}
   \hline
     & № типа амперметра & 2 \\
   \hline
   1 & Диапазон измерений, мА & 0-20 \\
   \hline
   2 & Внутреннее сопротивление, кОм & 0.01 \\
   \hline
   3 & Класс точности & 0.02/0.02 \\
   \hline
  \end{tabular}
  \caption{Параметры амперметра}
  \label{tableref3}
\end{table}

\section{Решение}

\begin{figure}[H]
 \centering
  \includesvg[width=0.5\textwidth]{chart1.svg}
  \caption{Изначальная схема}
  \label{pictureref1}
\end{figure}

% Можем сослаться на рисунок~\ref{pictureref}.

\begin{figure}[H]
 \centering
  \includesvg[width=0.7\textwidth]{chart2.svg}
  \caption{Преобразованная схема}
  \label{pictureref2}
\end{figure}

Закон Ома для схемы:

\begin{align}
  &I_A = I_V + I_r \\
  &U_V = U_r = U \\
  &\varepsilon = U_V + I_A (R_A + R_{\varepsilon})
\end{align}

Так как внутреннее сопротивление источника $R_{\varepsilon}$ = 0, то ЭДС равна:

\begin{align}
  \varepsilon = U_V + I_A R_A
\end{align}

Выделяемая на R мощность P равна:

\begin{align}
  P = I_r U_r = I_r U
\end{align}

Следовательно, из (1) и (4) получаем:

\begin{align}
  I_A = \frac{P}{U} + \frac{U}{R_V} = \frac{P R_V + U^2}{U R_V}
\end{align}

Погрешность вычисления мощности $\Delta P = P \times \delta_p$,

\begin{align}
  \delta_p = \sqrt{\delta_A^2 + \delta_V^2}
\end{align}

где $\delta_p \delta_v \delta_a$ - относительные погрешности измерения мощности, напряжения и тока соответственно.

\begin{align}
  \delta_A = \pm c_A + d_A (|\frac{D_A}{I_A}| - 1))
\end{align}

где $c_A, d_A$ - класс точности амперметра, $D_A$ - диапазон измерений амперметра.

\begin{align}
  \delta_V = \pm c_V + d_V (|\frac{U_K}{U_V}| - 1))
\end{align}

где $c_V, d_V$ - класс точности вольтметра, $U_k$ – нормирующее значение напряжения (предельное значение),  $U_v$ - действительное значение измеряемой величины.

Для решения задачи нам нужно минимизировать величину $\delta_p$ по напряжению.
Для удобства исследуем производную $(\delta_p^2)'$ и найдем значение U,\\
которое минимизирует $\delta_p^2$. Это же значение U будет минимизировать величину δp, т.к. возведение в квадрат и извлечение корня – монотонные преобразования.

\begin{align*}
  \delta_p^2 &= \delta_A^2 + \delta_V^2 = \\
  &= (c_A + d_A (|\frac{D_A}{I_A}| - 1))^2 + (c_V + d_V (|\frac{U_K}{U_V}| - 1))^2 = \\
  &= (0.02 + 0.02(|\frac{5000U}{125*10^3 + U^2}| - 1))^2 + \\
  &+ (0.1 + 0.2(|\frac{50}{U_V}| - 1))^2
\end{align*}

\begin{figure}[H]
  %% Creator: Matplotlib, PGF backend
%%
%% To include the figure in your LaTeX document, write
%%   \input{<filename>.pgf}
%%
%% Make sure the required packages are loaded in your preamble
%%   \usepackage{pgf}
%%
%% Also ensure that all the required font packages are loaded; for instance,
%% the lmodern package is sometimes necessary when using math font.
%%   \usepackage{lmodern}
%%
%% Figures using additional raster images can only be included by \input if
%% they are in the same directory as the main LaTeX file. For loading figures
%% from other directories you can use the `import` package
%%   \usepackage{import}
%%
%% and then include the figures with
%%   \import{<path to file>}{<filename>.pgf}
%%
%% Matplotlib used the following preamble
%%
\begingroup%
\makeatletter%
\begin{pgfpicture}%
\pgfpathrectangle{\pgfpointorigin}{\pgfqpoint{6.543671in}{4.119792in}}%
\pgfusepath{use as bounding box, clip}%
\begin{pgfscope}%
\pgfsetbuttcap%
\pgfsetmiterjoin%
\pgfsetlinewidth{0.000000pt}%
\definecolor{currentstroke}{rgb}{0.000000,0.000000,0.000000}%
\pgfsetstrokecolor{currentstroke}%
\pgfsetstrokeopacity{0.000000}%
\pgfsetdash{}{0pt}%
\pgfpathmoveto{\pgfqpoint{0.000000in}{-0.000000in}}%
\pgfpathlineto{\pgfqpoint{6.543671in}{-0.000000in}}%
\pgfpathlineto{\pgfqpoint{6.543671in}{4.119792in}}%
\pgfpathlineto{\pgfqpoint{0.000000in}{4.119792in}}%
\pgfpathlineto{\pgfqpoint{0.000000in}{-0.000000in}}%
\pgfpathclose%
\pgfusepath{}%
\end{pgfscope}%
\begin{pgfscope}%
\pgfsetbuttcap%
\pgfsetmiterjoin%
\pgfsetlinewidth{0.000000pt}%
\definecolor{currentstroke}{rgb}{0.000000,0.000000,0.000000}%
\pgfsetstrokecolor{currentstroke}%
\pgfsetstrokeopacity{0.000000}%
\pgfsetdash{}{0pt}%
\pgfpathmoveto{\pgfqpoint{0.343671in}{0.226389in}}%
\pgfpathlineto{\pgfqpoint{6.543671in}{0.226389in}}%
\pgfpathlineto{\pgfqpoint{6.543671in}{4.076389in}}%
\pgfpathlineto{\pgfqpoint{0.343671in}{4.076389in}}%
\pgfpathlineto{\pgfqpoint{0.343671in}{0.226389in}}%
\pgfpathclose%
\pgfusepath{}%
\end{pgfscope}%
\begin{pgfscope}%
\pgfpathrectangle{\pgfqpoint{0.343671in}{0.226389in}}{\pgfqpoint{6.200000in}{3.850000in}}%
\pgfusepath{clip}%
\pgfsetroundcap%
\pgfsetroundjoin%
\pgfsetlinewidth{0.803000pt}%
\definecolor{currentstroke}{rgb}{0.800000,0.800000,0.800000}%
\pgfsetstrokecolor{currentstroke}%
\pgfsetdash{}{0pt}%
\pgfpathmoveto{\pgfqpoint{0.624362in}{0.226389in}}%
\pgfpathlineto{\pgfqpoint{0.624362in}{4.076389in}}%
\pgfusepath{stroke}%
\end{pgfscope}%
\begin{pgfscope}%
\definecolor{textcolor}{rgb}{0.501961,0.501961,0.501961}%
\pgfsetstrokecolor{textcolor}%
\pgfsetfillcolor{textcolor}%
\pgftext[x=0.624362in,y=0.111111in,,top]{\color{textcolor}\rmfamily\fontsize{8.800000}{10.560000}\selectfont \(\displaystyle {0}\)}%
\end{pgfscope}%
\begin{pgfscope}%
\pgfpathrectangle{\pgfqpoint{0.343671in}{0.226389in}}{\pgfqpoint{6.200000in}{3.850000in}}%
\pgfusepath{clip}%
\pgfsetroundcap%
\pgfsetroundjoin%
\pgfsetlinewidth{0.803000pt}%
\definecolor{currentstroke}{rgb}{0.800000,0.800000,0.800000}%
\pgfsetstrokecolor{currentstroke}%
\pgfsetdash{}{0pt}%
\pgfpathmoveto{\pgfqpoint{1.752086in}{0.226389in}}%
\pgfpathlineto{\pgfqpoint{1.752086in}{4.076389in}}%
\pgfusepath{stroke}%
\end{pgfscope}%
\begin{pgfscope}%
\definecolor{textcolor}{rgb}{0.501961,0.501961,0.501961}%
\pgfsetstrokecolor{textcolor}%
\pgfsetfillcolor{textcolor}%
\pgftext[x=1.752086in,y=0.111111in,,top]{\color{textcolor}\rmfamily\fontsize{8.800000}{10.560000}\selectfont \(\displaystyle {10}\)}%
\end{pgfscope}%
\begin{pgfscope}%
\pgfpathrectangle{\pgfqpoint{0.343671in}{0.226389in}}{\pgfqpoint{6.200000in}{3.850000in}}%
\pgfusepath{clip}%
\pgfsetroundcap%
\pgfsetroundjoin%
\pgfsetlinewidth{0.803000pt}%
\definecolor{currentstroke}{rgb}{0.800000,0.800000,0.800000}%
\pgfsetstrokecolor{currentstroke}%
\pgfsetdash{}{0pt}%
\pgfpathmoveto{\pgfqpoint{2.879809in}{0.226389in}}%
\pgfpathlineto{\pgfqpoint{2.879809in}{4.076389in}}%
\pgfusepath{stroke}%
\end{pgfscope}%
\begin{pgfscope}%
\definecolor{textcolor}{rgb}{0.501961,0.501961,0.501961}%
\pgfsetstrokecolor{textcolor}%
\pgfsetfillcolor{textcolor}%
\pgftext[x=2.879809in,y=0.111111in,,top]{\color{textcolor}\rmfamily\fontsize{8.800000}{10.560000}\selectfont \(\displaystyle {20}\)}%
\end{pgfscope}%
\begin{pgfscope}%
\pgfpathrectangle{\pgfqpoint{0.343671in}{0.226389in}}{\pgfqpoint{6.200000in}{3.850000in}}%
\pgfusepath{clip}%
\pgfsetroundcap%
\pgfsetroundjoin%
\pgfsetlinewidth{0.803000pt}%
\definecolor{currentstroke}{rgb}{0.800000,0.800000,0.800000}%
\pgfsetstrokecolor{currentstroke}%
\pgfsetdash{}{0pt}%
\pgfpathmoveto{\pgfqpoint{4.007533in}{0.226389in}}%
\pgfpathlineto{\pgfqpoint{4.007533in}{4.076389in}}%
\pgfusepath{stroke}%
\end{pgfscope}%
\begin{pgfscope}%
\definecolor{textcolor}{rgb}{0.501961,0.501961,0.501961}%
\pgfsetstrokecolor{textcolor}%
\pgfsetfillcolor{textcolor}%
\pgftext[x=4.007533in,y=0.111111in,,top]{\color{textcolor}\rmfamily\fontsize{8.800000}{10.560000}\selectfont \(\displaystyle {30}\)}%
\end{pgfscope}%
\begin{pgfscope}%
\pgfpathrectangle{\pgfqpoint{0.343671in}{0.226389in}}{\pgfqpoint{6.200000in}{3.850000in}}%
\pgfusepath{clip}%
\pgfsetroundcap%
\pgfsetroundjoin%
\pgfsetlinewidth{0.803000pt}%
\definecolor{currentstroke}{rgb}{0.800000,0.800000,0.800000}%
\pgfsetstrokecolor{currentstroke}%
\pgfsetdash{}{0pt}%
\pgfpathmoveto{\pgfqpoint{5.135257in}{0.226389in}}%
\pgfpathlineto{\pgfqpoint{5.135257in}{4.076389in}}%
\pgfusepath{stroke}%
\end{pgfscope}%
\begin{pgfscope}%
\definecolor{textcolor}{rgb}{0.501961,0.501961,0.501961}%
\pgfsetstrokecolor{textcolor}%
\pgfsetfillcolor{textcolor}%
\pgftext[x=5.135257in,y=0.111111in,,top]{\color{textcolor}\rmfamily\fontsize{8.800000}{10.560000}\selectfont \(\displaystyle {40}\)}%
\end{pgfscope}%
\begin{pgfscope}%
\pgfpathrectangle{\pgfqpoint{0.343671in}{0.226389in}}{\pgfqpoint{6.200000in}{3.850000in}}%
\pgfusepath{clip}%
\pgfsetroundcap%
\pgfsetroundjoin%
\pgfsetlinewidth{0.803000pt}%
\definecolor{currentstroke}{rgb}{0.800000,0.800000,0.800000}%
\pgfsetstrokecolor{currentstroke}%
\pgfsetdash{}{0pt}%
\pgfpathmoveto{\pgfqpoint{6.262981in}{0.226389in}}%
\pgfpathlineto{\pgfqpoint{6.262981in}{4.076389in}}%
\pgfusepath{stroke}%
\end{pgfscope}%
\begin{pgfscope}%
\definecolor{textcolor}{rgb}{0.501961,0.501961,0.501961}%
\pgfsetstrokecolor{textcolor}%
\pgfsetfillcolor{textcolor}%
\pgftext[x=6.262981in,y=0.111111in,,top]{\color{textcolor}\rmfamily\fontsize{8.800000}{10.560000}\selectfont \(\displaystyle {50}\)}%
\end{pgfscope}%
\begin{pgfscope}%
\pgfpathrectangle{\pgfqpoint{0.343671in}{0.226389in}}{\pgfqpoint{6.200000in}{3.850000in}}%
\pgfusepath{clip}%
\pgfsetroundcap%
\pgfsetroundjoin%
\pgfsetlinewidth{0.803000pt}%
\definecolor{currentstroke}{rgb}{0.800000,0.800000,0.800000}%
\pgfsetstrokecolor{currentstroke}%
\pgfsetdash{}{0pt}%
\pgfpathmoveto{\pgfqpoint{0.343671in}{0.226389in}}%
\pgfpathlineto{\pgfqpoint{6.543671in}{0.226389in}}%
\pgfusepath{stroke}%
\end{pgfscope}%
\begin{pgfscope}%
\definecolor{textcolor}{rgb}{0.501961,0.501961,0.501961}%
\pgfsetstrokecolor{textcolor}%
\pgfsetfillcolor{textcolor}%
\pgftext[x=-0.000000in, y=0.182986in, left, base]{\color{textcolor}\rmfamily\fontsize{8.800000}{10.560000}\selectfont \(\displaystyle {\ensuremath{-}50}\)}%
\end{pgfscope}%
\begin{pgfscope}%
\pgfpathrectangle{\pgfqpoint{0.343671in}{0.226389in}}{\pgfqpoint{6.200000in}{3.850000in}}%
\pgfusepath{clip}%
\pgfsetroundcap%
\pgfsetroundjoin%
\pgfsetlinewidth{0.803000pt}%
\definecolor{currentstroke}{rgb}{0.800000,0.800000,0.800000}%
\pgfsetstrokecolor{currentstroke}%
\pgfsetdash{}{0pt}%
\pgfpathmoveto{\pgfqpoint{0.343671in}{0.776389in}}%
\pgfpathlineto{\pgfqpoint{6.543671in}{0.776389in}}%
\pgfusepath{stroke}%
\end{pgfscope}%
\begin{pgfscope}%
\definecolor{textcolor}{rgb}{0.501961,0.501961,0.501961}%
\pgfsetstrokecolor{textcolor}%
\pgfsetfillcolor{textcolor}%
\pgftext[x=0.164158in, y=0.732986in, left, base]{\color{textcolor}\rmfamily\fontsize{8.800000}{10.560000}\selectfont \(\displaystyle {0}\)}%
\end{pgfscope}%
\begin{pgfscope}%
\pgfpathrectangle{\pgfqpoint{0.343671in}{0.226389in}}{\pgfqpoint{6.200000in}{3.850000in}}%
\pgfusepath{clip}%
\pgfsetroundcap%
\pgfsetroundjoin%
\pgfsetlinewidth{0.803000pt}%
\definecolor{currentstroke}{rgb}{0.800000,0.800000,0.800000}%
\pgfsetstrokecolor{currentstroke}%
\pgfsetdash{}{0pt}%
\pgfpathmoveto{\pgfqpoint{0.343671in}{1.326389in}}%
\pgfpathlineto{\pgfqpoint{6.543671in}{1.326389in}}%
\pgfusepath{stroke}%
\end{pgfscope}%
\begin{pgfscope}%
\definecolor{textcolor}{rgb}{0.501961,0.501961,0.501961}%
\pgfsetstrokecolor{textcolor}%
\pgfsetfillcolor{textcolor}%
\pgftext[x=0.099922in, y=1.282986in, left, base]{\color{textcolor}\rmfamily\fontsize{8.800000}{10.560000}\selectfont \(\displaystyle {50}\)}%
\end{pgfscope}%
\begin{pgfscope}%
\pgfpathrectangle{\pgfqpoint{0.343671in}{0.226389in}}{\pgfqpoint{6.200000in}{3.850000in}}%
\pgfusepath{clip}%
\pgfsetroundcap%
\pgfsetroundjoin%
\pgfsetlinewidth{0.803000pt}%
\definecolor{currentstroke}{rgb}{0.800000,0.800000,0.800000}%
\pgfsetstrokecolor{currentstroke}%
\pgfsetdash{}{0pt}%
\pgfpathmoveto{\pgfqpoint{0.343671in}{1.876389in}}%
\pgfpathlineto{\pgfqpoint{6.543671in}{1.876389in}}%
\pgfusepath{stroke}%
\end{pgfscope}%
\begin{pgfscope}%
\definecolor{textcolor}{rgb}{0.501961,0.501961,0.501961}%
\pgfsetstrokecolor{textcolor}%
\pgfsetfillcolor{textcolor}%
\pgftext[x=0.035687in, y=1.832986in, left, base]{\color{textcolor}\rmfamily\fontsize{8.800000}{10.560000}\selectfont \(\displaystyle {100}\)}%
\end{pgfscope}%
\begin{pgfscope}%
\pgfpathrectangle{\pgfqpoint{0.343671in}{0.226389in}}{\pgfqpoint{6.200000in}{3.850000in}}%
\pgfusepath{clip}%
\pgfsetroundcap%
\pgfsetroundjoin%
\pgfsetlinewidth{0.803000pt}%
\definecolor{currentstroke}{rgb}{0.800000,0.800000,0.800000}%
\pgfsetstrokecolor{currentstroke}%
\pgfsetdash{}{0pt}%
\pgfpathmoveto{\pgfqpoint{0.343671in}{2.426389in}}%
\pgfpathlineto{\pgfqpoint{6.543671in}{2.426389in}}%
\pgfusepath{stroke}%
\end{pgfscope}%
\begin{pgfscope}%
\definecolor{textcolor}{rgb}{0.501961,0.501961,0.501961}%
\pgfsetstrokecolor{textcolor}%
\pgfsetfillcolor{textcolor}%
\pgftext[x=0.035687in, y=2.382986in, left, base]{\color{textcolor}\rmfamily\fontsize{8.800000}{10.560000}\selectfont \(\displaystyle {150}\)}%
\end{pgfscope}%
\begin{pgfscope}%
\pgfpathrectangle{\pgfqpoint{0.343671in}{0.226389in}}{\pgfqpoint{6.200000in}{3.850000in}}%
\pgfusepath{clip}%
\pgfsetroundcap%
\pgfsetroundjoin%
\pgfsetlinewidth{0.803000pt}%
\definecolor{currentstroke}{rgb}{0.800000,0.800000,0.800000}%
\pgfsetstrokecolor{currentstroke}%
\pgfsetdash{}{0pt}%
\pgfpathmoveto{\pgfqpoint{0.343671in}{2.976389in}}%
\pgfpathlineto{\pgfqpoint{6.543671in}{2.976389in}}%
\pgfusepath{stroke}%
\end{pgfscope}%
\begin{pgfscope}%
\definecolor{textcolor}{rgb}{0.501961,0.501961,0.501961}%
\pgfsetstrokecolor{textcolor}%
\pgfsetfillcolor{textcolor}%
\pgftext[x=0.035687in, y=2.932986in, left, base]{\color{textcolor}\rmfamily\fontsize{8.800000}{10.560000}\selectfont \(\displaystyle {200}\)}%
\end{pgfscope}%
\begin{pgfscope}%
\pgfpathrectangle{\pgfqpoint{0.343671in}{0.226389in}}{\pgfqpoint{6.200000in}{3.850000in}}%
\pgfusepath{clip}%
\pgfsetroundcap%
\pgfsetroundjoin%
\pgfsetlinewidth{0.803000pt}%
\definecolor{currentstroke}{rgb}{0.800000,0.800000,0.800000}%
\pgfsetstrokecolor{currentstroke}%
\pgfsetdash{}{0pt}%
\pgfpathmoveto{\pgfqpoint{0.343671in}{3.526389in}}%
\pgfpathlineto{\pgfqpoint{6.543671in}{3.526389in}}%
\pgfusepath{stroke}%
\end{pgfscope}%
\begin{pgfscope}%
\definecolor{textcolor}{rgb}{0.501961,0.501961,0.501961}%
\pgfsetstrokecolor{textcolor}%
\pgfsetfillcolor{textcolor}%
\pgftext[x=0.035687in, y=3.482986in, left, base]{\color{textcolor}\rmfamily\fontsize{8.800000}{10.560000}\selectfont \(\displaystyle {250}\)}%
\end{pgfscope}%
\begin{pgfscope}%
\pgfpathrectangle{\pgfqpoint{0.343671in}{0.226389in}}{\pgfqpoint{6.200000in}{3.850000in}}%
\pgfusepath{clip}%
\pgfsetroundcap%
\pgfsetroundjoin%
\pgfsetlinewidth{0.803000pt}%
\definecolor{currentstroke}{rgb}{0.800000,0.800000,0.800000}%
\pgfsetstrokecolor{currentstroke}%
\pgfsetdash{}{0pt}%
\pgfpathmoveto{\pgfqpoint{0.343671in}{4.076389in}}%
\pgfpathlineto{\pgfqpoint{6.543671in}{4.076389in}}%
\pgfusepath{stroke}%
\end{pgfscope}%
\begin{pgfscope}%
\definecolor{textcolor}{rgb}{0.501961,0.501961,0.501961}%
\pgfsetstrokecolor{textcolor}%
\pgfsetfillcolor{textcolor}%
\pgftext[x=0.035687in, y=4.032986in, left, base]{\color{textcolor}\rmfamily\fontsize{8.800000}{10.560000}\selectfont \(\displaystyle {300}\)}%
\end{pgfscope}%
\begin{pgfscope}%
\pgfpathrectangle{\pgfqpoint{0.343671in}{0.226389in}}{\pgfqpoint{6.200000in}{3.850000in}}%
\pgfusepath{clip}%
\pgfsetroundcap%
\pgfsetroundjoin%
\pgfsetlinewidth{1.204500pt}%
\definecolor{currentstroke}{rgb}{0.007843,0.243137,1.000000}%
\pgfsetstrokecolor{currentstroke}%
\pgfsetdash{}{0pt}%
\pgfpathmoveto{\pgfqpoint{0.689088in}{4.078056in}}%
\pgfpathlineto{\pgfqpoint{0.692025in}{3.795388in}}%
\pgfpathlineto{\pgfqpoint{0.695408in}{3.513057in}}%
\pgfpathlineto{\pgfqpoint{0.698792in}{3.268418in}}%
\pgfpathlineto{\pgfqpoint{0.702175in}{3.055054in}}%
\pgfpathlineto{\pgfqpoint{0.705558in}{2.867857in}}%
\pgfpathlineto{\pgfqpoint{0.708941in}{2.702721in}}%
\pgfpathlineto{\pgfqpoint{0.712324in}{2.556315in}}%
\pgfpathlineto{\pgfqpoint{0.715707in}{2.425912in}}%
\pgfpathlineto{\pgfqpoint{0.719091in}{2.309265in}}%
\pgfpathlineto{\pgfqpoint{0.722474in}{2.204508in}}%
\pgfpathlineto{\pgfqpoint{0.725857in}{2.110079in}}%
\pgfpathlineto{\pgfqpoint{0.729240in}{2.024666in}}%
\pgfpathlineto{\pgfqpoint{0.732623in}{1.947159in}}%
\pgfpathlineto{\pgfqpoint{0.736007in}{1.876611in}}%
\pgfpathlineto{\pgfqpoint{0.739390in}{1.812216in}}%
\pgfpathlineto{\pgfqpoint{0.742773in}{1.753279in}}%
\pgfpathlineto{\pgfqpoint{0.747284in}{1.682163in}}%
\pgfpathlineto{\pgfqpoint{0.751795in}{1.618491in}}%
\pgfpathlineto{\pgfqpoint{0.756306in}{1.561260in}}%
\pgfpathlineto{\pgfqpoint{0.760816in}{1.509632in}}%
\pgfpathlineto{\pgfqpoint{0.765327in}{1.462899in}}%
\pgfpathlineto{\pgfqpoint{0.769838in}{1.420463in}}%
\pgfpathlineto{\pgfqpoint{0.774349in}{1.381813in}}%
\pgfpathlineto{\pgfqpoint{0.778860in}{1.346513in}}%
\pgfpathlineto{\pgfqpoint{0.783371in}{1.314188in}}%
\pgfpathlineto{\pgfqpoint{0.787882in}{1.284513in}}%
\pgfpathlineto{\pgfqpoint{0.792393in}{1.257207in}}%
\pgfpathlineto{\pgfqpoint{0.796904in}{1.232025in}}%
\pgfpathlineto{\pgfqpoint{0.801414in}{1.208752in}}%
\pgfpathlineto{\pgfqpoint{0.805925in}{1.187201in}}%
\pgfpathlineto{\pgfqpoint{0.810436in}{1.167206in}}%
\pgfpathlineto{\pgfqpoint{0.814947in}{1.148622in}}%
\pgfpathlineto{\pgfqpoint{0.819458in}{1.131319in}}%
\pgfpathlineto{\pgfqpoint{0.825097in}{1.111318in}}%
\pgfpathlineto{\pgfqpoint{0.830735in}{1.092943in}}%
\pgfpathlineto{\pgfqpoint{0.836374in}{1.076024in}}%
\pgfpathlineto{\pgfqpoint{0.842013in}{1.060410in}}%
\pgfpathlineto{\pgfqpoint{0.847651in}{1.045971in}}%
\pgfpathlineto{\pgfqpoint{0.853290in}{1.032593in}}%
\pgfpathlineto{\pgfqpoint{0.858928in}{1.020175in}}%
\pgfpathlineto{\pgfqpoint{0.864567in}{1.008627in}}%
\pgfpathlineto{\pgfqpoint{0.870206in}{0.997869in}}%
\pgfpathlineto{\pgfqpoint{0.875844in}{0.987832in}}%
\pgfpathlineto{\pgfqpoint{0.881483in}{0.978453in}}%
\pgfpathlineto{\pgfqpoint{0.888249in}{0.967988in}}%
\pgfpathlineto{\pgfqpoint{0.895016in}{0.958304in}}%
\pgfpathlineto{\pgfqpoint{0.901782in}{0.949326in}}%
\pgfpathlineto{\pgfqpoint{0.908548in}{0.940986in}}%
\pgfpathlineto{\pgfqpoint{0.915315in}{0.933226in}}%
\pgfpathlineto{\pgfqpoint{0.922081in}{0.925994in}}%
\pgfpathlineto{\pgfqpoint{0.928847in}{0.919242in}}%
\pgfpathlineto{\pgfqpoint{0.936741in}{0.911918in}}%
\pgfpathlineto{\pgfqpoint{0.944635in}{0.905134in}}%
\pgfpathlineto{\pgfqpoint{0.952529in}{0.898838in}}%
\pgfpathlineto{\pgfqpoint{0.960424in}{0.892985in}}%
\pgfpathlineto{\pgfqpoint{0.969445in}{0.886786in}}%
\pgfpathlineto{\pgfqpoint{0.978467in}{0.881059in}}%
\pgfpathlineto{\pgfqpoint{0.987489in}{0.875758in}}%
\pgfpathlineto{\pgfqpoint{0.997638in}{0.870253in}}%
\pgfpathlineto{\pgfqpoint{1.007788in}{0.865184in}}%
\pgfpathlineto{\pgfqpoint{1.017937in}{0.860506in}}%
\pgfpathlineto{\pgfqpoint{1.029215in}{0.855721in}}%
\pgfpathlineto{\pgfqpoint{1.040492in}{0.851324in}}%
\pgfpathlineto{\pgfqpoint{1.052897in}{0.846887in}}%
\pgfpathlineto{\pgfqpoint{1.066430in}{0.842472in}}%
\pgfpathlineto{\pgfqpoint{1.079962in}{0.838449in}}%
\pgfpathlineto{\pgfqpoint{1.094623in}{0.834482in}}%
\pgfpathlineto{\pgfqpoint{1.110411in}{0.830610in}}%
\pgfpathlineto{\pgfqpoint{1.127327in}{0.826866in}}%
\pgfpathlineto{\pgfqpoint{1.145370in}{0.823273in}}%
\pgfpathlineto{\pgfqpoint{1.164542in}{0.819849in}}%
\pgfpathlineto{\pgfqpoint{1.184841in}{0.816605in}}%
\pgfpathlineto{\pgfqpoint{1.206267in}{0.813549in}}%
\pgfpathlineto{\pgfqpoint{1.229950in}{0.810548in}}%
\pgfpathlineto{\pgfqpoint{1.255887in}{0.807647in}}%
\pgfpathlineto{\pgfqpoint{1.284080in}{0.804881in}}%
\pgfpathlineto{\pgfqpoint{1.314529in}{0.802273in}}%
\pgfpathlineto{\pgfqpoint{1.348361in}{0.799761in}}%
\pgfpathlineto{\pgfqpoint{1.385575in}{0.797383in}}%
\pgfpathlineto{\pgfqpoint{1.426173in}{0.795165in}}%
\pgfpathlineto{\pgfqpoint{1.471282in}{0.793073in}}%
\pgfpathlineto{\pgfqpoint{1.522030in}{0.791096in}}%
\pgfpathlineto{\pgfqpoint{1.579544in}{0.789235in}}%
\pgfpathlineto{\pgfqpoint{1.644952in}{0.787499in}}%
\pgfpathlineto{\pgfqpoint{1.719382in}{0.785901in}}%
\pgfpathlineto{\pgfqpoint{1.806216in}{0.784416in}}%
\pgfpathlineto{\pgfqpoint{1.907712in}{0.783061in}}%
\pgfpathlineto{\pgfqpoint{2.028378in}{0.781830in}}%
\pgfpathlineto{\pgfqpoint{2.174982in}{0.780718in}}%
\pgfpathlineto{\pgfqpoint{2.355418in}{0.779736in}}%
\pgfpathlineto{\pgfqpoint{2.584346in}{0.778877in}}%
\pgfpathlineto{\pgfqpoint{2.883193in}{0.778145in}}%
\pgfpathlineto{\pgfqpoint{3.291429in}{0.777539in}}%
\pgfpathlineto{\pgfqpoint{3.882356in}{0.777061in}}%
\pgfpathlineto{\pgfqpoint{4.814984in}{0.776713in}}%
\pgfpathlineto{\pgfqpoint{6.261853in}{0.776516in}}%
\pgfpathlineto{\pgfqpoint{6.261853in}{0.776516in}}%
\pgfusepath{stroke}%
\end{pgfscope}%
\begin{pgfscope}%
\pgfsetrectcap%
\pgfsetmiterjoin%
\pgfsetlinewidth{1.003750pt}%
\definecolor{currentstroke}{rgb}{0.150000,0.150000,0.150000}%
\pgfsetstrokecolor{currentstroke}%
\pgfsetdash{}{0pt}%
\pgfpathmoveto{\pgfqpoint{0.343671in}{0.226389in}}%
\pgfpathlineto{\pgfqpoint{0.343671in}{4.076389in}}%
\pgfusepath{stroke}%
\end{pgfscope}%
\begin{pgfscope}%
\pgfsetrectcap%
\pgfsetmiterjoin%
\pgfsetlinewidth{1.003750pt}%
\definecolor{currentstroke}{rgb}{0.150000,0.150000,0.150000}%
\pgfsetstrokecolor{currentstroke}%
\pgfsetdash{}{0pt}%
\pgfpathmoveto{\pgfqpoint{6.543671in}{0.226389in}}%
\pgfpathlineto{\pgfqpoint{6.543671in}{4.076389in}}%
\pgfusepath{stroke}%
\end{pgfscope}%
\begin{pgfscope}%
\pgfsetrectcap%
\pgfsetmiterjoin%
\pgfsetlinewidth{1.003750pt}%
\definecolor{currentstroke}{rgb}{0.150000,0.150000,0.150000}%
\pgfsetstrokecolor{currentstroke}%
\pgfsetdash{}{0pt}%
\pgfpathmoveto{\pgfqpoint{0.343671in}{0.226389in}}%
\pgfpathlineto{\pgfqpoint{6.543671in}{0.226389in}}%
\pgfusepath{stroke}%
\end{pgfscope}%
\begin{pgfscope}%
\pgfsetrectcap%
\pgfsetmiterjoin%
\pgfsetlinewidth{1.003750pt}%
\definecolor{currentstroke}{rgb}{0.150000,0.150000,0.150000}%
\pgfsetstrokecolor{currentstroke}%
\pgfsetdash{}{0pt}%
\pgfpathmoveto{\pgfqpoint{0.343671in}{4.076389in}}%
\pgfpathlineto{\pgfqpoint{6.543671in}{4.076389in}}%
\pgfusepath{stroke}%
\end{pgfscope}%
\end{pgfpicture}%
\makeatother%
\endgroup%

  \caption{$ \delta^2(U) $ на диапазоне измерений }
\end{figure}

Функция строго убывающая, т.е. минимум достигается в правой краевой точке, т.е. там, где напряжение будет наибольшим на допустимом множестве.

Таким образом, подставим значение $U = 50 B$ в (6):

\begin{align}
  I_A = \frac{0.5 \times 250 \times 10^3 + 50^2}{50 \times 250 \times 10^3} = 0.0102 (A)
\end{align}

Найденное значение $I_A$ попадает в заданный диапазон (0-20 мА), и таким образом минимум достигается при $I_A = 10.2$ мА.

Найдем ЭДС источника из (4):

\begin{align}
  \varepsilon = 50 + 0.0102 \times 0.01 \times 10^3 = 50.102 (B)
\end{align}

Погрешность определения мощности на нагрузке минимальна при $ \varepsilon = 50.102 B $.

\end{document}
