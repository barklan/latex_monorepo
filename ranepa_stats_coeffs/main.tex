% !TeX program = xelatex
\documentclass[a4paper,11pt]{article}
\usepackage{fontspec}
\usepackage{amsmath}
\setmonofont{JetBrains Mono}

\usepackage{unicode-math}
\defaultfontfeatures{Scale=MatchLowercase}
\setmainfont[Ligatures=TeX]{Noto Sans}
\setmathfont{STIX Two Math}

% \usepackage{fullpage}
\usepackage[a4paper,inner=1.7cm,outer=2.7cm,top=2cm,bottom=2cm,bindingoffset=1.2cm]{geometry}

\usepackage{microtype}

% Graphics
\usepackage{graphicx}
\usepackage{pgf}
\usepackage{subfig}
\usepackage{wrapfig}

\usepackage{enumitem}
\usepackage{fancyhdr}
% \usepackage[font=scriptsize]{caption}
\usepackage{index}
\makeindex

% SPACING
% Paragraths without indentation
% \usepackage{parskip}

% Indent first paragraph
\usepackage{indentfirst}

% Line spacing
\usepackage[onehalfspacing]{setspace}

% Code blocks
\usepackage{minted}
\renewcommand{\MintedPygmentize}{/home/barklan/.local/bin/pygmentize}

\usepackage[colorlinks=true,urlcolor=blue]{hyperref}

\usepackage{subfiles} % Best loaded last in the preamble

\usepackage{ragged2e}

\title{
    Лаборатораная работа 1 \\
    "Численные методы решения нелинейных уравнений"
}
\date{2021-09-24}
    \author{Бузин Глеб Борисович - Б03-907}
\begin{document}

\begin{flushright}
    Бузин Глеб Борисович - СП-19/1
\end{flushright}

\begin{center}
    \huge{Иордания}
\end{center}

\section{Расчет коэффициентов}

\href{https://docs.google.com/spreadsheets/d/1DhepzMyJwSX1xKBgFEJX6ws6Pb4RSUK4NXO9V3fjGvQ/edit?usp=sharing}{Расчеты сделаны в Google Sheets}

\bigskip

\indent
\textbf{Мода}
\begin{align*}
M_0 = x_l + \varDelta x_i \frac{y_i - y_{i-1}}{(y_i - y_{i-1}) + (y_i - y_{i+1})} = 7.5 + (10 - 7.5) \frac{152720 - 148160}{(152720 - 148160) + (152720 - 115860)} = 7.78 тыс долл
\end{align*}

\textbf{Медиана}
\begin{align*}
M_e = x_l + \varDelta x_i \frac{50 - p_{i-1}}{w_i} = 10 + (15 - 10) \frac{50 - 44.3}{23.79} = 11.20
\end{align*}

\textbf{Коэффициент Джини}
\begin{align*}
G = 1 - \sum_{k=1}^n (p_i - p_{i-1})(q_i + q_{i-1}) = 1 - 55.16 = 44.84
\end{align*}

\textbf{Коэффициент Герфиндаля}
\begin{align*}
H = \sum_{i=1}^n \left(\frac{x_i w_i}{\sum x_i w_i}\right)^2 = 0.131
\end{align*}

\textbf{Коэффициент Лоренца}
\begin{align*}
L = \frac{1}{2} \sum |w_i - \frac{x_i w_i}{\sum x_i w_i}| = \frac{0.651}{2} = 0.325
\end{align*}

\textbf{Коэффициент Пирсона}
\begin{align*}
K_{ас} = \frac{\overline{x} - M_0}{\sigma} = \frac{16.15 - 7.78}{24.53} = 0.341 > 0
\end{align*}

Положительная правосторонняя
асимметрия в распределении частот.

\textbf{Центральный момент 3-го порядка}
\begin{align*}
\mu_3 = \frac{\sum (x_i - \overline{x})^3 w_i}{\sum w_i} = \frac{127058}{100} = 127058
\end{align*}

\textbf{Нормированный момент 3-го порядка}
\begin{align*}
r_3 = \frac{\mu_3}{\sigma^3} = \frac{127058}{24.53^3} = 8.60 > 0
\end{align*}

В ряду преобладают варианты, которые больше чем средняя, т.е. ряд положительно асимметричен.

\section{Аналитическая записка}

\noindent
Наибольшее количество домашних хозяйств имеют 7.78 тыс долл США годового располагаемого дохода. Сднее взвешенное значение - 16.15 тыс долл США. Медиана - 11.20 тыс долл США. Межквартильный размах 11.50 тыс долл (разброс 50\% домохозяйств).

Рассчитанный коэффициент Джини для 2020 - 44.48. Для 2019 года он состовлял 39.1 (World Bank estimate). Расслоение общества по доходу достаточно быстро увеличивается. Коэффициент Пирсона больше нуля. Положительная правосторонняя
асимметрия в распределении частот. Нормированный момент 3-го порядка больше нуля. В ряду преобладают варианты, которые больше чем средняя, т.е. ряд положительно асимметричен.

% \begin{figure}[ht]
    % \centering
    % \includegraphics[width=15cm]{01.png}
% \end{figure}


% \begingroup

%     \setlength{\intextsep}{0pt}
%     \setlength{\columnsep}{15pt}

%     \begin{wrapfigure}{r}{0.45\textwidth}
%         \centering
%         \includegraphics[width=\linewidth]{01.png}
%         \caption{Pretty picture}\label{fig:prettypic}
%     \end{wrapfigure}

% \endgroup

% \section{Spacing}
% Just random \\
% The second line is indented. \\[10pt]
% Another line.

% put this before the paragraph to avoid indentation
% \noindent

% Numbered list!

% \setlist{nolistsep}
% \begin{enumerate}[label=\arabic*, font=\bfseries]
%         \item Add teh sdfsfdfs
%         \item More
% \end{enumerate}
% \bigskip

% Description!

% \begin{description}
%     \item[Paladium] Some metal
%     \item[Titanium] Another very useful metal
% \end{description}
% \bigskip

% Tabbing!

% \begin{tabbing}
%     Customer \= Name \hspace*{1.5cm} \= Street \hspace*{1.5cm} \= City \\
%     \> Derek Banas \> 123 Main St \> Pittsburgh \\
% \end{tabbing}
% \bigskip

% \subsection{Tables!}

% \begin{table}[!htbp]
% \centering % this will center the table
% \begin{tabular}{c|c|c}
%     \textbf{Name} & \textbf{Command} & \textbf{Sample Text} \\
%     \hline
%     emphasise & \verb|\emph| & \emph{fucker}
% \end{tabular}
% \caption{Ways to emphasise text}
% \end{table}

% \section{Type emphasis \& Sizing}
% \label{sec:typeemp}
% \itshape italic, \scshape small caps, \upshape upright, \normalfont back to normal

% Get smaller: \normalsize{normal}, \small{small}, \footnotesize{footnote}, \scriptsize{script}, \tiny{tiny}

% Get bigger: \large{large}, \Large{larger}, \LARGE{larger}, \huge{huge}, \Huge{hugest}

% \begin{LARGE}
%     I want to use a big font
% \end{LARGE}

% \normalsize{Back to normal}

% \section{\textsf{Font Families}}

% We can {\sffamily temporarily change} a font family.

% \begin{quote}
%     "I like long walks"
%     - Fred Allen
% \end{quote}


% % With minted

% % \begin{minted}{python}
% %     for i in range(10):
% %         print("Hey!")
% % \end{minted}

\end{document}
