%!TEX program = xelatex
\documentclass[a4paper,11pt]{article}
\usepackage{fontspec}
\usepackage{amsmath}
% \setmonofont{JetBrains Mono}
\usepackage{unicode-math}
\defaultfontfeatures{Scale=MatchLowercase}
\setmainfont[Ligatures=TeX]{Noto Sans}
% \setmathfont{STIX Two Math}
% \usepackage{fullpage}
\usepackage[a4paper,inner=1.7cm,outer=2.7cm,top=2cm,bottom=2cm,bindingoffset=1.2cm]{geometry}
\usepackage{microtype}
\usepackage{graphicx}
\usepackage{pgf}
\usepackage{subfig}
\usepackage{wrapfig}
\usepackage{enumitem}
\usepackage{fancyhdr}
% \usepackage[font=scriptsize]{caption}
\usepackage{index}
\makeindex
\usepackage{parskip}
\usepackage[onehalfspacing]{setspace}
\usepackage{minted}
% \renewcommand{\MintedPygmentize}{/home/barklan/.local/bin/pygmentize}
\usepackage[colorlinks=true,urlcolor=blue]{hyperref}
\usepackage{subfiles} % Best loaded last in the preamble
\usepackage{mathtools}
\renewcommand{\contentsname}{Содержание}

\title{
    Лаборатораная работа 3 \\
    "Численное интегрирование функций"
}
\date{2022-02-01}
\author{Глеб Бузин - Б03-907}
\begin{document}

\maketitle
\newpage
\tableofcontents
\newpage


\section{Теория}

\subsection{Численное интегрирование}

Численное интегрирование - методы вычисления значения интеграла

$$
J=\int ^{b}_{a}f\left( x\right) dx
$$

Самые широко используемые в практических вычислениях - методы прямоугольников, трапеций, Симпсона.
Способ их получения состоит в следующем. Разобьем отрезок интегрирования $[a, b]$ на $N$ элементарных шагов. Точки разбиения $x_n(n = 0,1,...,N); h_n = x_{n+1} - x_n$, так что
$\sum_{n = 0}^{N - 1}h_n = b - a $.
В дальнейшем будем называть $ x_n $ узлами, $h_n$ - шагами интегрирования.
(В частном случае шаг интегрирования может быть постоянным $h = (b - a)/N$.) Искомое значение интеграла представим в виде

\begin{equation}
J=\sum ^{N-1}_{n=0}\int ^{x_{n+1}}_{x_{n}}f\left( x\right) dx= \sum ^{N-1}_{n=0}J_{n},
\end{equation}

где $ J_n = \int ^{x_{n+1}}_{x_{n}}f\left( x\right) dx$.

\subsection{Метод прямоугольников}

Считая $h_n$ малым параметром, заменим $J_n$ в (1) площадью прямоугольника с основанием $h_n$ И высотой $f_{n+1/2} = f(x_n + h_n/2)$. Тогда придем к локальной формуле прямоугольников

$$
\tilde{J}_{n}=h_n f_{n+1/2}
$$

Суммируя в соответствии с (1) приближенные значения по всем элементарным отрезкам, получаем формулу прямоугольников для вычисления приближения $J$:

$$
\tilde{J} = \sum_{n=0}^{N-1} h_n f_{n+1/2}
$$

В частном случае, когда $h_n = h = const$, формула прямоугольников
записывается в виде

$$
\tilde{J} = h \sum_{n=0}^{N-1} f_{n+1/2}
$$

\subsection{Метод трапеций}

На элементарном отрезке $ [x_n, x_{n+1}] $ заменим подынтегральную функцию
интерполяционным полиномом первой степени:

$$
f(x) \approx f_n + \frac{f_{n+1} - f_n}{x_{n+1} - x_n}(x - x_n)
$$

Выполняя интегрирование по отрезку, приходим к локальной формуле
трапеций:

\begin{equation}
\tilde{J_n} = \frac{1}{2} (x_{n+1} - x_n)(f_{n+1} + f_n) = \frac{1}{2} h_n (f_{n+1} + f_n)
\end{equation}

Суммируя (2) по всем отрезкам, получаем формулу трапеций
для вычисления приближения к $J$:

$$
\tilde{J} = \frac{1}{2} \sum_{n=0}^{N-1} h_n (f_n + f_{n+1})
$$

\subsection{Метод Симпсона}

На элементарном отрезке $[x_n, x_{n+1}]$,
привлекая значение функции в середине, заменим подынтегральную функцию интерполяционным полиномом второй степени

\begin{equation}
\begin{split}
f(x) \approx P_2(x) = f_{n+1/2} + \frac{f_{n+1} - f_n}{h_n} (x - \frac{x_{n+1} + x_n}{2}) + \\
+ \frac{f_{n+1} - 2f_{n+1/2} + f_n}{2(h_n/2)^2} (x - \frac{x_{n+1} + x_n}{2})^2
\end{split}
\end{equation}

Вычисляя интеграл от полинома по отрезку $[x_n, x_{n+1}]$, Приходим к
локальной формуле Симпсона

\begin{equation}
\tilde{J_n} = \frac{h_n}{6} (f_n + 4f_{n+1/2} + f_{n+1})
\end{equation}

Суммируя (4) по всем отрезкам, получаем формулу Симпсона для
вычисления приближения к $J$:

\begin{equation}
\tilde{J} = \frac{1}{6} \sum_{n=0}^{N-1} h_n (f_n + 4f_{n+1/2} + f_{n+1})
\end{equation}

\subsection{Погрешность квадратурных формул}

Для рассмотренных квадратурных формул оценки погрешности имеют вид:

Формула прямоугольников (левых и правых) - $ |\tilde{J} - J| \leq \frac{1}{2}(b - a)M_1 \overline{h} $

Формула прямоугольников (средних) - $ |\tilde{J} - J| \leq \frac{1}{24}(b - a)M_2 \overline{h}^2 $

Формула трапеций - $ |\tilde{J} - J| \leq \frac{1}{12}(b - a)M_2 \overline{h}^2 $

Формула Симпсона - $ |\tilde{J} - J| \leq \frac{1}{180}(b - a)M_4 \overline{h}^4 $

\section{Задача}

Используя метод численного интегрирования вычислить интеграл от заданной функции $f(x)$ по заданному интервалу $[a,b]$.

$$
\int ^{4.14159}_{1}\ln \left( x\right) \left| \cos \left( 128x\right) \right| dx
$$

\begin{figure}[ht]
    %% Creator: Matplotlib, PGF backend
%%
%% To include the figure in your LaTeX document, write
%%   \input{<filename>.pgf}
%%
%% Make sure the required packages are loaded in your preamble
%%   \usepackage{pgf}
%%
%% Also ensure that all the required font packages are loaded; for instance,
%% the lmodern package is sometimes necessary when using math font.
%%   \usepackage{lmodern}
%%
%% Figures using additional raster images can only be included by \input if
%% they are in the same directory as the main LaTeX file. For loading figures
%% from other directories you can use the `import` package
%%   \usepackage{import}
%%
%% and then include the figures with
%%   \import{<path to file>}{<filename>.pgf}
%%
%% Matplotlib used the following preamble
%%
\begingroup%
\makeatletter%
\begin{pgfpicture}%
\pgfpathrectangle{\pgfpointorigin}{\pgfqpoint{6.479436in}{4.076389in}}%
\pgfusepath{use as bounding box, clip}%
\begin{pgfscope}%
\pgfsetbuttcap%
\pgfsetmiterjoin%
\pgfsetlinewidth{0.000000pt}%
\definecolor{currentstroke}{rgb}{0.000000,0.000000,0.000000}%
\pgfsetstrokecolor{currentstroke}%
\pgfsetstrokeopacity{0.000000}%
\pgfsetdash{}{0pt}%
\pgfpathmoveto{\pgfqpoint{0.000000in}{0.000000in}}%
\pgfpathlineto{\pgfqpoint{6.479436in}{0.000000in}}%
\pgfpathlineto{\pgfqpoint{6.479436in}{4.076389in}}%
\pgfpathlineto{\pgfqpoint{0.000000in}{4.076389in}}%
\pgfpathlineto{\pgfqpoint{0.000000in}{0.000000in}}%
\pgfpathclose%
\pgfusepath{}%
\end{pgfscope}%
\begin{pgfscope}%
\pgfsetbuttcap%
\pgfsetmiterjoin%
\pgfsetlinewidth{0.000000pt}%
\definecolor{currentstroke}{rgb}{0.000000,0.000000,0.000000}%
\pgfsetstrokecolor{currentstroke}%
\pgfsetstrokeopacity{0.000000}%
\pgfsetdash{}{0pt}%
\pgfpathmoveto{\pgfqpoint{0.279436in}{0.226389in}}%
\pgfpathlineto{\pgfqpoint{6.479436in}{0.226389in}}%
\pgfpathlineto{\pgfqpoint{6.479436in}{4.076389in}}%
\pgfpathlineto{\pgfqpoint{0.279436in}{4.076389in}}%
\pgfpathlineto{\pgfqpoint{0.279436in}{0.226389in}}%
\pgfpathclose%
\pgfusepath{}%
\end{pgfscope}%
\begin{pgfscope}%
\pgfpathrectangle{\pgfqpoint{0.279436in}{0.226389in}}{\pgfqpoint{6.200000in}{3.850000in}}%
\pgfusepath{clip}%
\pgfsetroundcap%
\pgfsetroundjoin%
\pgfsetlinewidth{0.803000pt}%
\definecolor{currentstroke}{rgb}{0.800000,0.800000,0.800000}%
\pgfsetstrokecolor{currentstroke}%
\pgfsetdash{}{0pt}%
\pgfpathmoveto{\pgfqpoint{0.561254in}{0.226389in}}%
\pgfpathlineto{\pgfqpoint{0.561254in}{4.076389in}}%
\pgfusepath{stroke}%
\end{pgfscope}%
\begin{pgfscope}%
\definecolor{textcolor}{rgb}{0.501961,0.501961,0.501961}%
\pgfsetstrokecolor{textcolor}%
\pgfsetfillcolor{textcolor}%
\pgftext[x=0.561254in,y=0.111111in,,top]{\color{textcolor}\rmfamily\fontsize{8.800000}{10.560000}\selectfont \(\displaystyle {1.0}\)}%
\end{pgfscope}%
\begin{pgfscope}%
\pgfpathrectangle{\pgfqpoint{0.279436in}{0.226389in}}{\pgfqpoint{6.200000in}{3.850000in}}%
\pgfusepath{clip}%
\pgfsetroundcap%
\pgfsetroundjoin%
\pgfsetlinewidth{0.803000pt}%
\definecolor{currentstroke}{rgb}{0.800000,0.800000,0.800000}%
\pgfsetstrokecolor{currentstroke}%
\pgfsetdash{}{0pt}%
\pgfpathmoveto{\pgfqpoint{1.458310in}{0.226389in}}%
\pgfpathlineto{\pgfqpoint{1.458310in}{4.076389in}}%
\pgfusepath{stroke}%
\end{pgfscope}%
\begin{pgfscope}%
\definecolor{textcolor}{rgb}{0.501961,0.501961,0.501961}%
\pgfsetstrokecolor{textcolor}%
\pgfsetfillcolor{textcolor}%
\pgftext[x=1.458310in,y=0.111111in,,top]{\color{textcolor}\rmfamily\fontsize{8.800000}{10.560000}\selectfont \(\displaystyle {1.5}\)}%
\end{pgfscope}%
\begin{pgfscope}%
\pgfpathrectangle{\pgfqpoint{0.279436in}{0.226389in}}{\pgfqpoint{6.200000in}{3.850000in}}%
\pgfusepath{clip}%
\pgfsetroundcap%
\pgfsetroundjoin%
\pgfsetlinewidth{0.803000pt}%
\definecolor{currentstroke}{rgb}{0.800000,0.800000,0.800000}%
\pgfsetstrokecolor{currentstroke}%
\pgfsetdash{}{0pt}%
\pgfpathmoveto{\pgfqpoint{2.355366in}{0.226389in}}%
\pgfpathlineto{\pgfqpoint{2.355366in}{4.076389in}}%
\pgfusepath{stroke}%
\end{pgfscope}%
\begin{pgfscope}%
\definecolor{textcolor}{rgb}{0.501961,0.501961,0.501961}%
\pgfsetstrokecolor{textcolor}%
\pgfsetfillcolor{textcolor}%
\pgftext[x=2.355366in,y=0.111111in,,top]{\color{textcolor}\rmfamily\fontsize{8.800000}{10.560000}\selectfont \(\displaystyle {2.0}\)}%
\end{pgfscope}%
\begin{pgfscope}%
\pgfpathrectangle{\pgfqpoint{0.279436in}{0.226389in}}{\pgfqpoint{6.200000in}{3.850000in}}%
\pgfusepath{clip}%
\pgfsetroundcap%
\pgfsetroundjoin%
\pgfsetlinewidth{0.803000pt}%
\definecolor{currentstroke}{rgb}{0.800000,0.800000,0.800000}%
\pgfsetstrokecolor{currentstroke}%
\pgfsetdash{}{0pt}%
\pgfpathmoveto{\pgfqpoint{3.252422in}{0.226389in}}%
\pgfpathlineto{\pgfqpoint{3.252422in}{4.076389in}}%
\pgfusepath{stroke}%
\end{pgfscope}%
\begin{pgfscope}%
\definecolor{textcolor}{rgb}{0.501961,0.501961,0.501961}%
\pgfsetstrokecolor{textcolor}%
\pgfsetfillcolor{textcolor}%
\pgftext[x=3.252422in,y=0.111111in,,top]{\color{textcolor}\rmfamily\fontsize{8.800000}{10.560000}\selectfont \(\displaystyle {2.5}\)}%
\end{pgfscope}%
\begin{pgfscope}%
\pgfpathrectangle{\pgfqpoint{0.279436in}{0.226389in}}{\pgfqpoint{6.200000in}{3.850000in}}%
\pgfusepath{clip}%
\pgfsetroundcap%
\pgfsetroundjoin%
\pgfsetlinewidth{0.803000pt}%
\definecolor{currentstroke}{rgb}{0.800000,0.800000,0.800000}%
\pgfsetstrokecolor{currentstroke}%
\pgfsetdash{}{0pt}%
\pgfpathmoveto{\pgfqpoint{4.149477in}{0.226389in}}%
\pgfpathlineto{\pgfqpoint{4.149477in}{4.076389in}}%
\pgfusepath{stroke}%
\end{pgfscope}%
\begin{pgfscope}%
\definecolor{textcolor}{rgb}{0.501961,0.501961,0.501961}%
\pgfsetstrokecolor{textcolor}%
\pgfsetfillcolor{textcolor}%
\pgftext[x=4.149477in,y=0.111111in,,top]{\color{textcolor}\rmfamily\fontsize{8.800000}{10.560000}\selectfont \(\displaystyle {3.0}\)}%
\end{pgfscope}%
\begin{pgfscope}%
\pgfpathrectangle{\pgfqpoint{0.279436in}{0.226389in}}{\pgfqpoint{6.200000in}{3.850000in}}%
\pgfusepath{clip}%
\pgfsetroundcap%
\pgfsetroundjoin%
\pgfsetlinewidth{0.803000pt}%
\definecolor{currentstroke}{rgb}{0.800000,0.800000,0.800000}%
\pgfsetstrokecolor{currentstroke}%
\pgfsetdash{}{0pt}%
\pgfpathmoveto{\pgfqpoint{5.046533in}{0.226389in}}%
\pgfpathlineto{\pgfqpoint{5.046533in}{4.076389in}}%
\pgfusepath{stroke}%
\end{pgfscope}%
\begin{pgfscope}%
\definecolor{textcolor}{rgb}{0.501961,0.501961,0.501961}%
\pgfsetstrokecolor{textcolor}%
\pgfsetfillcolor{textcolor}%
\pgftext[x=5.046533in,y=0.111111in,,top]{\color{textcolor}\rmfamily\fontsize{8.800000}{10.560000}\selectfont \(\displaystyle {3.5}\)}%
\end{pgfscope}%
\begin{pgfscope}%
\pgfpathrectangle{\pgfqpoint{0.279436in}{0.226389in}}{\pgfqpoint{6.200000in}{3.850000in}}%
\pgfusepath{clip}%
\pgfsetroundcap%
\pgfsetroundjoin%
\pgfsetlinewidth{0.803000pt}%
\definecolor{currentstroke}{rgb}{0.800000,0.800000,0.800000}%
\pgfsetstrokecolor{currentstroke}%
\pgfsetdash{}{0pt}%
\pgfpathmoveto{\pgfqpoint{5.943589in}{0.226389in}}%
\pgfpathlineto{\pgfqpoint{5.943589in}{4.076389in}}%
\pgfusepath{stroke}%
\end{pgfscope}%
\begin{pgfscope}%
\definecolor{textcolor}{rgb}{0.501961,0.501961,0.501961}%
\pgfsetstrokecolor{textcolor}%
\pgfsetfillcolor{textcolor}%
\pgftext[x=5.943589in,y=0.111111in,,top]{\color{textcolor}\rmfamily\fontsize{8.800000}{10.560000}\selectfont \(\displaystyle {4.0}\)}%
\end{pgfscope}%
\begin{pgfscope}%
\pgfpathrectangle{\pgfqpoint{0.279436in}{0.226389in}}{\pgfqpoint{6.200000in}{3.850000in}}%
\pgfusepath{clip}%
\pgfsetroundcap%
\pgfsetroundjoin%
\pgfsetlinewidth{0.803000pt}%
\definecolor{currentstroke}{rgb}{0.800000,0.800000,0.800000}%
\pgfsetstrokecolor{currentstroke}%
\pgfsetdash{}{0pt}%
\pgfpathmoveto{\pgfqpoint{0.279436in}{0.401389in}}%
\pgfpathlineto{\pgfqpoint{6.479436in}{0.401389in}}%
\pgfusepath{stroke}%
\end{pgfscope}%
\begin{pgfscope}%
\definecolor{textcolor}{rgb}{0.501961,0.501961,0.501961}%
\pgfsetstrokecolor{textcolor}%
\pgfsetfillcolor{textcolor}%
\pgftext[x=-0.000000in, y=0.357986in, left, base]{\color{textcolor}\rmfamily\fontsize{8.800000}{10.560000}\selectfont \(\displaystyle {0.0}\)}%
\end{pgfscope}%
\begin{pgfscope}%
\pgfpathrectangle{\pgfqpoint{0.279436in}{0.226389in}}{\pgfqpoint{6.200000in}{3.850000in}}%
\pgfusepath{clip}%
\pgfsetroundcap%
\pgfsetroundjoin%
\pgfsetlinewidth{0.803000pt}%
\definecolor{currentstroke}{rgb}{0.800000,0.800000,0.800000}%
\pgfsetstrokecolor{currentstroke}%
\pgfsetdash{}{0pt}%
\pgfpathmoveto{\pgfqpoint{0.279436in}{0.895507in}}%
\pgfpathlineto{\pgfqpoint{6.479436in}{0.895507in}}%
\pgfusepath{stroke}%
\end{pgfscope}%
\begin{pgfscope}%
\definecolor{textcolor}{rgb}{0.501961,0.501961,0.501961}%
\pgfsetstrokecolor{textcolor}%
\pgfsetfillcolor{textcolor}%
\pgftext[x=-0.000000in, y=0.852104in, left, base]{\color{textcolor}\rmfamily\fontsize{8.800000}{10.560000}\selectfont \(\displaystyle {0.2}\)}%
\end{pgfscope}%
\begin{pgfscope}%
\pgfpathrectangle{\pgfqpoint{0.279436in}{0.226389in}}{\pgfqpoint{6.200000in}{3.850000in}}%
\pgfusepath{clip}%
\pgfsetroundcap%
\pgfsetroundjoin%
\pgfsetlinewidth{0.803000pt}%
\definecolor{currentstroke}{rgb}{0.800000,0.800000,0.800000}%
\pgfsetstrokecolor{currentstroke}%
\pgfsetdash{}{0pt}%
\pgfpathmoveto{\pgfqpoint{0.279436in}{1.389625in}}%
\pgfpathlineto{\pgfqpoint{6.479436in}{1.389625in}}%
\pgfusepath{stroke}%
\end{pgfscope}%
\begin{pgfscope}%
\definecolor{textcolor}{rgb}{0.501961,0.501961,0.501961}%
\pgfsetstrokecolor{textcolor}%
\pgfsetfillcolor{textcolor}%
\pgftext[x=-0.000000in, y=1.346223in, left, base]{\color{textcolor}\rmfamily\fontsize{8.800000}{10.560000}\selectfont \(\displaystyle {0.4}\)}%
\end{pgfscope}%
\begin{pgfscope}%
\pgfpathrectangle{\pgfqpoint{0.279436in}{0.226389in}}{\pgfqpoint{6.200000in}{3.850000in}}%
\pgfusepath{clip}%
\pgfsetroundcap%
\pgfsetroundjoin%
\pgfsetlinewidth{0.803000pt}%
\definecolor{currentstroke}{rgb}{0.800000,0.800000,0.800000}%
\pgfsetstrokecolor{currentstroke}%
\pgfsetdash{}{0pt}%
\pgfpathmoveto{\pgfqpoint{0.279436in}{1.883744in}}%
\pgfpathlineto{\pgfqpoint{6.479436in}{1.883744in}}%
\pgfusepath{stroke}%
\end{pgfscope}%
\begin{pgfscope}%
\definecolor{textcolor}{rgb}{0.501961,0.501961,0.501961}%
\pgfsetstrokecolor{textcolor}%
\pgfsetfillcolor{textcolor}%
\pgftext[x=-0.000000in, y=1.840341in, left, base]{\color{textcolor}\rmfamily\fontsize{8.800000}{10.560000}\selectfont \(\displaystyle {0.6}\)}%
\end{pgfscope}%
\begin{pgfscope}%
\pgfpathrectangle{\pgfqpoint{0.279436in}{0.226389in}}{\pgfqpoint{6.200000in}{3.850000in}}%
\pgfusepath{clip}%
\pgfsetroundcap%
\pgfsetroundjoin%
\pgfsetlinewidth{0.803000pt}%
\definecolor{currentstroke}{rgb}{0.800000,0.800000,0.800000}%
\pgfsetstrokecolor{currentstroke}%
\pgfsetdash{}{0pt}%
\pgfpathmoveto{\pgfqpoint{0.279436in}{2.377862in}}%
\pgfpathlineto{\pgfqpoint{6.479436in}{2.377862in}}%
\pgfusepath{stroke}%
\end{pgfscope}%
\begin{pgfscope}%
\definecolor{textcolor}{rgb}{0.501961,0.501961,0.501961}%
\pgfsetstrokecolor{textcolor}%
\pgfsetfillcolor{textcolor}%
\pgftext[x=-0.000000in, y=2.334459in, left, base]{\color{textcolor}\rmfamily\fontsize{8.800000}{10.560000}\selectfont \(\displaystyle {0.8}\)}%
\end{pgfscope}%
\begin{pgfscope}%
\pgfpathrectangle{\pgfqpoint{0.279436in}{0.226389in}}{\pgfqpoint{6.200000in}{3.850000in}}%
\pgfusepath{clip}%
\pgfsetroundcap%
\pgfsetroundjoin%
\pgfsetlinewidth{0.803000pt}%
\definecolor{currentstroke}{rgb}{0.800000,0.800000,0.800000}%
\pgfsetstrokecolor{currentstroke}%
\pgfsetdash{}{0pt}%
\pgfpathmoveto{\pgfqpoint{0.279436in}{2.871980in}}%
\pgfpathlineto{\pgfqpoint{6.479436in}{2.871980in}}%
\pgfusepath{stroke}%
\end{pgfscope}%
\begin{pgfscope}%
\definecolor{textcolor}{rgb}{0.501961,0.501961,0.501961}%
\pgfsetstrokecolor{textcolor}%
\pgfsetfillcolor{textcolor}%
\pgftext[x=-0.000000in, y=2.828577in, left, base]{\color{textcolor}\rmfamily\fontsize{8.800000}{10.560000}\selectfont \(\displaystyle {1.0}\)}%
\end{pgfscope}%
\begin{pgfscope}%
\pgfpathrectangle{\pgfqpoint{0.279436in}{0.226389in}}{\pgfqpoint{6.200000in}{3.850000in}}%
\pgfusepath{clip}%
\pgfsetroundcap%
\pgfsetroundjoin%
\pgfsetlinewidth{0.803000pt}%
\definecolor{currentstroke}{rgb}{0.800000,0.800000,0.800000}%
\pgfsetstrokecolor{currentstroke}%
\pgfsetdash{}{0pt}%
\pgfpathmoveto{\pgfqpoint{0.279436in}{3.366099in}}%
\pgfpathlineto{\pgfqpoint{6.479436in}{3.366099in}}%
\pgfusepath{stroke}%
\end{pgfscope}%
\begin{pgfscope}%
\definecolor{textcolor}{rgb}{0.501961,0.501961,0.501961}%
\pgfsetstrokecolor{textcolor}%
\pgfsetfillcolor{textcolor}%
\pgftext[x=-0.000000in, y=3.322696in, left, base]{\color{textcolor}\rmfamily\fontsize{8.800000}{10.560000}\selectfont \(\displaystyle {1.2}\)}%
\end{pgfscope}%
\begin{pgfscope}%
\pgfpathrectangle{\pgfqpoint{0.279436in}{0.226389in}}{\pgfqpoint{6.200000in}{3.850000in}}%
\pgfusepath{clip}%
\pgfsetroundcap%
\pgfsetroundjoin%
\pgfsetlinewidth{0.803000pt}%
\definecolor{currentstroke}{rgb}{0.800000,0.800000,0.800000}%
\pgfsetstrokecolor{currentstroke}%
\pgfsetdash{}{0pt}%
\pgfpathmoveto{\pgfqpoint{0.279436in}{3.860217in}}%
\pgfpathlineto{\pgfqpoint{6.479436in}{3.860217in}}%
\pgfusepath{stroke}%
\end{pgfscope}%
\begin{pgfscope}%
\definecolor{textcolor}{rgb}{0.501961,0.501961,0.501961}%
\pgfsetstrokecolor{textcolor}%
\pgfsetfillcolor{textcolor}%
\pgftext[x=-0.000000in, y=3.816814in, left, base]{\color{textcolor}\rmfamily\fontsize{8.800000}{10.560000}\selectfont \(\displaystyle {1.4}\)}%
\end{pgfscope}%
\begin{pgfscope}%
\pgfpathrectangle{\pgfqpoint{0.279436in}{0.226389in}}{\pgfqpoint{6.200000in}{3.850000in}}%
\pgfusepath{clip}%
\pgfsetroundcap%
\pgfsetroundjoin%
\pgfsetlinewidth{1.204500pt}%
\definecolor{currentstroke}{rgb}{0.007843,0.243137,1.000000}%
\pgfsetstrokecolor{currentstroke}%
\pgfsetdash{}{0pt}%
\pgfpathmoveto{\pgfqpoint{0.561254in}{0.401389in}}%
\pgfpathlineto{\pgfqpoint{0.566475in}{0.407906in}}%
\pgfpathlineto{\pgfqpoint{0.577796in}{0.422489in}}%
\pgfpathlineto{\pgfqpoint{0.581474in}{0.423642in}}%
\pgfpathlineto{\pgfqpoint{0.584524in}{0.422285in}}%
\pgfpathlineto{\pgfqpoint{0.587878in}{0.418084in}}%
\pgfpathlineto{\pgfqpoint{0.592041in}{0.408899in}}%
\pgfpathlineto{\pgfqpoint{0.594553in}{0.401408in}}%
\pgfpathlineto{\pgfqpoint{0.595198in}{0.403503in}}%
\pgfpathlineto{\pgfqpoint{0.603075in}{0.433889in}}%
\pgfpathlineto{\pgfqpoint{0.612619in}{0.468369in}}%
\pgfpathlineto{\pgfqpoint{0.617033in}{0.476989in}}%
\pgfpathlineto{\pgfqpoint{0.619921in}{0.478630in}}%
\pgfpathlineto{\pgfqpoint{0.622146in}{0.477422in}}%
\pgfpathlineto{\pgfqpoint{0.624783in}{0.473035in}}%
\pgfpathlineto{\pgfqpoint{0.628228in}{0.462407in}}%
\pgfpathlineto{\pgfqpoint{0.632785in}{0.440273in}}%
\pgfpathlineto{\pgfqpoint{0.638598in}{0.401430in}}%
\pgfpathlineto{\pgfqpoint{0.639477in}{0.408039in}}%
\pgfpathlineto{\pgfqpoint{0.654960in}{0.517092in}}%
\pgfpathlineto{\pgfqpoint{0.659320in}{0.532314in}}%
\pgfpathlineto{\pgfqpoint{0.662119in}{0.535739in}}%
\pgfpathlineto{\pgfqpoint{0.663913in}{0.535059in}}%
\pgfpathlineto{\pgfqpoint{0.666012in}{0.531306in}}%
\pgfpathlineto{\pgfqpoint{0.668901in}{0.520888in}}%
\pgfpathlineto{\pgfqpoint{0.672812in}{0.497427in}}%
\pgfpathlineto{\pgfqpoint{0.678176in}{0.450087in}}%
\pgfpathlineto{\pgfqpoint{0.682626in}{0.401402in}}%
\pgfpathlineto{\pgfqpoint{0.683289in}{0.409073in}}%
\pgfpathlineto{\pgfqpoint{0.696907in}{0.554720in}}%
\pgfpathlineto{\pgfqpoint{0.701697in}{0.583491in}}%
\pgfpathlineto{\pgfqpoint{0.704855in}{0.591598in}}%
\pgfpathlineto{\pgfqpoint{0.706667in}{0.591927in}}%
\pgfpathlineto{\pgfqpoint{0.708317in}{0.589384in}}%
\pgfpathlineto{\pgfqpoint{0.710703in}{0.580878in}}%
\pgfpathlineto{\pgfqpoint{0.714040in}{0.559604in}}%
\pgfpathlineto{\pgfqpoint{0.718580in}{0.514597in}}%
\pgfpathlineto{\pgfqpoint{0.725218in}{0.423593in}}%
\pgfpathlineto{\pgfqpoint{0.726653in}{0.401509in}}%
\pgfpathlineto{\pgfqpoint{0.727209in}{0.409939in}}%
\pgfpathlineto{\pgfqpoint{0.739983in}{0.592298in}}%
\pgfpathlineto{\pgfqpoint{0.744953in}{0.633769in}}%
\pgfpathlineto{\pgfqpoint{0.748254in}{0.646232in}}%
\pgfpathlineto{\pgfqpoint{0.750120in}{0.647362in}}%
\pgfpathlineto{\pgfqpoint{0.750156in}{0.647340in}}%
\pgfpathlineto{\pgfqpoint{0.751591in}{0.645140in}}%
\pgfpathlineto{\pgfqpoint{0.753744in}{0.636911in}}%
\pgfpathlineto{\pgfqpoint{0.756830in}{0.614985in}}%
\pgfpathlineto{\pgfqpoint{0.761064in}{0.566865in}}%
\pgfpathlineto{\pgfqpoint{0.767074in}{0.469928in}}%
\pgfpathlineto{\pgfqpoint{0.770698in}{0.401458in}}%
\pgfpathlineto{\pgfqpoint{0.771291in}{0.413008in}}%
\pgfpathlineto{\pgfqpoint{0.783473in}{0.629438in}}%
\pgfpathlineto{\pgfqpoint{0.788532in}{0.683062in}}%
\pgfpathlineto{\pgfqpoint{0.791923in}{0.699737in}}%
\pgfpathlineto{\pgfqpoint{0.793771in}{0.701605in}}%
\pgfpathlineto{\pgfqpoint{0.793878in}{0.701553in}}%
\pgfpathlineto{\pgfqpoint{0.795170in}{0.699525in}}%
\pgfpathlineto{\pgfqpoint{0.797162in}{0.691344in}}%
\pgfpathlineto{\pgfqpoint{0.800068in}{0.668577in}}%
\pgfpathlineto{\pgfqpoint{0.804087in}{0.617279in}}%
\pgfpathlineto{\pgfqpoint{0.809738in}{0.513097in}}%
\pgfpathlineto{\pgfqpoint{0.814726in}{0.401460in}}%
\pgfpathlineto{\pgfqpoint{0.815408in}{0.417232in}}%
\pgfpathlineto{\pgfqpoint{0.827159in}{0.666033in}}%
\pgfpathlineto{\pgfqpoint{0.832272in}{0.731406in}}%
\pgfpathlineto{\pgfqpoint{0.835717in}{0.752130in}}%
\pgfpathlineto{\pgfqpoint{0.837583in}{0.754674in}}%
\pgfpathlineto{\pgfqpoint{0.837709in}{0.754620in}}%
\pgfpathlineto{\pgfqpoint{0.838893in}{0.752707in}}%
\pgfpathlineto{\pgfqpoint{0.840776in}{0.744426in}}%
\pgfpathlineto{\pgfqpoint{0.843557in}{0.720628in}}%
\pgfpathlineto{\pgfqpoint{0.847433in}{0.665807in}}%
\pgfpathlineto{\pgfqpoint{0.852851in}{0.553718in}}%
\pgfpathlineto{\pgfqpoint{0.858771in}{0.401612in}}%
\pgfpathlineto{\pgfqpoint{0.859507in}{0.421542in}}%
\pgfpathlineto{\pgfqpoint{0.870953in}{0.702016in}}%
\pgfpathlineto{\pgfqpoint{0.876103in}{0.778808in}}%
\pgfpathlineto{\pgfqpoint{0.879565in}{0.803385in}}%
\pgfpathlineto{\pgfqpoint{0.881485in}{0.806631in}}%
\pgfpathlineto{\pgfqpoint{0.881593in}{0.806588in}}%
\pgfpathlineto{\pgfqpoint{0.882687in}{0.804787in}}%
\pgfpathlineto{\pgfqpoint{0.884481in}{0.796460in}}%
\pgfpathlineto{\pgfqpoint{0.887172in}{0.771637in}}%
\pgfpathlineto{\pgfqpoint{0.890922in}{0.713731in}}%
\pgfpathlineto{\pgfqpoint{0.896161in}{0.594201in}}%
\pgfpathlineto{\pgfqpoint{0.902799in}{0.401439in}}%
\pgfpathlineto{\pgfqpoint{0.903624in}{0.426830in}}%
\pgfpathlineto{\pgfqpoint{0.914819in}{0.737513in}}%
\pgfpathlineto{\pgfqpoint{0.920004in}{0.825491in}}%
\pgfpathlineto{\pgfqpoint{0.923485in}{0.853698in}}%
\pgfpathlineto{\pgfqpoint{0.925405in}{0.857533in}}%
\pgfpathlineto{\pgfqpoint{0.925530in}{0.857486in}}%
\pgfpathlineto{\pgfqpoint{0.926571in}{0.855682in}}%
\pgfpathlineto{\pgfqpoint{0.928293in}{0.847151in}}%
\pgfpathlineto{\pgfqpoint{0.930913in}{0.821115in}}%
\pgfpathlineto{\pgfqpoint{0.934591in}{0.759359in}}%
\pgfpathlineto{\pgfqpoint{0.939722in}{0.631099in}}%
\pgfpathlineto{\pgfqpoint{0.946826in}{0.401560in}}%
\pgfpathlineto{\pgfqpoint{0.947706in}{0.431425in}}%
\pgfpathlineto{\pgfqpoint{0.958739in}{0.772692in}}%
\pgfpathlineto{\pgfqpoint{0.963924in}{0.871088in}}%
\pgfpathlineto{\pgfqpoint{0.967423in}{0.902969in}}%
\pgfpathlineto{\pgfqpoint{0.969360in}{0.907412in}}%
\pgfpathlineto{\pgfqpoint{0.969468in}{0.907375in}}%
\pgfpathlineto{\pgfqpoint{0.970455in}{0.905647in}}%
\pgfpathlineto{\pgfqpoint{0.972123in}{0.897004in}}%
\pgfpathlineto{\pgfqpoint{0.974671in}{0.870129in}}%
\pgfpathlineto{\pgfqpoint{0.978277in}{0.805247in}}%
\pgfpathlineto{\pgfqpoint{0.983283in}{0.670175in}}%
\pgfpathlineto{\pgfqpoint{0.990872in}{0.401628in}}%
\pgfpathlineto{\pgfqpoint{0.991805in}{0.436973in}}%
\pgfpathlineto{\pgfqpoint{1.002677in}{0.807045in}}%
\pgfpathlineto{\pgfqpoint{1.007880in}{0.915980in}}%
\pgfpathlineto{\pgfqpoint{1.011397in}{0.951358in}}%
\pgfpathlineto{\pgfqpoint{1.013334in}{0.956308in}}%
\pgfpathlineto{\pgfqpoint{1.013460in}{0.956263in}}%
\pgfpathlineto{\pgfqpoint{1.014411in}{0.954477in}}%
\pgfpathlineto{\pgfqpoint{1.016043in}{0.945442in}}%
\pgfpathlineto{\pgfqpoint{1.018555in}{0.917020in}}%
\pgfpathlineto{\pgfqpoint{1.022107in}{0.848252in}}%
\pgfpathlineto{\pgfqpoint{1.027059in}{0.704003in}}%
\pgfpathlineto{\pgfqpoint{1.034899in}{0.401401in}}%
\pgfpathlineto{\pgfqpoint{1.035886in}{0.442167in}}%
\pgfpathlineto{\pgfqpoint{1.046633in}{0.840745in}}%
\pgfpathlineto{\pgfqpoint{1.051836in}{0.959776in}}%
\pgfpathlineto{\pgfqpoint{1.055370in}{0.998749in}}%
\pgfpathlineto{\pgfqpoint{1.057326in}{1.004258in}}%
\pgfpathlineto{\pgfqpoint{1.057433in}{1.004221in}}%
\pgfpathlineto{\pgfqpoint{1.058330in}{1.002527in}}%
\pgfpathlineto{\pgfqpoint{1.059909in}{0.993543in}}%
\pgfpathlineto{\pgfqpoint{1.062367in}{0.964513in}}%
\pgfpathlineto{\pgfqpoint{1.065866in}{0.893091in}}%
\pgfpathlineto{\pgfqpoint{1.070728in}{0.742629in}}%
\pgfpathlineto{\pgfqpoint{1.078407in}{0.424921in}}%
\pgfpathlineto{\pgfqpoint{1.078927in}{0.401698in}}%
\pgfpathlineto{\pgfqpoint{1.079483in}{0.425949in}}%
\pgfpathlineto{\pgfqpoint{1.090607in}{0.873956in}}%
\pgfpathlineto{\pgfqpoint{1.095827in}{1.003137in}}%
\pgfpathlineto{\pgfqpoint{1.099362in}{1.045290in}}%
\pgfpathlineto{\pgfqpoint{1.101317in}{1.051299in}}%
\pgfpathlineto{\pgfqpoint{1.101425in}{1.051263in}}%
\pgfpathlineto{\pgfqpoint{1.102304in}{1.049533in}}%
\pgfpathlineto{\pgfqpoint{1.103865in}{1.040158in}}%
\pgfpathlineto{\pgfqpoint{1.106305in}{1.009557in}}%
\pgfpathlineto{\pgfqpoint{1.109768in}{0.934305in}}%
\pgfpathlineto{\pgfqpoint{1.114594in}{0.774993in}}%
\pgfpathlineto{\pgfqpoint{1.122165in}{0.439870in}}%
\pgfpathlineto{\pgfqpoint{1.122972in}{0.401600in}}%
\pgfpathlineto{\pgfqpoint{1.123511in}{0.427456in}}%
\pgfpathlineto{\pgfqpoint{1.134580in}{0.906232in}}%
\pgfpathlineto{\pgfqpoint{1.139801in}{1.045231in}}%
\pgfpathlineto{\pgfqpoint{1.143336in}{1.090798in}}%
\pgfpathlineto{\pgfqpoint{1.145327in}{1.097462in}}%
\pgfpathlineto{\pgfqpoint{1.145417in}{1.097432in}}%
\pgfpathlineto{\pgfqpoint{1.146260in}{1.095756in}}%
\pgfpathlineto{\pgfqpoint{1.147767in}{1.086484in}}%
\pgfpathlineto{\pgfqpoint{1.150153in}{1.055545in}}%
\pgfpathlineto{\pgfqpoint{1.153562in}{0.978319in}}%
\pgfpathlineto{\pgfqpoint{1.158298in}{0.814315in}}%
\pgfpathlineto{\pgfqpoint{1.165618in}{0.472083in}}%
\pgfpathlineto{\pgfqpoint{1.167000in}{0.401504in}}%
\pgfpathlineto{\pgfqpoint{1.167556in}{0.429807in}}%
\pgfpathlineto{\pgfqpoint{1.178572in}{0.938252in}}%
\pgfpathlineto{\pgfqpoint{1.183793in}{1.086740in}}%
\pgfpathlineto{\pgfqpoint{1.187345in}{1.135670in}}%
\pgfpathlineto{\pgfqpoint{1.189319in}{1.142784in}}%
\pgfpathlineto{\pgfqpoint{1.189426in}{1.142750in}}%
\pgfpathlineto{\pgfqpoint{1.190252in}{1.141036in}}%
\pgfpathlineto{\pgfqpoint{1.191741in}{1.131441in}}%
\pgfpathlineto{\pgfqpoint{1.194109in}{1.099160in}}%
\pgfpathlineto{\pgfqpoint{1.197500in}{1.018160in}}%
\pgfpathlineto{\pgfqpoint{1.202218in}{0.845467in}}%
\pgfpathlineto{\pgfqpoint{1.209484in}{0.485585in}}%
\pgfpathlineto{\pgfqpoint{1.211027in}{0.401872in}}%
\pgfpathlineto{\pgfqpoint{1.211584in}{0.431220in}}%
\pgfpathlineto{\pgfqpoint{1.222563in}{0.969476in}}%
\pgfpathlineto{\pgfqpoint{1.227802in}{1.127764in}}%
\pgfpathlineto{\pgfqpoint{1.231355in}{1.179711in}}%
\pgfpathlineto{\pgfqpoint{1.233328in}{1.187288in}}%
\pgfpathlineto{\pgfqpoint{1.233436in}{1.187254in}}%
\pgfpathlineto{\pgfqpoint{1.234225in}{1.185588in}}%
\pgfpathlineto{\pgfqpoint{1.235696in}{1.175834in}}%
\pgfpathlineto{\pgfqpoint{1.238029in}{1.142845in}}%
\pgfpathlineto{\pgfqpoint{1.241384in}{1.059325in}}%
\pgfpathlineto{\pgfqpoint{1.246048in}{0.880750in}}%
\pgfpathlineto{\pgfqpoint{1.253171in}{0.510252in}}%
\pgfpathlineto{\pgfqpoint{1.255073in}{0.401530in}}%
\pgfpathlineto{\pgfqpoint{1.255629in}{0.433595in}}%
\pgfpathlineto{\pgfqpoint{1.266573in}{1.000660in}}%
\pgfpathlineto{\pgfqpoint{1.271812in}{1.167993in}}%
\pgfpathlineto{\pgfqpoint{1.275364in}{1.222947in}}%
\pgfpathlineto{\pgfqpoint{1.277356in}{1.231004in}}%
\pgfpathlineto{\pgfqpoint{1.277445in}{1.230974in}}%
\pgfpathlineto{\pgfqpoint{1.278217in}{1.229309in}}%
\pgfpathlineto{\pgfqpoint{1.279652in}{1.219526in}}%
\pgfpathlineto{\pgfqpoint{1.281949in}{1.185998in}}%
\pgfpathlineto{\pgfqpoint{1.285268in}{1.100296in}}%
\pgfpathlineto{\pgfqpoint{1.289879in}{0.916485in}}%
\pgfpathlineto{\pgfqpoint{1.296876in}{0.535811in}}%
\pgfpathlineto{\pgfqpoint{1.299100in}{0.401642in}}%
\pgfpathlineto{\pgfqpoint{1.299656in}{0.434919in}}%
\pgfpathlineto{\pgfqpoint{1.310583in}{1.031177in}}%
\pgfpathlineto{\pgfqpoint{1.315821in}{1.207455in}}%
\pgfpathlineto{\pgfqpoint{1.319374in}{1.265408in}}%
\pgfpathlineto{\pgfqpoint{1.321365in}{1.273964in}}%
\pgfpathlineto{\pgfqpoint{1.321473in}{1.273926in}}%
\pgfpathlineto{\pgfqpoint{1.322244in}{1.272145in}}%
\pgfpathlineto{\pgfqpoint{1.323698in}{1.261615in}}%
\pgfpathlineto{\pgfqpoint{1.326012in}{1.225733in}}%
\pgfpathlineto{\pgfqpoint{1.329349in}{1.134436in}}%
\pgfpathlineto{\pgfqpoint{1.333978in}{0.939416in}}%
\pgfpathlineto{\pgfqpoint{1.341029in}{0.535103in}}%
\pgfpathlineto{\pgfqpoint{1.343146in}{0.401844in}}%
\pgfpathlineto{\pgfqpoint{1.343684in}{0.436173in}}%
\pgfpathlineto{\pgfqpoint{1.354574in}{1.060245in}}%
\pgfpathlineto{\pgfqpoint{1.359813in}{1.245725in}}%
\pgfpathlineto{\pgfqpoint{1.363365in}{1.306953in}}%
\pgfpathlineto{\pgfqpoint{1.365375in}{1.316190in}}%
\pgfpathlineto{\pgfqpoint{1.365482in}{1.316155in}}%
\pgfpathlineto{\pgfqpoint{1.366236in}{1.314402in}}%
\pgfpathlineto{\pgfqpoint{1.367653in}{1.303946in}}%
\pgfpathlineto{\pgfqpoint{1.369932in}{1.267778in}}%
\pgfpathlineto{\pgfqpoint{1.373215in}{1.175427in}}%
\pgfpathlineto{\pgfqpoint{1.377790in}{0.976417in}}%
\pgfpathlineto{\pgfqpoint{1.384697in}{0.565343in}}%
\pgfpathlineto{\pgfqpoint{1.387173in}{0.401424in}}%
\pgfpathlineto{\pgfqpoint{1.387729in}{0.438557in}}%
\pgfpathlineto{\pgfqpoint{1.398602in}{1.090299in}}%
\pgfpathlineto{\pgfqpoint{1.403841in}{1.284175in}}%
\pgfpathlineto{\pgfqpoint{1.407393in}{1.348103in}}%
\pgfpathlineto{\pgfqpoint{1.409402in}{1.357706in}}%
\pgfpathlineto{\pgfqpoint{1.409492in}{1.357677in}}%
\pgfpathlineto{\pgfqpoint{1.410210in}{1.356038in}}%
\pgfpathlineto{\pgfqpoint{1.411591in}{1.345830in}}%
\pgfpathlineto{\pgfqpoint{1.413834in}{1.309714in}}%
\pgfpathlineto{\pgfqpoint{1.417081in}{1.216261in}}%
\pgfpathlineto{\pgfqpoint{1.421602in}{1.013871in}}%
\pgfpathlineto{\pgfqpoint{1.428366in}{0.597465in}}%
\pgfpathlineto{\pgfqpoint{1.431201in}{0.401814in}}%
\pgfpathlineto{\pgfqpoint{1.431775in}{0.440981in}}%
\pgfpathlineto{\pgfqpoint{1.442611in}{1.118943in}}%
\pgfpathlineto{\pgfqpoint{1.447850in}{1.321480in}}%
\pgfpathlineto{\pgfqpoint{1.451402in}{1.388385in}}%
\pgfpathlineto{\pgfqpoint{1.453430in}{1.398536in}}%
\pgfpathlineto{\pgfqpoint{1.453519in}{1.398504in}}%
\pgfpathlineto{\pgfqpoint{1.454237in}{1.396779in}}%
\pgfpathlineto{\pgfqpoint{1.455619in}{1.386107in}}%
\pgfpathlineto{\pgfqpoint{1.457861in}{1.348411in}}%
\pgfpathlineto{\pgfqpoint{1.461109in}{1.250941in}}%
\pgfpathlineto{\pgfqpoint{1.465630in}{1.039961in}}%
\pgfpathlineto{\pgfqpoint{1.472411in}{0.604826in}}%
\pgfpathlineto{\pgfqpoint{1.475246in}{0.401767in}}%
\pgfpathlineto{\pgfqpoint{1.475802in}{0.442141in}}%
\pgfpathlineto{\pgfqpoint{1.486621in}{1.147001in}}%
\pgfpathlineto{\pgfqpoint{1.491860in}{1.358112in}}%
\pgfpathlineto{\pgfqpoint{1.495430in}{1.428177in}}%
\pgfpathlineto{\pgfqpoint{1.497439in}{1.438704in}}%
\pgfpathlineto{\pgfqpoint{1.497547in}{1.438668in}}%
\pgfpathlineto{\pgfqpoint{1.498265in}{1.436862in}}%
\pgfpathlineto{\pgfqpoint{1.499646in}{1.425738in}}%
\pgfpathlineto{\pgfqpoint{1.501889in}{1.386494in}}%
\pgfpathlineto{\pgfqpoint{1.505136in}{1.285084in}}%
\pgfpathlineto{\pgfqpoint{1.509657in}{1.065663in}}%
\pgfpathlineto{\pgfqpoint{1.516439in}{0.613311in}}%
\pgfpathlineto{\pgfqpoint{1.519274in}{0.401495in}}%
\pgfpathlineto{\pgfqpoint{1.519830in}{0.443236in}}%
\pgfpathlineto{\pgfqpoint{1.530648in}{1.175443in}}%
\pgfpathlineto{\pgfqpoint{1.535887in}{1.394622in}}%
\pgfpathlineto{\pgfqpoint{1.539457in}{1.467325in}}%
\pgfpathlineto{\pgfqpoint{1.541467in}{1.478229in}}%
\pgfpathlineto{\pgfqpoint{1.541574in}{1.478190in}}%
\pgfpathlineto{\pgfqpoint{1.542274in}{1.476388in}}%
\pgfpathlineto{\pgfqpoint{1.543656in}{1.464960in}}%
\pgfpathlineto{\pgfqpoint{1.545898in}{1.424417in}}%
\pgfpathlineto{\pgfqpoint{1.549146in}{1.319430in}}%
\pgfpathlineto{\pgfqpoint{1.553667in}{1.092056in}}%
\pgfpathlineto{\pgfqpoint{1.560449in}{0.623074in}}%
\pgfpathlineto{\pgfqpoint{1.563301in}{0.402018in}}%
\pgfpathlineto{\pgfqpoint{1.563875in}{0.445673in}}%
\pgfpathlineto{\pgfqpoint{1.574658in}{1.202418in}}%
\pgfpathlineto{\pgfqpoint{1.579897in}{1.429989in}}%
\pgfpathlineto{\pgfqpoint{1.583467in}{1.505648in}}%
\pgfpathlineto{\pgfqpoint{1.585494in}{1.517132in}}%
\pgfpathlineto{\pgfqpoint{1.585584in}{1.517102in}}%
\pgfpathlineto{\pgfqpoint{1.586266in}{1.515381in}}%
\pgfpathlineto{\pgfqpoint{1.587611in}{1.504247in}}%
\pgfpathlineto{\pgfqpoint{1.589818in}{1.463998in}}%
\pgfpathlineto{\pgfqpoint{1.593012in}{1.359256in}}%
\pgfpathlineto{\pgfqpoint{1.597461in}{1.131304in}}%
\pgfpathlineto{\pgfqpoint{1.604081in}{0.662478in}}%
\pgfpathlineto{\pgfqpoint{1.607347in}{0.401654in}}%
\pgfpathlineto{\pgfqpoint{1.607921in}{0.448147in}}%
\pgfpathlineto{\pgfqpoint{1.618685in}{1.229880in}}%
\pgfpathlineto{\pgfqpoint{1.623924in}{1.465312in}}%
\pgfpathlineto{\pgfqpoint{1.627495in}{1.543563in}}%
\pgfpathlineto{\pgfqpoint{1.629522in}{1.555433in}}%
\pgfpathlineto{\pgfqpoint{1.629612in}{1.555401in}}%
\pgfpathlineto{\pgfqpoint{1.630293in}{1.553618in}}%
\pgfpathlineto{\pgfqpoint{1.631639in}{1.542096in}}%
\pgfpathlineto{\pgfqpoint{1.633846in}{1.500461in}}%
\pgfpathlineto{\pgfqpoint{1.637039in}{1.392138in}}%
\pgfpathlineto{\pgfqpoint{1.641489in}{1.156437in}}%
\pgfpathlineto{\pgfqpoint{1.648109in}{0.671791in}}%
\pgfpathlineto{\pgfqpoint{1.651374in}{0.401669in}}%
\pgfpathlineto{\pgfqpoint{1.651948in}{0.449158in}}%
\pgfpathlineto{\pgfqpoint{1.662695in}{1.255833in}}%
\pgfpathlineto{\pgfqpoint{1.667952in}{1.500079in}}%
\pgfpathlineto{\pgfqpoint{1.671522in}{1.580893in}}%
\pgfpathlineto{\pgfqpoint{1.673549in}{1.593148in}}%
\pgfpathlineto{\pgfqpoint{1.673639in}{1.593116in}}%
\pgfpathlineto{\pgfqpoint{1.674303in}{1.591358in}}%
\pgfpathlineto{\pgfqpoint{1.675631in}{1.579834in}}%
\pgfpathlineto{\pgfqpoint{1.677819in}{1.537773in}}%
\pgfpathlineto{\pgfqpoint{1.680995in}{1.427670in}}%
\pgfpathlineto{\pgfqpoint{1.685426in}{1.187010in}}%
\pgfpathlineto{\pgfqpoint{1.691993in}{0.692991in}}%
\pgfpathlineto{\pgfqpoint{1.695420in}{0.402077in}}%
\pgfpathlineto{\pgfqpoint{1.695994in}{0.451659in}}%
\pgfpathlineto{\pgfqpoint{1.706722in}{1.282372in}}%
\pgfpathlineto{\pgfqpoint{1.711979in}{1.534305in}}%
\pgfpathlineto{\pgfqpoint{1.715550in}{1.617655in}}%
\pgfpathlineto{\pgfqpoint{1.717577in}{1.630297in}}%
\pgfpathlineto{\pgfqpoint{1.717667in}{1.630264in}}%
\pgfpathlineto{\pgfqpoint{1.718330in}{1.628453in}}%
\pgfpathlineto{\pgfqpoint{1.719658in}{1.616574in}}%
\pgfpathlineto{\pgfqpoint{1.721847in}{1.573212in}}%
\pgfpathlineto{\pgfqpoint{1.725022in}{1.459707in}}%
\pgfpathlineto{\pgfqpoint{1.729454in}{1.211635in}}%
\pgfpathlineto{\pgfqpoint{1.736020in}{0.702476in}}%
\pgfpathlineto{\pgfqpoint{1.739447in}{0.401508in}}%
\pgfpathlineto{\pgfqpoint{1.740021in}{0.452600in}}%
\pgfpathlineto{\pgfqpoint{1.750750in}{1.308487in}}%
\pgfpathlineto{\pgfqpoint{1.756007in}{1.568006in}}%
\pgfpathlineto{\pgfqpoint{1.759577in}{1.653866in}}%
\pgfpathlineto{\pgfqpoint{1.761604in}{1.666896in}}%
\pgfpathlineto{\pgfqpoint{1.761694in}{1.666862in}}%
\pgfpathlineto{\pgfqpoint{1.762340in}{1.665089in}}%
\pgfpathlineto{\pgfqpoint{1.763650in}{1.653255in}}%
\pgfpathlineto{\pgfqpoint{1.765821in}{1.609597in}}%
\pgfpathlineto{\pgfqpoint{1.768978in}{1.494582in}}%
\pgfpathlineto{\pgfqpoint{1.773374in}{1.243252in}}%
\pgfpathlineto{\pgfqpoint{1.779886in}{0.726123in}}%
\pgfpathlineto{\pgfqpoint{1.783475in}{0.401872in}}%
\pgfpathlineto{\pgfqpoint{1.784067in}{0.455127in}}%
\pgfpathlineto{\pgfqpoint{1.794777in}{1.334189in}}%
\pgfpathlineto{\pgfqpoint{1.800034in}{1.601199in}}%
\pgfpathlineto{\pgfqpoint{1.803605in}{1.689542in}}%
\pgfpathlineto{\pgfqpoint{1.805632in}{1.702961in}}%
\pgfpathlineto{\pgfqpoint{1.805722in}{1.702927in}}%
\pgfpathlineto{\pgfqpoint{1.806367in}{1.701109in}}%
\pgfpathlineto{\pgfqpoint{1.807677in}{1.688949in}}%
\pgfpathlineto{\pgfqpoint{1.809830in}{1.644565in}}%
\pgfpathlineto{\pgfqpoint{1.812970in}{1.527502in}}%
\pgfpathlineto{\pgfqpoint{1.817347in}{1.271176in}}%
\pgfpathlineto{\pgfqpoint{1.823824in}{0.743937in}}%
\pgfpathlineto{\pgfqpoint{1.827520in}{0.401957in}}%
\pgfpathlineto{\pgfqpoint{1.828094in}{0.456002in}}%
\pgfpathlineto{\pgfqpoint{1.838805in}{1.359490in}}%
\pgfpathlineto{\pgfqpoint{1.844062in}{1.633896in}}%
\pgfpathlineto{\pgfqpoint{1.847632in}{1.724698in}}%
\pgfpathlineto{\pgfqpoint{1.849659in}{1.738507in}}%
\pgfpathlineto{\pgfqpoint{1.849749in}{1.738473in}}%
\pgfpathlineto{\pgfqpoint{1.850395in}{1.736612in}}%
\pgfpathlineto{\pgfqpoint{1.851705in}{1.724136in}}%
\pgfpathlineto{\pgfqpoint{1.853858in}{1.678568in}}%
\pgfpathlineto{\pgfqpoint{1.856997in}{1.558361in}}%
\pgfpathlineto{\pgfqpoint{1.861375in}{1.295140in}}%
\pgfpathlineto{\pgfqpoint{1.867852in}{0.753747in}}%
\pgfpathlineto{\pgfqpoint{1.871548in}{0.401446in}}%
\pgfpathlineto{\pgfqpoint{1.872140in}{0.458555in}}%
\pgfpathlineto{\pgfqpoint{1.882815in}{1.383181in}}%
\pgfpathlineto{\pgfqpoint{1.888071in}{1.665431in}}%
\pgfpathlineto{\pgfqpoint{1.891642in}{1.759097in}}%
\pgfpathlineto{\pgfqpoint{1.893687in}{1.773549in}}%
\pgfpathlineto{\pgfqpoint{1.893777in}{1.773516in}}%
\pgfpathlineto{\pgfqpoint{1.894405in}{1.771707in}}%
\pgfpathlineto{\pgfqpoint{1.895696in}{1.759340in}}%
\pgfpathlineto{\pgfqpoint{1.897831in}{1.713654in}}%
\pgfpathlineto{\pgfqpoint{1.900953in}{1.592319in}}%
\pgfpathlineto{\pgfqpoint{1.905313in}{1.325340in}}%
\pgfpathlineto{\pgfqpoint{1.911736in}{0.777116in}}%
\pgfpathlineto{\pgfqpoint{1.915575in}{0.402104in}}%
\pgfpathlineto{\pgfqpoint{1.916167in}{0.459367in}}%
\pgfpathlineto{\pgfqpoint{1.926842in}{1.407683in}}%
\pgfpathlineto{\pgfqpoint{1.932099in}{1.697163in}}%
\pgfpathlineto{\pgfqpoint{1.935669in}{1.793251in}}%
\pgfpathlineto{\pgfqpoint{1.937714in}{1.808101in}}%
\pgfpathlineto{\pgfqpoint{1.937804in}{1.808068in}}%
\pgfpathlineto{\pgfqpoint{1.938432in}{1.806225in}}%
\pgfpathlineto{\pgfqpoint{1.939724in}{1.793568in}}%
\pgfpathlineto{\pgfqpoint{1.941859in}{1.746770in}}%
\pgfpathlineto{\pgfqpoint{1.944981in}{1.622445in}}%
\pgfpathlineto{\pgfqpoint{1.949340in}{1.348860in}}%
\pgfpathlineto{\pgfqpoint{1.955763in}{0.787072in}}%
\pgfpathlineto{\pgfqpoint{1.959620in}{0.401806in}}%
\pgfpathlineto{\pgfqpoint{1.960213in}{0.461947in}}%
\pgfpathlineto{\pgfqpoint{1.970870in}{1.431816in}}%
\pgfpathlineto{\pgfqpoint{1.976126in}{1.728440in}}%
\pgfpathlineto{\pgfqpoint{1.979697in}{1.826928in}}%
\pgfpathlineto{\pgfqpoint{1.981742in}{1.842177in}}%
\pgfpathlineto{\pgfqpoint{1.981832in}{1.842145in}}%
\pgfpathlineto{\pgfqpoint{1.982442in}{1.840363in}}%
\pgfpathlineto{\pgfqpoint{1.983715in}{1.827854in}}%
\pgfpathlineto{\pgfqpoint{1.985832in}{1.781050in}}%
\pgfpathlineto{\pgfqpoint{1.988936in}{1.655832in}}%
\pgfpathlineto{\pgfqpoint{1.993260in}{1.380273in}}%
\pgfpathlineto{\pgfqpoint{1.999611in}{0.814777in}}%
\pgfpathlineto{\pgfqpoint{2.003648in}{0.401651in}}%
\pgfpathlineto{\pgfqpoint{2.004240in}{0.462697in}}%
\pgfpathlineto{\pgfqpoint{2.014897in}{1.455592in}}%
\pgfpathlineto{\pgfqpoint{2.020154in}{1.759275in}}%
\pgfpathlineto{\pgfqpoint{2.023724in}{1.860141in}}%
\pgfpathlineto{\pgfqpoint{2.025769in}{1.875789in}}%
\pgfpathlineto{\pgfqpoint{2.025859in}{1.875758in}}%
\pgfpathlineto{\pgfqpoint{2.026469in}{1.873947in}}%
\pgfpathlineto{\pgfqpoint{2.027743in}{1.861175in}}%
\pgfpathlineto{\pgfqpoint{2.029860in}{1.813326in}}%
\pgfpathlineto{\pgfqpoint{2.032964in}{1.685264in}}%
\pgfpathlineto{\pgfqpoint{2.037288in}{1.403403in}}%
\pgfpathlineto{\pgfqpoint{2.043639in}{0.824950in}}%
\pgfpathlineto{\pgfqpoint{2.047693in}{0.402324in}}%
\pgfpathlineto{\pgfqpoint{2.048285in}{0.465304in}}%
\pgfpathlineto{\pgfqpoint{2.058925in}{1.479018in}}%
\pgfpathlineto{\pgfqpoint{2.064181in}{1.789679in}}%
\pgfpathlineto{\pgfqpoint{2.067752in}{1.892901in}}%
\pgfpathlineto{\pgfqpoint{2.069815in}{1.908949in}}%
\pgfpathlineto{\pgfqpoint{2.069887in}{1.908921in}}%
\pgfpathlineto{\pgfqpoint{2.070479in}{1.907178in}}%
\pgfpathlineto{\pgfqpoint{2.071735in}{1.894591in}}%
\pgfpathlineto{\pgfqpoint{2.073834in}{1.846839in}}%
\pgfpathlineto{\pgfqpoint{2.076902in}{1.719048in}}%
\pgfpathlineto{\pgfqpoint{2.081189in}{1.436160in}}%
\pgfpathlineto{\pgfqpoint{2.087469in}{0.855465in}}%
\pgfpathlineto{\pgfqpoint{2.091721in}{0.401625in}}%
\pgfpathlineto{\pgfqpoint{2.092313in}{0.465995in}}%
\pgfpathlineto{\pgfqpoint{2.102952in}{1.502106in}}%
\pgfpathlineto{\pgfqpoint{2.108209in}{1.819664in}}%
\pgfpathlineto{\pgfqpoint{2.111797in}{1.925508in}}%
\pgfpathlineto{\pgfqpoint{2.113842in}{1.941671in}}%
\pgfpathlineto{\pgfqpoint{2.113914in}{1.941644in}}%
\pgfpathlineto{\pgfqpoint{2.114506in}{1.939879in}}%
\pgfpathlineto{\pgfqpoint{2.115744in}{1.927321in}}%
\pgfpathlineto{\pgfqpoint{2.117825in}{1.879435in}}%
\pgfpathlineto{\pgfqpoint{2.120893in}{1.749757in}}%
\pgfpathlineto{\pgfqpoint{2.125163in}{1.463259in}}%
\pgfpathlineto{\pgfqpoint{2.131407in}{0.875339in}}%
\pgfpathlineto{\pgfqpoint{2.135748in}{0.401884in}}%
\pgfpathlineto{\pgfqpoint{2.136358in}{0.468629in}}%
\pgfpathlineto{\pgfqpoint{2.146998in}{1.526264in}}%
\pgfpathlineto{\pgfqpoint{2.152254in}{1.850024in}}%
\pgfpathlineto{\pgfqpoint{2.155825in}{1.957406in}}%
\pgfpathlineto{\pgfqpoint{2.157870in}{1.973966in}}%
\pgfpathlineto{\pgfqpoint{2.157960in}{1.973927in}}%
\pgfpathlineto{\pgfqpoint{2.158570in}{1.971956in}}%
\pgfpathlineto{\pgfqpoint{2.159843in}{1.958253in}}%
\pgfpathlineto{\pgfqpoint{2.161960in}{1.907093in}}%
\pgfpathlineto{\pgfqpoint{2.165064in}{1.770362in}}%
\pgfpathlineto{\pgfqpoint{2.169388in}{1.469663in}}%
\pgfpathlineto{\pgfqpoint{2.175757in}{0.851015in}}%
\pgfpathlineto{\pgfqpoint{2.179794in}{0.402166in}}%
\pgfpathlineto{\pgfqpoint{2.180386in}{0.469264in}}%
\pgfpathlineto{\pgfqpoint{2.191025in}{1.548727in}}%
\pgfpathlineto{\pgfqpoint{2.196282in}{1.879220in}}%
\pgfpathlineto{\pgfqpoint{2.199852in}{1.988887in}}%
\pgfpathlineto{\pgfqpoint{2.201897in}{2.005843in}}%
\pgfpathlineto{\pgfqpoint{2.201987in}{2.005807in}}%
\pgfpathlineto{\pgfqpoint{2.202579in}{2.003916in}}%
\pgfpathlineto{\pgfqpoint{2.203835in}{1.990445in}}%
\pgfpathlineto{\pgfqpoint{2.205934in}{1.939507in}}%
\pgfpathlineto{\pgfqpoint{2.209020in}{1.802367in}}%
\pgfpathlineto{\pgfqpoint{2.213326in}{1.499229in}}%
\pgfpathlineto{\pgfqpoint{2.219641in}{0.876564in}}%
\pgfpathlineto{\pgfqpoint{2.223821in}{0.401416in}}%
\pgfpathlineto{\pgfqpoint{2.224413in}{0.469851in}}%
\pgfpathlineto{\pgfqpoint{2.235053in}{1.570875in}}%
\pgfpathlineto{\pgfqpoint{2.240309in}{1.908027in}}%
\pgfpathlineto{\pgfqpoint{2.243880in}{2.019961in}}%
\pgfpathlineto{\pgfqpoint{2.245925in}{2.037315in}}%
\pgfpathlineto{\pgfqpoint{2.246015in}{2.037281in}}%
\pgfpathlineto{\pgfqpoint{2.246607in}{2.035372in}}%
\pgfpathlineto{\pgfqpoint{2.247862in}{2.021676in}}%
\pgfpathlineto{\pgfqpoint{2.249962in}{1.969808in}}%
\pgfpathlineto{\pgfqpoint{2.253047in}{1.830081in}}%
\pgfpathlineto{\pgfqpoint{2.257353in}{1.521149in}}%
\pgfpathlineto{\pgfqpoint{2.263651in}{0.888520in}}%
\pgfpathlineto{\pgfqpoint{2.267849in}{0.402142in}}%
\pgfpathlineto{\pgfqpoint{2.268459in}{0.472505in}}%
\pgfpathlineto{\pgfqpoint{2.279080in}{1.592716in}}%
\pgfpathlineto{\pgfqpoint{2.284337in}{1.936456in}}%
\pgfpathlineto{\pgfqpoint{2.287907in}{2.050638in}}%
\pgfpathlineto{\pgfqpoint{2.289970in}{2.068390in}}%
\pgfpathlineto{\pgfqpoint{2.290042in}{2.068359in}}%
\pgfpathlineto{\pgfqpoint{2.290616in}{2.066536in}}%
\pgfpathlineto{\pgfqpoint{2.291854in}{2.053103in}}%
\pgfpathlineto{\pgfqpoint{2.293935in}{2.001547in}}%
\pgfpathlineto{\pgfqpoint{2.296985in}{1.862642in}}%
\pgfpathlineto{\pgfqpoint{2.301255in}{1.553700in}}%
\pgfpathlineto{\pgfqpoint{2.307481in}{0.920908in}}%
\pgfpathlineto{\pgfqpoint{2.311894in}{0.401980in}}%
\pgfpathlineto{\pgfqpoint{2.312504in}{0.475192in}}%
\pgfpathlineto{\pgfqpoint{2.323108in}{1.614258in}}%
\pgfpathlineto{\pgfqpoint{2.328364in}{1.964517in}}%
\pgfpathlineto{\pgfqpoint{2.331953in}{2.081243in}}%
\pgfpathlineto{\pgfqpoint{2.333998in}{2.099080in}}%
\pgfpathlineto{\pgfqpoint{2.334070in}{2.099051in}}%
\pgfpathlineto{\pgfqpoint{2.334644in}{2.097215in}}%
\pgfpathlineto{\pgfqpoint{2.335864in}{2.083871in}}%
\pgfpathlineto{\pgfqpoint{2.337927in}{2.032363in}}%
\pgfpathlineto{\pgfqpoint{2.340959in}{1.892929in}}%
\pgfpathlineto{\pgfqpoint{2.345193in}{1.583180in}}%
\pgfpathlineto{\pgfqpoint{2.351365in}{0.947711in}}%
\pgfpathlineto{\pgfqpoint{2.355922in}{0.401597in}}%
\pgfpathlineto{\pgfqpoint{2.356532in}{0.475720in}}%
\pgfpathlineto{\pgfqpoint{2.367135in}{1.635508in}}%
\pgfpathlineto{\pgfqpoint{2.372392in}{1.992217in}}%
\pgfpathlineto{\pgfqpoint{2.375980in}{2.111161in}}%
\pgfpathlineto{\pgfqpoint{2.378025in}{2.129394in}}%
\pgfpathlineto{\pgfqpoint{2.378115in}{2.129353in}}%
\pgfpathlineto{\pgfqpoint{2.378707in}{2.127308in}}%
\pgfpathlineto{\pgfqpoint{2.379963in}{2.112781in}}%
\pgfpathlineto{\pgfqpoint{2.382062in}{2.057900in}}%
\pgfpathlineto{\pgfqpoint{2.385148in}{1.910204in}}%
\pgfpathlineto{\pgfqpoint{2.389454in}{1.583837in}}%
\pgfpathlineto{\pgfqpoint{2.395769in}{0.913699in}}%
\pgfpathlineto{\pgfqpoint{2.399949in}{0.402424in}}%
\pgfpathlineto{\pgfqpoint{2.400559in}{0.476204in}}%
\pgfpathlineto{\pgfqpoint{2.411163in}{1.656472in}}%
\pgfpathlineto{\pgfqpoint{2.416419in}{2.019566in}}%
\pgfpathlineto{\pgfqpoint{2.420008in}{2.140711in}}%
\pgfpathlineto{\pgfqpoint{2.422053in}{2.159340in}}%
\pgfpathlineto{\pgfqpoint{2.422142in}{2.159302in}}%
\pgfpathlineto{\pgfqpoint{2.422717in}{2.157353in}}%
\pgfpathlineto{\pgfqpoint{2.423955in}{2.143131in}}%
\pgfpathlineto{\pgfqpoint{2.426036in}{2.088675in}}%
\pgfpathlineto{\pgfqpoint{2.429086in}{1.942092in}}%
\pgfpathlineto{\pgfqpoint{2.433356in}{1.616251in}}%
\pgfpathlineto{\pgfqpoint{2.439599in}{0.946990in}}%
\pgfpathlineto{\pgfqpoint{2.443995in}{0.401767in}}%
\pgfpathlineto{\pgfqpoint{2.444605in}{0.478911in}}%
\pgfpathlineto{\pgfqpoint{2.455190in}{1.677158in}}%
\pgfpathlineto{\pgfqpoint{2.460447in}{2.046572in}}%
\pgfpathlineto{\pgfqpoint{2.464035in}{2.169902in}}%
\pgfpathlineto{\pgfqpoint{2.466080in}{2.188927in}}%
\pgfpathlineto{\pgfqpoint{2.466170in}{2.188892in}}%
\pgfpathlineto{\pgfqpoint{2.466744in}{2.186934in}}%
\pgfpathlineto{\pgfqpoint{2.467982in}{2.172523in}}%
\pgfpathlineto{\pgfqpoint{2.470063in}{2.117236in}}%
\pgfpathlineto{\pgfqpoint{2.473113in}{1.968311in}}%
\pgfpathlineto{\pgfqpoint{2.477383in}{1.637159in}}%
\pgfpathlineto{\pgfqpoint{2.483609in}{0.959062in}}%
\pgfpathlineto{\pgfqpoint{2.488022in}{0.401856in}}%
\pgfpathlineto{\pgfqpoint{2.488632in}{0.479345in}}%
\pgfpathlineto{\pgfqpoint{2.499235in}{1.699195in}}%
\pgfpathlineto{\pgfqpoint{2.504492in}{2.074151in}}%
\pgfpathlineto{\pgfqpoint{2.508080in}{2.199078in}}%
\pgfpathlineto{\pgfqpoint{2.510126in}{2.218163in}}%
\pgfpathlineto{\pgfqpoint{2.510198in}{2.218132in}}%
\pgfpathlineto{\pgfqpoint{2.510754in}{2.216274in}}%
\pgfpathlineto{\pgfqpoint{2.511956in}{2.202501in}}%
\pgfpathlineto{\pgfqpoint{2.514001in}{2.148694in}}%
\pgfpathlineto{\pgfqpoint{2.517015in}{2.002017in}}%
\pgfpathlineto{\pgfqpoint{2.521231in}{1.674591in}}%
\pgfpathlineto{\pgfqpoint{2.527367in}{1.002107in}}%
\pgfpathlineto{\pgfqpoint{2.532068in}{0.402393in}}%
\pgfpathlineto{\pgfqpoint{2.532678in}{0.482080in}}%
\pgfpathlineto{\pgfqpoint{2.543263in}{1.719369in}}%
\pgfpathlineto{\pgfqpoint{2.548520in}{2.100510in}}%
\pgfpathlineto{\pgfqpoint{2.552108in}{2.227580in}}%
\pgfpathlineto{\pgfqpoint{2.554153in}{2.247059in}}%
\pgfpathlineto{\pgfqpoint{2.554243in}{2.247016in}}%
\pgfpathlineto{\pgfqpoint{2.554817in}{2.244947in}}%
\pgfpathlineto{\pgfqpoint{2.556055in}{2.229968in}}%
\pgfpathlineto{\pgfqpoint{2.558136in}{2.172721in}}%
\pgfpathlineto{\pgfqpoint{2.561186in}{2.018742in}}%
\pgfpathlineto{\pgfqpoint{2.565456in}{1.676610in}}%
\pgfpathlineto{\pgfqpoint{2.571700in}{0.974149in}}%
\pgfpathlineto{\pgfqpoint{2.576095in}{0.401529in}}%
\pgfpathlineto{\pgfqpoint{2.576705in}{0.482467in}}%
\pgfpathlineto{\pgfqpoint{2.587290in}{1.739284in}}%
\pgfpathlineto{\pgfqpoint{2.592547in}{2.126549in}}%
\pgfpathlineto{\pgfqpoint{2.596135in}{2.255747in}}%
\pgfpathlineto{\pgfqpoint{2.598181in}{2.275620in}}%
\pgfpathlineto{\pgfqpoint{2.598270in}{2.275580in}}%
\pgfpathlineto{\pgfqpoint{2.598845in}{2.273506in}}%
\pgfpathlineto{\pgfqpoint{2.600083in}{2.258352in}}%
\pgfpathlineto{\pgfqpoint{2.602164in}{2.200314in}}%
\pgfpathlineto{\pgfqpoint{2.605214in}{2.044089in}}%
\pgfpathlineto{\pgfqpoint{2.609484in}{1.696850in}}%
\pgfpathlineto{\pgfqpoint{2.615727in}{0.983763in}}%
\pgfpathlineto{\pgfqpoint{2.620123in}{0.402139in}}%
\pgfpathlineto{\pgfqpoint{2.620733in}{0.482812in}}%
\pgfpathlineto{\pgfqpoint{2.631318in}{1.758944in}}%
\pgfpathlineto{\pgfqpoint{2.636575in}{2.152276in}}%
\pgfpathlineto{\pgfqpoint{2.640163in}{2.283587in}}%
\pgfpathlineto{\pgfqpoint{2.642208in}{2.303855in}}%
\pgfpathlineto{\pgfqpoint{2.642298in}{2.303819in}}%
\pgfpathlineto{\pgfqpoint{2.642854in}{2.301854in}}%
\pgfpathlineto{\pgfqpoint{2.644074in}{2.287069in}}%
\pgfpathlineto{\pgfqpoint{2.646137in}{2.229638in}}%
\pgfpathlineto{\pgfqpoint{2.649169in}{2.073837in}}%
\pgfpathlineto{\pgfqpoint{2.653404in}{1.727430in}}%
\pgfpathlineto{\pgfqpoint{2.659575in}{1.016511in}}%
\pgfpathlineto{\pgfqpoint{2.664168in}{0.402175in}}%
\pgfpathlineto{\pgfqpoint{2.664778in}{0.485569in}}%
\pgfpathlineto{\pgfqpoint{2.675345in}{1.778355in}}%
\pgfpathlineto{\pgfqpoint{2.680602in}{2.177697in}}%
\pgfpathlineto{\pgfqpoint{2.684190in}{2.311106in}}%
\pgfpathlineto{\pgfqpoint{2.686254in}{2.331771in}}%
\pgfpathlineto{\pgfqpoint{2.686325in}{2.331738in}}%
\pgfpathlineto{\pgfqpoint{2.686882in}{2.329771in}}%
\pgfpathlineto{\pgfqpoint{2.688084in}{2.315154in}}%
\pgfpathlineto{\pgfqpoint{2.690129in}{2.258015in}}%
\pgfpathlineto{\pgfqpoint{2.693143in}{2.102237in}}%
\pgfpathlineto{\pgfqpoint{2.697359in}{1.754501in}}%
\pgfpathlineto{\pgfqpoint{2.703495in}{1.040388in}}%
\pgfpathlineto{\pgfqpoint{2.708196in}{0.401510in}}%
\pgfpathlineto{\pgfqpoint{2.708824in}{0.488356in}}%
\pgfpathlineto{\pgfqpoint{2.719391in}{1.799275in}}%
\pgfpathlineto{\pgfqpoint{2.724648in}{2.203798in}}%
\pgfpathlineto{\pgfqpoint{2.728236in}{2.338676in}}%
\pgfpathlineto{\pgfqpoint{2.730281in}{2.359375in}}%
\pgfpathlineto{\pgfqpoint{2.730371in}{2.359330in}}%
\pgfpathlineto{\pgfqpoint{2.730945in}{2.357146in}}%
\pgfpathlineto{\pgfqpoint{2.732183in}{2.341277in}}%
\pgfpathlineto{\pgfqpoint{2.734264in}{2.280587in}}%
\pgfpathlineto{\pgfqpoint{2.737314in}{2.117318in}}%
\pgfpathlineto{\pgfqpoint{2.741584in}{1.754545in}}%
\pgfpathlineto{\pgfqpoint{2.747828in}{1.009774in}}%
\pgfpathlineto{\pgfqpoint{2.752223in}{0.402444in}}%
\pgfpathlineto{\pgfqpoint{2.752833in}{0.486130in}}%
\pgfpathlineto{\pgfqpoint{2.763418in}{1.818231in}}%
\pgfpathlineto{\pgfqpoint{2.768675in}{2.228642in}}%
\pgfpathlineto{\pgfqpoint{2.772263in}{2.365582in}}%
\pgfpathlineto{\pgfqpoint{2.774309in}{2.386674in}}%
\pgfpathlineto{\pgfqpoint{2.774398in}{2.386633in}}%
\pgfpathlineto{\pgfqpoint{2.774955in}{2.384567in}}%
\pgfpathlineto{\pgfqpoint{2.776175in}{2.369103in}}%
\pgfpathlineto{\pgfqpoint{2.778238in}{2.309118in}}%
\pgfpathlineto{\pgfqpoint{2.781270in}{2.146476in}}%
\pgfpathlineto{\pgfqpoint{2.785504in}{1.784974in}}%
\pgfpathlineto{\pgfqpoint{2.791676in}{1.043278in}}%
\pgfpathlineto{\pgfqpoint{2.796269in}{0.401933in}}%
\pgfpathlineto{\pgfqpoint{2.796879in}{0.488909in}}%
\pgfpathlineto{\pgfqpoint{2.807446in}{1.836954in}}%
\pgfpathlineto{\pgfqpoint{2.812703in}{2.253200in}}%
\pgfpathlineto{\pgfqpoint{2.816291in}{2.392188in}}%
\pgfpathlineto{\pgfqpoint{2.818354in}{2.413674in}}%
\pgfpathlineto{\pgfqpoint{2.818426in}{2.413638in}}%
\pgfpathlineto{\pgfqpoint{2.818982in}{2.411572in}}%
\pgfpathlineto{\pgfqpoint{2.820184in}{2.396302in}}%
\pgfpathlineto{\pgfqpoint{2.822229in}{2.336689in}}%
\pgfpathlineto{\pgfqpoint{2.825244in}{2.174246in}}%
\pgfpathlineto{\pgfqpoint{2.829460in}{1.811743in}}%
\pgfpathlineto{\pgfqpoint{2.835596in}{1.067495in}}%
\pgfpathlineto{\pgfqpoint{2.840296in}{0.401795in}}%
\pgfpathlineto{\pgfqpoint{2.840924in}{0.491719in}}%
\pgfpathlineto{\pgfqpoint{2.851473in}{1.855448in}}%
\pgfpathlineto{\pgfqpoint{2.856730in}{2.277479in}}%
\pgfpathlineto{\pgfqpoint{2.860318in}{2.418501in}}%
\pgfpathlineto{\pgfqpoint{2.862382in}{2.440383in}}%
\pgfpathlineto{\pgfqpoint{2.862453in}{2.440351in}}%
\pgfpathlineto{\pgfqpoint{2.862992in}{2.438405in}}%
\pgfpathlineto{\pgfqpoint{2.864176in}{2.423556in}}%
\pgfpathlineto{\pgfqpoint{2.866203in}{2.364714in}}%
\pgfpathlineto{\pgfqpoint{2.869199in}{2.203063in}}%
\pgfpathlineto{\pgfqpoint{2.873380in}{1.842188in}}%
\pgfpathlineto{\pgfqpoint{2.879444in}{1.101839in}}%
\pgfpathlineto{\pgfqpoint{2.884342in}{0.402634in}}%
\pgfpathlineto{\pgfqpoint{2.884970in}{0.494559in}}%
\pgfpathlineto{\pgfqpoint{2.895519in}{1.875568in}}%
\pgfpathlineto{\pgfqpoint{2.900776in}{2.302516in}}%
\pgfpathlineto{\pgfqpoint{2.904364in}{2.444912in}}%
\pgfpathlineto{\pgfqpoint{2.906409in}{2.466806in}}%
\pgfpathlineto{\pgfqpoint{2.906499in}{2.466762in}}%
\pgfpathlineto{\pgfqpoint{2.907055in}{2.464598in}}%
\pgfpathlineto{\pgfqpoint{2.908275in}{2.448481in}}%
\pgfpathlineto{\pgfqpoint{2.910338in}{2.386031in}}%
\pgfpathlineto{\pgfqpoint{2.913370in}{2.216778in}}%
\pgfpathlineto{\pgfqpoint{2.917604in}{1.840679in}}%
\pgfpathlineto{\pgfqpoint{2.923776in}{1.069209in}}%
\pgfpathlineto{\pgfqpoint{2.928369in}{0.401668in}}%
\pgfpathlineto{\pgfqpoint{2.928979in}{0.492112in}}%
\pgfpathlineto{\pgfqpoint{2.939546in}{1.893645in}}%
\pgfpathlineto{\pgfqpoint{2.944803in}{2.326266in}}%
\pgfpathlineto{\pgfqpoint{2.948391in}{2.470663in}}%
\pgfpathlineto{\pgfqpoint{2.950437in}{2.492950in}}%
\pgfpathlineto{\pgfqpoint{2.950526in}{2.492910in}}%
\pgfpathlineto{\pgfqpoint{2.951065in}{2.490872in}}%
\pgfpathlineto{\pgfqpoint{2.952267in}{2.475202in}}%
\pgfpathlineto{\pgfqpoint{2.954294in}{2.414311in}}%
\pgfpathlineto{\pgfqpoint{2.957290in}{2.247756in}}%
\pgfpathlineto{\pgfqpoint{2.961488in}{1.874813in}}%
\pgfpathlineto{\pgfqpoint{2.967588in}{1.108963in}}%
\pgfpathlineto{\pgfqpoint{2.972397in}{0.402101in}}%
\pgfpathlineto{\pgfqpoint{2.973025in}{0.494944in}}%
\pgfpathlineto{\pgfqpoint{2.983574in}{1.911508in}}%
\pgfpathlineto{\pgfqpoint{2.988831in}{2.349753in}}%
\pgfpathlineto{\pgfqpoint{2.992419in}{2.496138in}}%
\pgfpathlineto{\pgfqpoint{2.994482in}{2.518819in}}%
\pgfpathlineto{\pgfqpoint{2.994554in}{2.518784in}}%
\pgfpathlineto{\pgfqpoint{2.995092in}{2.516752in}}%
\pgfpathlineto{\pgfqpoint{2.996276in}{2.501307in}}%
\pgfpathlineto{\pgfqpoint{2.998304in}{2.440164in}}%
\pgfpathlineto{\pgfqpoint{3.001300in}{2.272255in}}%
\pgfpathlineto{\pgfqpoint{3.005480in}{1.897494in}}%
\pgfpathlineto{\pgfqpoint{3.011544in}{1.128814in}}%
\pgfpathlineto{\pgfqpoint{3.016442in}{0.402388in}}%
\pgfpathlineto{\pgfqpoint{3.017070in}{0.497807in}}%
\pgfpathlineto{\pgfqpoint{3.027601in}{1.929160in}}%
\pgfpathlineto{\pgfqpoint{3.032858in}{2.372982in}}%
\pgfpathlineto{\pgfqpoint{3.036446in}{2.521344in}}%
\pgfpathlineto{\pgfqpoint{3.038510in}{2.544421in}}%
\pgfpathlineto{\pgfqpoint{3.038581in}{2.544390in}}%
\pgfpathlineto{\pgfqpoint{3.039102in}{2.542483in}}%
\pgfpathlineto{\pgfqpoint{3.040268in}{2.527499in}}%
\pgfpathlineto{\pgfqpoint{3.042259in}{2.467974in}}%
\pgfpathlineto{\pgfqpoint{3.045220in}{2.302927in}}%
\pgfpathlineto{\pgfqpoint{3.049364in}{1.931677in}}%
\pgfpathlineto{\pgfqpoint{3.055356in}{1.169426in}}%
\pgfpathlineto{\pgfqpoint{3.060470in}{0.401397in}}%
\pgfpathlineto{\pgfqpoint{3.061115in}{0.500698in}}%
\pgfpathlineto{\pgfqpoint{3.071647in}{1.948548in}}%
\pgfpathlineto{\pgfqpoint{3.076904in}{2.397045in}}%
\pgfpathlineto{\pgfqpoint{3.080492in}{2.546692in}}%
\pgfpathlineto{\pgfqpoint{3.082537in}{2.569761in}}%
\pgfpathlineto{\pgfqpoint{3.082627in}{2.569717in}}%
\pgfpathlineto{\pgfqpoint{3.083165in}{2.567594in}}%
\pgfpathlineto{\pgfqpoint{3.084367in}{2.551328in}}%
\pgfpathlineto{\pgfqpoint{3.086394in}{2.488167in}}%
\pgfpathlineto{\pgfqpoint{3.089391in}{2.315463in}}%
\pgfpathlineto{\pgfqpoint{3.093589in}{1.928828in}}%
\pgfpathlineto{\pgfqpoint{3.099689in}{1.135002in}}%
\pgfpathlineto{\pgfqpoint{3.104497in}{0.402428in}}%
\pgfpathlineto{\pgfqpoint{3.105125in}{0.498040in}}%
\pgfpathlineto{\pgfqpoint{3.115674in}{1.965816in}}%
\pgfpathlineto{\pgfqpoint{3.120931in}{2.419789in}}%
\pgfpathlineto{\pgfqpoint{3.124519in}{2.571381in}}%
\pgfpathlineto{\pgfqpoint{3.126583in}{2.594842in}}%
\pgfpathlineto{\pgfqpoint{3.126654in}{2.594804in}}%
\pgfpathlineto{\pgfqpoint{3.127193in}{2.592690in}}%
\pgfpathlineto{\pgfqpoint{3.128377in}{2.576671in}}%
\pgfpathlineto{\pgfqpoint{3.130404in}{2.513302in}}%
\pgfpathlineto{\pgfqpoint{3.133400in}{2.339334in}}%
\pgfpathlineto{\pgfqpoint{3.137580in}{1.951125in}}%
\pgfpathlineto{\pgfqpoint{3.143662in}{1.152344in}}%
\pgfpathlineto{\pgfqpoint{3.148542in}{0.402119in}}%
\pgfpathlineto{\pgfqpoint{3.149170in}{0.500925in}}%
\pgfpathlineto{\pgfqpoint{3.159702in}{1.982886in}}%
\pgfpathlineto{\pgfqpoint{3.164959in}{2.442290in}}%
\pgfpathlineto{\pgfqpoint{3.168547in}{2.595817in}}%
\pgfpathlineto{\pgfqpoint{3.170610in}{2.619672in}}%
\pgfpathlineto{\pgfqpoint{3.170682in}{2.619639in}}%
\pgfpathlineto{\pgfqpoint{3.171202in}{2.617657in}}%
\pgfpathlineto{\pgfqpoint{3.172368in}{2.602128in}}%
\pgfpathlineto{\pgfqpoint{3.174360in}{2.540485in}}%
\pgfpathlineto{\pgfqpoint{3.177320in}{2.369616in}}%
\pgfpathlineto{\pgfqpoint{3.181464in}{1.985336in}}%
\pgfpathlineto{\pgfqpoint{3.187457in}{1.196456in}}%
\pgfpathlineto{\pgfqpoint{3.192570in}{0.401705in}}%
\pgfpathlineto{\pgfqpoint{3.193216in}{0.503839in}}%
\pgfpathlineto{\pgfqpoint{3.203729in}{1.999762in}}%
\pgfpathlineto{\pgfqpoint{3.208986in}{2.464553in}}%
\pgfpathlineto{\pgfqpoint{3.212574in}{2.620004in}}%
\pgfpathlineto{\pgfqpoint{3.214638in}{2.644255in}}%
\pgfpathlineto{\pgfqpoint{3.214709in}{2.644226in}}%
\pgfpathlineto{\pgfqpoint{3.215230in}{2.642255in}}%
\pgfpathlineto{\pgfqpoint{3.216396in}{2.626629in}}%
\pgfpathlineto{\pgfqpoint{3.218387in}{2.564429in}}%
\pgfpathlineto{\pgfqpoint{3.221348in}{2.391847in}}%
\pgfpathlineto{\pgfqpoint{3.225492in}{2.003541in}}%
\pgfpathlineto{\pgfqpoint{3.231484in}{1.206189in}}%
\pgfpathlineto{\pgfqpoint{3.236597in}{0.402774in}}%
\pgfpathlineto{\pgfqpoint{3.237243in}{0.503897in}}%
\pgfpathlineto{\pgfqpoint{3.247775in}{2.018479in}}%
\pgfpathlineto{\pgfqpoint{3.253032in}{2.487719in}}%
\pgfpathlineto{\pgfqpoint{3.256620in}{2.644372in}}%
\pgfpathlineto{\pgfqpoint{3.258683in}{2.668595in}}%
\pgfpathlineto{\pgfqpoint{3.258755in}{2.668555in}}%
\pgfpathlineto{\pgfqpoint{3.259293in}{2.666362in}}%
\pgfpathlineto{\pgfqpoint{3.260477in}{2.649790in}}%
\pgfpathlineto{\pgfqpoint{3.262504in}{2.584266in}}%
\pgfpathlineto{\pgfqpoint{3.265501in}{2.404428in}}%
\pgfpathlineto{\pgfqpoint{3.269681in}{2.003179in}}%
\pgfpathlineto{\pgfqpoint{3.275763in}{1.177682in}}%
\pgfpathlineto{\pgfqpoint{3.280643in}{0.401829in}}%
\pgfpathlineto{\pgfqpoint{3.281271in}{0.503921in}}%
\pgfpathlineto{\pgfqpoint{3.291802in}{2.035000in}}%
\pgfpathlineto{\pgfqpoint{3.297059in}{2.509534in}}%
\pgfpathlineto{\pgfqpoint{3.300647in}{2.668083in}}%
\pgfpathlineto{\pgfqpoint{3.302711in}{2.692698in}}%
\pgfpathlineto{\pgfqpoint{3.302782in}{2.692663in}}%
\pgfpathlineto{\pgfqpoint{3.303303in}{2.690609in}}%
\pgfpathlineto{\pgfqpoint{3.304469in}{2.674556in}}%
\pgfpathlineto{\pgfqpoint{3.306460in}{2.610863in}}%
\pgfpathlineto{\pgfqpoint{3.309420in}{2.434350in}}%
\pgfpathlineto{\pgfqpoint{3.313565in}{2.037433in}}%
\pgfpathlineto{\pgfqpoint{3.319557in}{1.222717in}}%
\pgfpathlineto{\pgfqpoint{3.324670in}{0.402033in}}%
\pgfpathlineto{\pgfqpoint{3.325316in}{0.506858in}}%
\pgfpathlineto{\pgfqpoint{3.335830in}{2.051338in}}%
\pgfpathlineto{\pgfqpoint{3.341087in}{2.531125in}}%
\pgfpathlineto{\pgfqpoint{3.344675in}{2.691559in}}%
\pgfpathlineto{\pgfqpoint{3.346738in}{2.716569in}}%
\pgfpathlineto{\pgfqpoint{3.346810in}{2.716538in}}%
\pgfpathlineto{\pgfqpoint{3.347330in}{2.714498in}}%
\pgfpathlineto{\pgfqpoint{3.348496in}{2.698356in}}%
\pgfpathlineto{\pgfqpoint{3.350488in}{2.634131in}}%
\pgfpathlineto{\pgfqpoint{3.353448in}{2.455969in}}%
\pgfpathlineto{\pgfqpoint{3.357592in}{2.055160in}}%
\pgfpathlineto{\pgfqpoint{3.363585in}{1.232233in}}%
\pgfpathlineto{\pgfqpoint{3.368716in}{0.402617in}}%
\pgfpathlineto{\pgfqpoint{3.369362in}{0.509823in}}%
\pgfpathlineto{\pgfqpoint{3.379875in}{2.069591in}}%
\pgfpathlineto{\pgfqpoint{3.385132in}{2.553667in}}%
\pgfpathlineto{\pgfqpoint{3.388720in}{2.715243in}}%
\pgfpathlineto{\pgfqpoint{3.390766in}{2.740211in}}%
\pgfpathlineto{\pgfqpoint{3.390855in}{2.740168in}}%
\pgfpathlineto{\pgfqpoint{3.391376in}{2.738032in}}%
\pgfpathlineto{\pgfqpoint{3.392542in}{2.721556in}}%
\pgfpathlineto{\pgfqpoint{3.394551in}{2.655595in}}%
\pgfpathlineto{\pgfqpoint{3.397529in}{2.473240in}}%
\pgfpathlineto{\pgfqpoint{3.401692in}{2.064326in}}%
\pgfpathlineto{\pgfqpoint{3.407720in}{1.224847in}}%
\pgfpathlineto{\pgfqpoint{3.412743in}{0.401518in}}%
\pgfpathlineto{\pgfqpoint{3.413389in}{0.509809in}}%
\pgfpathlineto{\pgfqpoint{3.423903in}{2.085594in}}%
\pgfpathlineto{\pgfqpoint{3.429160in}{2.574834in}}%
\pgfpathlineto{\pgfqpoint{3.432748in}{2.738269in}}%
\pgfpathlineto{\pgfqpoint{3.434811in}{2.763628in}}%
\pgfpathlineto{\pgfqpoint{3.434883in}{2.763591in}}%
\pgfpathlineto{\pgfqpoint{3.435403in}{2.761469in}}%
\pgfpathlineto{\pgfqpoint{3.436569in}{2.744909in}}%
\pgfpathlineto{\pgfqpoint{3.438561in}{2.679229in}}%
\pgfpathlineto{\pgfqpoint{3.441521in}{2.497241in}}%
\pgfpathlineto{\pgfqpoint{3.445665in}{2.088057in}}%
\pgfpathlineto{\pgfqpoint{3.451658in}{1.248253in}}%
\pgfpathlineto{\pgfqpoint{3.456771in}{0.402380in}}%
\pgfpathlineto{\pgfqpoint{3.457417in}{0.509762in}}%
\pgfpathlineto{\pgfqpoint{3.467930in}{2.101423in}}%
\pgfpathlineto{\pgfqpoint{3.473187in}{2.595790in}}%
\pgfpathlineto{\pgfqpoint{3.476775in}{2.761073in}}%
\pgfpathlineto{\pgfqpoint{3.478838in}{2.786826in}}%
\pgfpathlineto{\pgfqpoint{3.478910in}{2.786794in}}%
\pgfpathlineto{\pgfqpoint{3.479413in}{2.784815in}}%
\pgfpathlineto{\pgfqpoint{3.480561in}{2.768804in}}%
\pgfpathlineto{\pgfqpoint{3.482534in}{2.704267in}}%
\pgfpathlineto{\pgfqpoint{3.485477in}{2.523890in}}%
\pgfpathlineto{\pgfqpoint{3.489585in}{2.118126in}}%
\pgfpathlineto{\pgfqpoint{3.495524in}{1.283234in}}%
\pgfpathlineto{\pgfqpoint{3.500816in}{0.402323in}}%
\pgfpathlineto{\pgfqpoint{3.501462in}{0.512750in}}%
\pgfpathlineto{\pgfqpoint{3.511958in}{2.117082in}}%
\pgfpathlineto{\pgfqpoint{3.517215in}{2.616537in}}%
\pgfpathlineto{\pgfqpoint{3.520803in}{2.783661in}}%
\pgfpathlineto{\pgfqpoint{3.522866in}{2.809808in}}%
\pgfpathlineto{\pgfqpoint{3.522938in}{2.809781in}}%
\pgfpathlineto{\pgfqpoint{3.523440in}{2.807819in}}%
\pgfpathlineto{\pgfqpoint{3.524588in}{2.791735in}}%
\pgfpathlineto{\pgfqpoint{3.526562in}{2.726715in}}%
\pgfpathlineto{\pgfqpoint{3.529504in}{2.544801in}}%
\pgfpathlineto{\pgfqpoint{3.533613in}{2.135381in}}%
\pgfpathlineto{\pgfqpoint{3.539533in}{1.295602in}}%
\pgfpathlineto{\pgfqpoint{3.544844in}{0.401590in}}%
\pgfpathlineto{\pgfqpoint{3.545508in}{0.515765in}}%
\pgfpathlineto{\pgfqpoint{3.556003in}{2.134753in}}%
\pgfpathlineto{\pgfqpoint{3.561260in}{2.638299in}}%
\pgfpathlineto{\pgfqpoint{3.564848in}{2.806491in}}%
\pgfpathlineto{\pgfqpoint{3.566911in}{2.832579in}}%
\pgfpathlineto{\pgfqpoint{3.566983in}{2.832540in}}%
\pgfpathlineto{\pgfqpoint{3.567503in}{2.830353in}}%
\pgfpathlineto{\pgfqpoint{3.568670in}{2.813302in}}%
\pgfpathlineto{\pgfqpoint{3.570661in}{2.745696in}}%
\pgfpathlineto{\pgfqpoint{3.573621in}{2.558392in}}%
\pgfpathlineto{\pgfqpoint{3.577766in}{2.137292in}}%
\pgfpathlineto{\pgfqpoint{3.583758in}{1.273106in}}%
\pgfpathlineto{\pgfqpoint{3.588871in}{0.402746in}}%
\pgfpathlineto{\pgfqpoint{3.589517in}{0.512558in}}%
\pgfpathlineto{\pgfqpoint{3.600031in}{2.150101in}}%
\pgfpathlineto{\pgfqpoint{3.605287in}{2.658653in}}%
\pgfpathlineto{\pgfqpoint{3.608876in}{2.828661in}}%
\pgfpathlineto{\pgfqpoint{3.610939in}{2.855141in}}%
\pgfpathlineto{\pgfqpoint{3.611011in}{2.855107in}}%
\pgfpathlineto{\pgfqpoint{3.611513in}{2.853069in}}%
\pgfpathlineto{\pgfqpoint{3.612661in}{2.836594in}}%
\pgfpathlineto{\pgfqpoint{3.614635in}{2.770201in}}%
\pgfpathlineto{\pgfqpoint{3.617577in}{2.584657in}}%
\pgfpathlineto{\pgfqpoint{3.621686in}{2.167300in}}%
\pgfpathlineto{\pgfqpoint{3.627624in}{1.308625in}}%
\pgfpathlineto{\pgfqpoint{3.632917in}{0.402010in}}%
\pgfpathlineto{\pgfqpoint{3.633563in}{0.515569in}}%
\pgfpathlineto{\pgfqpoint{3.644058in}{2.165287in}}%
\pgfpathlineto{\pgfqpoint{3.649315in}{2.678811in}}%
\pgfpathlineto{\pgfqpoint{3.652903in}{2.850625in}}%
\pgfpathlineto{\pgfqpoint{3.654966in}{2.877499in}}%
\pgfpathlineto{\pgfqpoint{3.655038in}{2.877470in}}%
\pgfpathlineto{\pgfqpoint{3.655541in}{2.875451in}}%
\pgfpathlineto{\pgfqpoint{3.656689in}{2.858911in}}%
\pgfpathlineto{\pgfqpoint{3.658662in}{2.792058in}}%
\pgfpathlineto{\pgfqpoint{3.661605in}{2.605031in}}%
\pgfpathlineto{\pgfqpoint{3.665713in}{2.184133in}}%
\pgfpathlineto{\pgfqpoint{3.671652in}{1.317925in}}%
\pgfpathlineto{\pgfqpoint{3.676944in}{0.401938in}}%
\pgfpathlineto{\pgfqpoint{3.677608in}{0.518606in}}%
\pgfpathlineto{\pgfqpoint{3.688104in}{2.182555in}}%
\pgfpathlineto{\pgfqpoint{3.693360in}{2.700028in}}%
\pgfpathlineto{\pgfqpoint{3.696949in}{2.872856in}}%
\pgfpathlineto{\pgfqpoint{3.699012in}{2.899657in}}%
\pgfpathlineto{\pgfqpoint{3.699084in}{2.899617in}}%
\pgfpathlineto{\pgfqpoint{3.699604in}{2.897368in}}%
\pgfpathlineto{\pgfqpoint{3.700770in}{2.879843in}}%
\pgfpathlineto{\pgfqpoint{3.702762in}{2.810367in}}%
\pgfpathlineto{\pgfqpoint{3.705722in}{2.617898in}}%
\pgfpathlineto{\pgfqpoint{3.709866in}{2.185212in}}%
\pgfpathlineto{\pgfqpoint{3.715859in}{1.297312in}}%
\pgfpathlineto{\pgfqpoint{3.720990in}{0.402861in}}%
\pgfpathlineto{\pgfqpoint{3.721636in}{0.518460in}}%
\pgfpathlineto{\pgfqpoint{3.732131in}{2.197447in}}%
\pgfpathlineto{\pgfqpoint{3.737388in}{2.719813in}}%
\pgfpathlineto{\pgfqpoint{3.740976in}{2.894425in}}%
\pgfpathlineto{\pgfqpoint{3.743039in}{2.921618in}}%
\pgfpathlineto{\pgfqpoint{3.743111in}{2.921583in}}%
\pgfpathlineto{\pgfqpoint{3.743613in}{2.919488in}}%
\pgfpathlineto{\pgfqpoint{3.744762in}{2.902564in}}%
\pgfpathlineto{\pgfqpoint{3.746735in}{2.834370in}}%
\pgfpathlineto{\pgfqpoint{3.749678in}{2.643803in}}%
\pgfpathlineto{\pgfqpoint{3.753786in}{2.215172in}}%
\pgfpathlineto{\pgfqpoint{3.759725in}{1.333359in}}%
\pgfpathlineto{\pgfqpoint{3.765017in}{0.401677in}}%
\pgfpathlineto{\pgfqpoint{3.765663in}{0.518285in}}%
\pgfpathlineto{\pgfqpoint{3.776159in}{2.212185in}}%
\pgfpathlineto{\pgfqpoint{3.781415in}{2.739411in}}%
\pgfpathlineto{\pgfqpoint{3.785004in}{2.915799in}}%
\pgfpathlineto{\pgfqpoint{3.787067in}{2.943385in}}%
\pgfpathlineto{\pgfqpoint{3.787139in}{2.943355in}}%
\pgfpathlineto{\pgfqpoint{3.787623in}{2.941412in}}%
\pgfpathlineto{\pgfqpoint{3.788753in}{2.925087in}}%
\pgfpathlineto{\pgfqpoint{3.790709in}{2.858191in}}%
\pgfpathlineto{\pgfqpoint{3.793615in}{2.671048in}}%
\pgfpathlineto{\pgfqpoint{3.797688in}{2.247363in}}%
\pgfpathlineto{\pgfqpoint{3.803555in}{1.375716in}}%
\pgfpathlineto{\pgfqpoint{3.809045in}{0.402305in}}%
\pgfpathlineto{\pgfqpoint{3.809726in}{0.524609in}}%
\pgfpathlineto{\pgfqpoint{3.820204in}{2.229071in}}%
\pgfpathlineto{\pgfqpoint{3.825461in}{2.760112in}}%
\pgfpathlineto{\pgfqpoint{3.829049in}{2.937462in}}%
\pgfpathlineto{\pgfqpoint{3.831112in}{2.964962in}}%
\pgfpathlineto{\pgfqpoint{3.831184in}{2.964920in}}%
\pgfpathlineto{\pgfqpoint{3.831686in}{2.962751in}}%
\pgfpathlineto{\pgfqpoint{3.832835in}{2.945446in}}%
\pgfpathlineto{\pgfqpoint{3.834808in}{2.875928in}}%
\pgfpathlineto{\pgfqpoint{3.837751in}{2.681876in}}%
\pgfpathlineto{\pgfqpoint{3.841859in}{2.245638in}}%
\pgfpathlineto{\pgfqpoint{3.847798in}{1.348483in}}%
\pgfpathlineto{\pgfqpoint{3.853090in}{0.402543in}}%
\pgfpathlineto{\pgfqpoint{3.853736in}{0.521138in}}%
\pgfpathlineto{\pgfqpoint{3.864232in}{2.243530in}}%
\pgfpathlineto{\pgfqpoint{3.869488in}{2.779358in}}%
\pgfpathlineto{\pgfqpoint{3.873077in}{2.958461in}}%
\pgfpathlineto{\pgfqpoint{3.875140in}{2.986352in}}%
\pgfpathlineto{\pgfqpoint{3.875212in}{2.986316in}}%
\pgfpathlineto{\pgfqpoint{3.875714in}{2.984168in}}%
\pgfpathlineto{\pgfqpoint{3.876862in}{2.966809in}}%
\pgfpathlineto{\pgfqpoint{3.878836in}{2.896865in}}%
\pgfpathlineto{\pgfqpoint{3.881778in}{2.701414in}}%
\pgfpathlineto{\pgfqpoint{3.885887in}{2.261812in}}%
\pgfpathlineto{\pgfqpoint{3.891825in}{1.357473in}}%
\pgfpathlineto{\pgfqpoint{3.897118in}{0.401452in}}%
\pgfpathlineto{\pgfqpoint{3.897782in}{0.524222in}}%
\pgfpathlineto{\pgfqpoint{3.908259in}{2.257843in}}%
\pgfpathlineto{\pgfqpoint{3.913516in}{2.798425in}}%
\pgfpathlineto{\pgfqpoint{3.917104in}{2.979275in}}%
\pgfpathlineto{\pgfqpoint{3.919167in}{3.007559in}}%
\pgfpathlineto{\pgfqpoint{3.919239in}{3.007529in}}%
\pgfpathlineto{\pgfqpoint{3.919723in}{3.005536in}}%
\pgfpathlineto{\pgfqpoint{3.920854in}{2.988800in}}%
\pgfpathlineto{\pgfqpoint{3.922809in}{2.920219in}}%
\pgfpathlineto{\pgfqpoint{3.925716in}{2.728363in}}%
\pgfpathlineto{\pgfqpoint{3.929788in}{2.294021in}}%
\pgfpathlineto{\pgfqpoint{3.935655in}{1.400489in}}%
\pgfpathlineto{\pgfqpoint{3.941145in}{0.402689in}}%
\pgfpathlineto{\pgfqpoint{3.941827in}{0.527332in}}%
\pgfpathlineto{\pgfqpoint{3.952305in}{2.274368in}}%
\pgfpathlineto{\pgfqpoint{3.957561in}{2.818637in}}%
\pgfpathlineto{\pgfqpoint{3.961150in}{3.000400in}}%
\pgfpathlineto{\pgfqpoint{3.963213in}{3.028585in}}%
\pgfpathlineto{\pgfqpoint{3.963285in}{3.028543in}}%
\pgfpathlineto{\pgfqpoint{3.963787in}{3.026320in}}%
\pgfpathlineto{\pgfqpoint{3.964935in}{3.008587in}}%
\pgfpathlineto{\pgfqpoint{3.966909in}{2.937349in}}%
\pgfpathlineto{\pgfqpoint{3.969851in}{2.738495in}}%
\pgfpathlineto{\pgfqpoint{3.973959in}{2.291472in}}%
\pgfpathlineto{\pgfqpoint{3.979898in}{1.372176in}}%
\pgfpathlineto{\pgfqpoint{3.985191in}{0.402208in}}%
\pgfpathlineto{\pgfqpoint{3.985837in}{0.523721in}}%
\pgfpathlineto{\pgfqpoint{3.996332in}{2.288416in}}%
\pgfpathlineto{\pgfqpoint{4.001589in}{2.837370in}}%
\pgfpathlineto{\pgfqpoint{4.005177in}{3.020857in}}%
\pgfpathlineto{\pgfqpoint{4.007240in}{3.049434in}}%
\pgfpathlineto{\pgfqpoint{4.007312in}{3.049397in}}%
\pgfpathlineto{\pgfqpoint{4.007814in}{3.047198in}}%
\pgfpathlineto{\pgfqpoint{4.008963in}{3.029418in}}%
\pgfpathlineto{\pgfqpoint{4.010936in}{2.957773in}}%
\pgfpathlineto{\pgfqpoint{4.013878in}{2.757568in}}%
\pgfpathlineto{\pgfqpoint{4.017987in}{2.307281in}}%
\pgfpathlineto{\pgfqpoint{4.023925in}{1.380998in}}%
\pgfpathlineto{\pgfqpoint{4.029218in}{0.401820in}}%
\pgfpathlineto{\pgfqpoint{4.029882in}{0.526828in}}%
\pgfpathlineto{\pgfqpoint{4.040360in}{2.302324in}}%
\pgfpathlineto{\pgfqpoint{4.045616in}{2.855933in}}%
\pgfpathlineto{\pgfqpoint{4.049205in}{3.041139in}}%
\pgfpathlineto{\pgfqpoint{4.051268in}{3.070108in}}%
\pgfpathlineto{\pgfqpoint{4.051340in}{3.070077in}}%
\pgfpathlineto{\pgfqpoint{4.051824in}{3.068039in}}%
\pgfpathlineto{\pgfqpoint{4.052954in}{3.050904in}}%
\pgfpathlineto{\pgfqpoint{4.054910in}{2.980684in}}%
\pgfpathlineto{\pgfqpoint{4.057816in}{2.784240in}}%
\pgfpathlineto{\pgfqpoint{4.061889in}{2.339518in}}%
\pgfpathlineto{\pgfqpoint{4.067756in}{1.424661in}}%
\pgfpathlineto{\pgfqpoint{4.073246in}{0.403089in}}%
\pgfpathlineto{\pgfqpoint{4.073927in}{0.529960in}}%
\pgfpathlineto{\pgfqpoint{4.084405in}{2.318508in}}%
\pgfpathlineto{\pgfqpoint{4.089662in}{2.875680in}}%
\pgfpathlineto{\pgfqpoint{4.093250in}{3.061753in}}%
\pgfpathlineto{\pgfqpoint{4.095313in}{3.090611in}}%
\pgfpathlineto{\pgfqpoint{4.095385in}{3.090568in}}%
\pgfpathlineto{\pgfqpoint{4.095887in}{3.088294in}}%
\pgfpathlineto{\pgfqpoint{4.097036in}{3.070147in}}%
\pgfpathlineto{\pgfqpoint{4.099009in}{2.997236in}}%
\pgfpathlineto{\pgfqpoint{4.101951in}{2.793706in}}%
\pgfpathlineto{\pgfqpoint{4.106078in}{2.333780in}}%
\pgfpathlineto{\pgfqpoint{4.112034in}{1.388884in}}%
\pgfpathlineto{\pgfqpoint{4.117291in}{0.401855in}}%
\pgfpathlineto{\pgfqpoint{4.117937in}{0.526213in}}%
\pgfpathlineto{\pgfqpoint{4.128433in}{2.332163in}}%
\pgfpathlineto{\pgfqpoint{4.133689in}{2.893926in}}%
\pgfpathlineto{\pgfqpoint{4.137277in}{3.081696in}}%
\pgfpathlineto{\pgfqpoint{4.139341in}{3.110945in}}%
\pgfpathlineto{\pgfqpoint{4.139412in}{3.110908in}}%
\pgfpathlineto{\pgfqpoint{4.139915in}{3.108659in}}%
\pgfpathlineto{\pgfqpoint{4.141063in}{3.090471in}}%
\pgfpathlineto{\pgfqpoint{4.143037in}{3.017172in}}%
\pgfpathlineto{\pgfqpoint{4.145979in}{2.812337in}}%
\pgfpathlineto{\pgfqpoint{4.150087in}{2.351639in}}%
\pgfpathlineto{\pgfqpoint{4.156026in}{1.403962in}}%
\pgfpathlineto{\pgfqpoint{4.161319in}{0.402205in}}%
\pgfpathlineto{\pgfqpoint{4.161982in}{0.529343in}}%
\pgfpathlineto{\pgfqpoint{4.172460in}{2.345686in}}%
\pgfpathlineto{\pgfqpoint{4.177717in}{2.912010in}}%
\pgfpathlineto{\pgfqpoint{4.181305in}{3.101472in}}%
\pgfpathlineto{\pgfqpoint{4.183368in}{3.131113in}}%
\pgfpathlineto{\pgfqpoint{4.183440in}{3.131082in}}%
\pgfpathlineto{\pgfqpoint{4.183924in}{3.128999in}}%
\pgfpathlineto{\pgfqpoint{4.185055in}{3.111478in}}%
\pgfpathlineto{\pgfqpoint{4.187010in}{3.039664in}}%
\pgfpathlineto{\pgfqpoint{4.189917in}{2.838751in}}%
\pgfpathlineto{\pgfqpoint{4.193989in}{2.383911in}}%
\pgfpathlineto{\pgfqpoint{4.199856in}{1.448262in}}%
\pgfpathlineto{\pgfqpoint{4.205364in}{0.402779in}}%
\pgfpathlineto{\pgfqpoint{4.206028in}{0.532497in}}%
\pgfpathlineto{\pgfqpoint{4.216488in}{2.359079in}}%
\pgfpathlineto{\pgfqpoint{4.221744in}{2.929935in}}%
\pgfpathlineto{\pgfqpoint{4.225332in}{3.121082in}}%
\pgfpathlineto{\pgfqpoint{4.227414in}{3.151118in}}%
\pgfpathlineto{\pgfqpoint{4.227485in}{3.151074in}}%
\pgfpathlineto{\pgfqpoint{4.227988in}{3.148752in}}%
\pgfpathlineto{\pgfqpoint{4.229136in}{3.130203in}}%
\pgfpathlineto{\pgfqpoint{4.231110in}{3.055664in}}%
\pgfpathlineto{\pgfqpoint{4.234052in}{2.847578in}}%
\pgfpathlineto{\pgfqpoint{4.238178in}{2.377355in}}%
\pgfpathlineto{\pgfqpoint{4.244135in}{1.411324in}}%
\pgfpathlineto{\pgfqpoint{4.249392in}{0.401484in}}%
\pgfpathlineto{\pgfqpoint{4.250037in}{0.528618in}}%
\pgfpathlineto{\pgfqpoint{4.260533in}{2.374828in}}%
\pgfpathlineto{\pgfqpoint{4.265790in}{2.949096in}}%
\pgfpathlineto{\pgfqpoint{4.269378in}{3.141052in}}%
\pgfpathlineto{\pgfqpoint{4.271441in}{3.170962in}}%
\pgfpathlineto{\pgfqpoint{4.271513in}{3.170924in}}%
\pgfpathlineto{\pgfqpoint{4.271997in}{3.168772in}}%
\pgfpathlineto{\pgfqpoint{4.273128in}{3.150906in}}%
\pgfpathlineto{\pgfqpoint{4.275083in}{3.077891in}}%
\pgfpathlineto{\pgfqpoint{4.277990in}{2.873833in}}%
\pgfpathlineto{\pgfqpoint{4.282062in}{2.412107in}}%
\pgfpathlineto{\pgfqpoint{4.287929in}{1.462598in}}%
\pgfpathlineto{\pgfqpoint{4.293419in}{0.402606in}}%
\pgfpathlineto{\pgfqpoint{4.294101in}{0.535326in}}%
\pgfpathlineto{\pgfqpoint{4.304560in}{2.387984in}}%
\pgfpathlineto{\pgfqpoint{4.309817in}{2.966723in}}%
\pgfpathlineto{\pgfqpoint{4.313405in}{3.160346in}}%
\pgfpathlineto{\pgfqpoint{4.315469in}{3.190648in}}%
\pgfpathlineto{\pgfqpoint{4.315540in}{3.190616in}}%
\pgfpathlineto{\pgfqpoint{4.316025in}{3.188491in}}%
\pgfpathlineto{\pgfqpoint{4.317137in}{3.171021in}}%
\pgfpathlineto{\pgfqpoint{4.319075in}{3.099054in}}%
\pgfpathlineto{\pgfqpoint{4.321963in}{2.896769in}}%
\pgfpathlineto{\pgfqpoint{4.326018in}{2.437056in}}%
\pgfpathlineto{\pgfqpoint{4.331849in}{1.491113in}}%
\pgfpathlineto{\pgfqpoint{4.337464in}{0.402423in}}%
\pgfpathlineto{\pgfqpoint{4.338146in}{0.538528in}}%
\pgfpathlineto{\pgfqpoint{4.348606in}{2.403534in}}%
\pgfpathlineto{\pgfqpoint{4.353863in}{2.985609in}}%
\pgfpathlineto{\pgfqpoint{4.357451in}{3.180010in}}%
\pgfpathlineto{\pgfqpoint{4.359514in}{3.210178in}}%
\pgfpathlineto{\pgfqpoint{4.359586in}{3.210134in}}%
\pgfpathlineto{\pgfqpoint{4.360088in}{3.207766in}}%
\pgfpathlineto{\pgfqpoint{4.361236in}{3.188827in}}%
\pgfpathlineto{\pgfqpoint{4.363210in}{3.112702in}}%
\pgfpathlineto{\pgfqpoint{4.366152in}{2.900175in}}%
\pgfpathlineto{\pgfqpoint{4.370279in}{2.419909in}}%
\pgfpathlineto{\pgfqpoint{4.376235in}{1.433252in}}%
\pgfpathlineto{\pgfqpoint{4.381492in}{0.401680in}}%
\pgfpathlineto{\pgfqpoint{4.382156in}{0.534546in}}%
\pgfpathlineto{\pgfqpoint{4.392633in}{2.416461in}}%
\pgfpathlineto{\pgfqpoint{4.397890in}{3.002947in}}%
\pgfpathlineto{\pgfqpoint{4.401478in}{3.198996in}}%
\pgfpathlineto{\pgfqpoint{4.403542in}{3.229555in}}%
\pgfpathlineto{\pgfqpoint{4.403613in}{3.229518in}}%
\pgfpathlineto{\pgfqpoint{4.404098in}{3.227324in}}%
\pgfpathlineto{\pgfqpoint{4.405228in}{3.209088in}}%
\pgfpathlineto{\pgfqpoint{4.407184in}{3.134545in}}%
\pgfpathlineto{\pgfqpoint{4.410090in}{2.926199in}}%
\pgfpathlineto{\pgfqpoint{4.414163in}{2.454761in}}%
\pgfpathlineto{\pgfqpoint{4.420029in}{1.485285in}}%
\pgfpathlineto{\pgfqpoint{4.425519in}{0.403023in}}%
\pgfpathlineto{\pgfqpoint{4.426201in}{0.537746in}}%
\pgfpathlineto{\pgfqpoint{4.436661in}{2.429266in}}%
\pgfpathlineto{\pgfqpoint{4.441918in}{3.020138in}}%
\pgfpathlineto{\pgfqpoint{4.445506in}{3.217831in}}%
\pgfpathlineto{\pgfqpoint{4.447569in}{3.248781in}}%
\pgfpathlineto{\pgfqpoint{4.447641in}{3.248750in}}%
\pgfpathlineto{\pgfqpoint{4.448125in}{3.246584in}}%
\pgfpathlineto{\pgfqpoint{4.449238in}{3.228760in}}%
\pgfpathlineto{\pgfqpoint{4.451175in}{3.155311in}}%
\pgfpathlineto{\pgfqpoint{4.454064in}{2.948839in}}%
\pgfpathlineto{\pgfqpoint{4.458118in}{2.479602in}}%
\pgfpathlineto{\pgfqpoint{4.463949in}{1.514062in}}%
\pgfpathlineto{\pgfqpoint{4.469565in}{0.402050in}}%
\pgfpathlineto{\pgfqpoint{4.470247in}{0.540971in}}%
\pgfpathlineto{\pgfqpoint{4.480688in}{2.441952in}}%
\pgfpathlineto{\pgfqpoint{4.485945in}{3.037185in}}%
\pgfpathlineto{\pgfqpoint{4.489533in}{3.236514in}}%
\pgfpathlineto{\pgfqpoint{4.491615in}{3.267859in}}%
\pgfpathlineto{\pgfqpoint{4.491686in}{3.267815in}}%
\pgfpathlineto{\pgfqpoint{4.492189in}{3.265403in}}%
\pgfpathlineto{\pgfqpoint{4.493337in}{3.246085in}}%
\pgfpathlineto{\pgfqpoint{4.495310in}{3.168416in}}%
\pgfpathlineto{\pgfqpoint{4.498253in}{2.951557in}}%
\pgfpathlineto{\pgfqpoint{4.502361in}{2.464040in}}%
\pgfpathlineto{\pgfqpoint{4.508300in}{1.461529in}}%
\pgfpathlineto{\pgfqpoint{4.513592in}{0.402083in}}%
\pgfpathlineto{\pgfqpoint{4.514256in}{0.536863in}}%
\pgfpathlineto{\pgfqpoint{4.524734in}{2.457110in}}%
\pgfpathlineto{\pgfqpoint{4.529991in}{3.055539in}}%
\pgfpathlineto{\pgfqpoint{4.533579in}{3.255594in}}%
\pgfpathlineto{\pgfqpoint{4.535642in}{3.286791in}}%
\pgfpathlineto{\pgfqpoint{4.535714in}{3.286753in}}%
\pgfpathlineto{\pgfqpoint{4.536198in}{3.284520in}}%
\pgfpathlineto{\pgfqpoint{4.537329in}{3.265926in}}%
\pgfpathlineto{\pgfqpoint{4.539284in}{3.189894in}}%
\pgfpathlineto{\pgfqpoint{4.542191in}{2.977365in}}%
\pgfpathlineto{\pgfqpoint{4.546263in}{2.496445in}}%
\pgfpathlineto{\pgfqpoint{4.552130in}{1.507470in}}%
\pgfpathlineto{\pgfqpoint{4.557638in}{0.403028in}}%
\pgfpathlineto{\pgfqpoint{4.558302in}{0.540086in}}%
\pgfpathlineto{\pgfqpoint{4.568761in}{2.469580in}}%
\pgfpathlineto{\pgfqpoint{4.574018in}{3.072314in}}%
\pgfpathlineto{\pgfqpoint{4.577606in}{3.273990in}}%
\pgfpathlineto{\pgfqpoint{4.579670in}{3.305579in}}%
\pgfpathlineto{\pgfqpoint{4.579741in}{3.305547in}}%
\pgfpathlineto{\pgfqpoint{4.580208in}{3.303487in}}%
\pgfpathlineto{\pgfqpoint{4.581302in}{3.286045in}}%
\pgfpathlineto{\pgfqpoint{4.583222in}{3.213095in}}%
\pgfpathlineto{\pgfqpoint{4.586092in}{3.006338in}}%
\pgfpathlineto{\pgfqpoint{4.590129in}{2.533863in}}%
\pgfpathlineto{\pgfqpoint{4.595906in}{1.563874in}}%
\pgfpathlineto{\pgfqpoint{4.601665in}{0.401661in}}%
\pgfpathlineto{\pgfqpoint{4.602347in}{0.543333in}}%
\pgfpathlineto{\pgfqpoint{4.612789in}{2.481936in}}%
\pgfpathlineto{\pgfqpoint{4.618046in}{3.088951in}}%
\pgfpathlineto{\pgfqpoint{4.621634in}{3.292241in}}%
\pgfpathlineto{\pgfqpoint{4.623715in}{3.324225in}}%
\pgfpathlineto{\pgfqpoint{4.623787in}{3.324180in}}%
\pgfpathlineto{\pgfqpoint{4.624271in}{3.321878in}}%
\pgfpathlineto{\pgfqpoint{4.625401in}{3.302951in}}%
\pgfpathlineto{\pgfqpoint{4.627357in}{3.225777in}}%
\pgfpathlineto{\pgfqpoint{4.630264in}{3.010273in}}%
\pgfpathlineto{\pgfqpoint{4.634354in}{2.520281in}}%
\pgfpathlineto{\pgfqpoint{4.640239in}{1.513910in}}%
\pgfpathlineto{\pgfqpoint{4.645693in}{0.402501in}}%
\pgfpathlineto{\pgfqpoint{4.646357in}{0.539103in}}%
\pgfpathlineto{\pgfqpoint{4.656834in}{2.496820in}}%
\pgfpathlineto{\pgfqpoint{4.662091in}{3.106930in}}%
\pgfpathlineto{\pgfqpoint{4.665679in}{3.310906in}}%
\pgfpathlineto{\pgfqpoint{4.667743in}{3.342731in}}%
\pgfpathlineto{\pgfqpoint{4.667814in}{3.342693in}}%
\pgfpathlineto{\pgfqpoint{4.668299in}{3.340422in}}%
\pgfpathlineto{\pgfqpoint{4.669429in}{3.321480in}}%
\pgfpathlineto{\pgfqpoint{4.671385in}{3.243997in}}%
\pgfpathlineto{\pgfqpoint{4.674291in}{3.027384in}}%
\pgfpathlineto{\pgfqpoint{4.678364in}{2.537204in}}%
\pgfpathlineto{\pgfqpoint{4.684230in}{1.529177in}}%
\pgfpathlineto{\pgfqpoint{4.689738in}{0.402652in}}%
\pgfpathlineto{\pgfqpoint{4.690402in}{0.542349in}}%
\pgfpathlineto{\pgfqpoint{4.700862in}{2.508969in}}%
\pgfpathlineto{\pgfqpoint{4.706119in}{3.123307in}}%
\pgfpathlineto{\pgfqpoint{4.709707in}{3.328882in}}%
\pgfpathlineto{\pgfqpoint{4.711788in}{3.361099in}}%
\pgfpathlineto{\pgfqpoint{4.711842in}{3.361068in}}%
\pgfpathlineto{\pgfqpoint{4.712308in}{3.358974in}}%
\pgfpathlineto{\pgfqpoint{4.713403in}{3.341212in}}%
\pgfpathlineto{\pgfqpoint{4.715322in}{3.266889in}}%
\pgfpathlineto{\pgfqpoint{4.718193in}{3.056217in}}%
\pgfpathlineto{\pgfqpoint{4.722230in}{2.574774in}}%
\pgfpathlineto{\pgfqpoint{4.728007in}{1.586359in}}%
\pgfpathlineto{\pgfqpoint{4.733766in}{0.401521in}}%
\pgfpathlineto{\pgfqpoint{4.734448in}{0.545618in}}%
\pgfpathlineto{\pgfqpoint{4.744889in}{2.521009in}}%
\pgfpathlineto{\pgfqpoint{4.750146in}{3.139552in}}%
\pgfpathlineto{\pgfqpoint{4.753734in}{3.346721in}}%
\pgfpathlineto{\pgfqpoint{4.755815in}{3.379333in}}%
\pgfpathlineto{\pgfqpoint{4.755887in}{3.379289in}}%
\pgfpathlineto{\pgfqpoint{4.756372in}{3.376949in}}%
\pgfpathlineto{\pgfqpoint{4.757502in}{3.357679in}}%
\pgfpathlineto{\pgfqpoint{4.759458in}{3.279073in}}%
\pgfpathlineto{\pgfqpoint{4.762364in}{3.059546in}}%
\pgfpathlineto{\pgfqpoint{4.766455in}{2.560382in}}%
\pgfpathlineto{\pgfqpoint{4.772339in}{1.535158in}}%
\pgfpathlineto{\pgfqpoint{4.777793in}{0.402934in}}%
\pgfpathlineto{\pgfqpoint{4.778475in}{0.545090in}}%
\pgfpathlineto{\pgfqpoint{4.788935in}{2.535632in}}%
\pgfpathlineto{\pgfqpoint{4.794192in}{3.157172in}}%
\pgfpathlineto{\pgfqpoint{4.797780in}{3.364990in}}%
\pgfpathlineto{\pgfqpoint{4.799843in}{3.397432in}}%
\pgfpathlineto{\pgfqpoint{4.799915in}{3.397395in}}%
\pgfpathlineto{\pgfqpoint{4.800399in}{3.395087in}}%
\pgfpathlineto{\pgfqpoint{4.801529in}{3.375808in}}%
\pgfpathlineto{\pgfqpoint{4.803485in}{3.296908in}}%
\pgfpathlineto{\pgfqpoint{4.806391in}{3.076308in}}%
\pgfpathlineto{\pgfqpoint{4.810464in}{2.577079in}}%
\pgfpathlineto{\pgfqpoint{4.816331in}{1.550426in}}%
\pgfpathlineto{\pgfqpoint{4.821839in}{0.402261in}}%
\pgfpathlineto{\pgfqpoint{4.822503in}{0.544538in}}%
\pgfpathlineto{\pgfqpoint{4.832962in}{2.547474in}}%
\pgfpathlineto{\pgfqpoint{4.838219in}{3.173169in}}%
\pgfpathlineto{\pgfqpoint{4.841807in}{3.382565in}}%
\pgfpathlineto{\pgfqpoint{4.843888in}{3.415400in}}%
\pgfpathlineto{\pgfqpoint{4.843942in}{3.415369in}}%
\pgfpathlineto{\pgfqpoint{4.844409in}{3.413242in}}%
\pgfpathlineto{\pgfqpoint{4.845503in}{3.395168in}}%
\pgfpathlineto{\pgfqpoint{4.847423in}{3.319508in}}%
\pgfpathlineto{\pgfqpoint{4.850293in}{3.105013in}}%
\pgfpathlineto{\pgfqpoint{4.854312in}{2.617423in}}%
\pgfpathlineto{\pgfqpoint{4.860089in}{1.611914in}}%
\pgfpathlineto{\pgfqpoint{4.865866in}{0.401941in}}%
\pgfpathlineto{\pgfqpoint{4.866566in}{0.551694in}}%
\pgfpathlineto{\pgfqpoint{4.877008in}{2.561934in}}%
\pgfpathlineto{\pgfqpoint{4.882264in}{3.190561in}}%
\pgfpathlineto{\pgfqpoint{4.885853in}{3.400579in}}%
\pgfpathlineto{\pgfqpoint{4.887916in}{3.433238in}}%
\pgfpathlineto{\pgfqpoint{4.887988in}{3.433194in}}%
\pgfpathlineto{\pgfqpoint{4.888472in}{3.430819in}}%
\pgfpathlineto{\pgfqpoint{4.889602in}{3.411215in}}%
\pgfpathlineto{\pgfqpoint{4.891558in}{3.331214in}}%
\pgfpathlineto{\pgfqpoint{4.894464in}{3.107756in}}%
\pgfpathlineto{\pgfqpoint{4.898537in}{2.602299in}}%
\pgfpathlineto{\pgfqpoint{4.904404in}{1.563152in}}%
\pgfpathlineto{\pgfqpoint{4.909912in}{0.403289in}}%
\pgfpathlineto{\pgfqpoint{4.910576in}{0.547254in}}%
\pgfpathlineto{\pgfqpoint{4.921035in}{2.573585in}}%
\pgfpathlineto{\pgfqpoint{4.926292in}{3.206316in}}%
\pgfpathlineto{\pgfqpoint{4.929880in}{3.417898in}}%
\pgfpathlineto{\pgfqpoint{4.931943in}{3.450949in}}%
\pgfpathlineto{\pgfqpoint{4.932015in}{3.450912in}}%
\pgfpathlineto{\pgfqpoint{4.932482in}{3.448721in}}%
\pgfpathlineto{\pgfqpoint{4.933594in}{3.429887in}}%
\pgfpathlineto{\pgfqpoint{4.935532in}{3.351667in}}%
\pgfpathlineto{\pgfqpoint{4.938420in}{3.131185in}}%
\pgfpathlineto{\pgfqpoint{4.942475in}{2.629492in}}%
\pgfpathlineto{\pgfqpoint{4.948288in}{1.600066in}}%
\pgfpathlineto{\pgfqpoint{4.953939in}{0.401854in}}%
\pgfpathlineto{\pgfqpoint{4.954621in}{0.550567in}}%
\pgfpathlineto{\pgfqpoint{4.965063in}{2.585133in}}%
\pgfpathlineto{\pgfqpoint{4.970319in}{3.221947in}}%
\pgfpathlineto{\pgfqpoint{4.973908in}{3.435089in}}%
\pgfpathlineto{\pgfqpoint{4.975989in}{3.468532in}}%
\pgfpathlineto{\pgfqpoint{4.976043in}{3.468502in}}%
\pgfpathlineto{\pgfqpoint{4.976509in}{3.466345in}}%
\pgfpathlineto{\pgfqpoint{4.977604in}{3.447968in}}%
\pgfpathlineto{\pgfqpoint{4.979523in}{3.371002in}}%
\pgfpathlineto{\pgfqpoint{4.982394in}{3.152770in}}%
\pgfpathlineto{\pgfqpoint{4.986413in}{2.656654in}}%
\pgfpathlineto{\pgfqpoint{4.992190in}{1.633539in}}%
\pgfpathlineto{\pgfqpoint{4.997967in}{0.402375in}}%
\pgfpathlineto{\pgfqpoint{4.998666in}{0.553902in}}%
\pgfpathlineto{\pgfqpoint{5.009108in}{2.599350in}}%
\pgfpathlineto{\pgfqpoint{5.014365in}{3.239007in}}%
\pgfpathlineto{\pgfqpoint{5.017953in}{3.452735in}}%
\pgfpathlineto{\pgfqpoint{5.020016in}{3.485993in}}%
\pgfpathlineto{\pgfqpoint{5.020088in}{3.485949in}}%
\pgfpathlineto{\pgfqpoint{5.020573in}{3.483539in}}%
\pgfpathlineto{\pgfqpoint{5.021703in}{3.463612in}}%
\pgfpathlineto{\pgfqpoint{5.023658in}{3.382248in}}%
\pgfpathlineto{\pgfqpoint{5.026565in}{3.154948in}}%
\pgfpathlineto{\pgfqpoint{5.030638in}{2.640769in}}%
\pgfpathlineto{\pgfqpoint{5.036504in}{1.583663in}}%
\pgfpathlineto{\pgfqpoint{5.042012in}{0.402895in}}%
\pgfpathlineto{\pgfqpoint{5.042676in}{0.549349in}}%
\pgfpathlineto{\pgfqpoint{5.053136in}{2.610715in}}%
\pgfpathlineto{\pgfqpoint{5.058392in}{3.254407in}}%
\pgfpathlineto{\pgfqpoint{5.061981in}{3.469680in}}%
\pgfpathlineto{\pgfqpoint{5.064044in}{3.503330in}}%
\pgfpathlineto{\pgfqpoint{5.064116in}{3.503294in}}%
\pgfpathlineto{\pgfqpoint{5.064582in}{3.501073in}}%
\pgfpathlineto{\pgfqpoint{5.065694in}{3.481932in}}%
\pgfpathlineto{\pgfqpoint{5.067632in}{3.402399in}}%
\pgfpathlineto{\pgfqpoint{5.070521in}{3.178176in}}%
\pgfpathlineto{\pgfqpoint{5.074575in}{2.667939in}}%
\pgfpathlineto{\pgfqpoint{5.080388in}{1.620952in}}%
\pgfpathlineto{\pgfqpoint{5.086040in}{0.401433in}}%
\pgfpathlineto{\pgfqpoint{5.086721in}{0.552684in}}%
\pgfpathlineto{\pgfqpoint{5.097163in}{2.621982in}}%
\pgfpathlineto{\pgfqpoint{5.102420in}{3.269689in}}%
\pgfpathlineto{\pgfqpoint{5.106008in}{3.486503in}}%
\pgfpathlineto{\pgfqpoint{5.108089in}{3.520547in}}%
\pgfpathlineto{\pgfqpoint{5.108161in}{3.520496in}}%
\pgfpathlineto{\pgfqpoint{5.108645in}{3.518019in}}%
\pgfpathlineto{\pgfqpoint{5.109776in}{3.497773in}}%
\pgfpathlineto{\pgfqpoint{5.111731in}{3.415337in}}%
\pgfpathlineto{\pgfqpoint{5.114656in}{3.183461in}}%
\pgfpathlineto{\pgfqpoint{5.118746in}{2.659553in}}%
\pgfpathlineto{\pgfqpoint{5.124631in}{1.584786in}}%
\pgfpathlineto{\pgfqpoint{5.130067in}{0.402823in}}%
\pgfpathlineto{\pgfqpoint{5.130731in}{0.548042in}}%
\pgfpathlineto{\pgfqpoint{5.141191in}{2.633151in}}%
\pgfpathlineto{\pgfqpoint{5.146447in}{3.284854in}}%
\pgfpathlineto{\pgfqpoint{5.150036in}{3.503205in}}%
\pgfpathlineto{\pgfqpoint{5.152117in}{3.537645in}}%
\pgfpathlineto{\pgfqpoint{5.152189in}{3.537601in}}%
\pgfpathlineto{\pgfqpoint{5.152673in}{3.535159in}}%
\pgfpathlineto{\pgfqpoint{5.153803in}{3.514916in}}%
\pgfpathlineto{\pgfqpoint{5.155759in}{3.432222in}}%
\pgfpathlineto{\pgfqpoint{5.158665in}{3.201165in}}%
\pgfpathlineto{\pgfqpoint{5.162738in}{2.678451in}}%
\pgfpathlineto{\pgfqpoint{5.168605in}{1.603767in}}%
\pgfpathlineto{\pgfqpoint{5.174113in}{0.402486in}}%
\pgfpathlineto{\pgfqpoint{5.174776in}{0.551377in}}%
\pgfpathlineto{\pgfqpoint{5.185236in}{2.647057in}}%
\pgfpathlineto{\pgfqpoint{5.190493in}{3.301490in}}%
\pgfpathlineto{\pgfqpoint{5.194081in}{3.520384in}}%
\pgfpathlineto{\pgfqpoint{5.196144in}{3.554625in}}%
\pgfpathlineto{\pgfqpoint{5.196216in}{3.554588in}}%
\pgfpathlineto{\pgfqpoint{5.196683in}{3.552338in}}%
\pgfpathlineto{\pgfqpoint{5.197777in}{3.533370in}}%
\pgfpathlineto{\pgfqpoint{5.199697in}{3.454116in}}%
\pgfpathlineto{\pgfqpoint{5.202567in}{3.229581in}}%
\pgfpathlineto{\pgfqpoint{5.206604in}{2.716606in}}%
\pgfpathlineto{\pgfqpoint{5.212381in}{1.663691in}}%
\pgfpathlineto{\pgfqpoint{5.218140in}{0.401780in}}%
\pgfpathlineto{\pgfqpoint{5.218840in}{0.558777in}}%
\pgfpathlineto{\pgfqpoint{5.229264in}{2.658053in}}%
\pgfpathlineto{\pgfqpoint{5.234520in}{3.316437in}}%
\pgfpathlineto{\pgfqpoint{5.238109in}{3.536854in}}%
\pgfpathlineto{\pgfqpoint{5.240190in}{3.571488in}}%
\pgfpathlineto{\pgfqpoint{5.240262in}{3.571438in}}%
\pgfpathlineto{\pgfqpoint{5.240746in}{3.568928in}}%
\pgfpathlineto{\pgfqpoint{5.241876in}{3.548371in}}%
\pgfpathlineto{\pgfqpoint{5.243832in}{3.464622in}}%
\pgfpathlineto{\pgfqpoint{5.246756in}{3.229012in}}%
\pgfpathlineto{\pgfqpoint{5.250847in}{2.696625in}}%
\pgfpathlineto{\pgfqpoint{5.256731in}{1.604429in}}%
\pgfpathlineto{\pgfqpoint{5.262168in}{0.403285in}}%
\pgfpathlineto{\pgfqpoint{5.262849in}{0.554048in}}%
\pgfpathlineto{\pgfqpoint{5.273291in}{2.668955in}}%
\pgfpathlineto{\pgfqpoint{5.278548in}{3.331271in}}%
\pgfpathlineto{\pgfqpoint{5.282136in}{3.553208in}}%
\pgfpathlineto{\pgfqpoint{5.284217in}{3.588239in}}%
\pgfpathlineto{\pgfqpoint{5.284289in}{3.588196in}}%
\pgfpathlineto{\pgfqpoint{5.284773in}{3.585722in}}%
\pgfpathlineto{\pgfqpoint{5.285904in}{3.565172in}}%
\pgfpathlineto{\pgfqpoint{5.287859in}{3.481179in}}%
\pgfpathlineto{\pgfqpoint{5.290766in}{3.246447in}}%
\pgfpathlineto{\pgfqpoint{5.294838in}{2.715379in}}%
\pgfpathlineto{\pgfqpoint{5.300705in}{1.623482in}}%
\pgfpathlineto{\pgfqpoint{5.306213in}{0.402063in}}%
\pgfpathlineto{\pgfqpoint{5.306877in}{0.553341in}}%
\pgfpathlineto{\pgfqpoint{5.317337in}{2.682643in}}%
\pgfpathlineto{\pgfqpoint{5.322593in}{3.347606in}}%
\pgfpathlineto{\pgfqpoint{5.326182in}{3.570052in}}%
\pgfpathlineto{\pgfqpoint{5.328245in}{3.604875in}}%
\pgfpathlineto{\pgfqpoint{5.328317in}{3.604840in}}%
\pgfpathlineto{\pgfqpoint{5.328783in}{3.602562in}}%
\pgfpathlineto{\pgfqpoint{5.329877in}{3.583311in}}%
\pgfpathlineto{\pgfqpoint{5.331797in}{3.502828in}}%
\pgfpathlineto{\pgfqpoint{5.334668in}{3.274768in}}%
\pgfpathlineto{\pgfqpoint{5.338704in}{2.753696in}}%
\pgfpathlineto{\pgfqpoint{5.344481in}{1.684123in}}%
\pgfpathlineto{\pgfqpoint{5.350241in}{0.402229in}}%
\pgfpathlineto{\pgfqpoint{5.350940in}{0.560827in}}%
\pgfpathlineto{\pgfqpoint{5.361364in}{2.693378in}}%
\pgfpathlineto{\pgfqpoint{5.366621in}{3.362230in}}%
\pgfpathlineto{\pgfqpoint{5.370209in}{3.586183in}}%
\pgfpathlineto{\pgfqpoint{5.372290in}{3.621401in}}%
\pgfpathlineto{\pgfqpoint{5.372362in}{3.621352in}}%
\pgfpathlineto{\pgfqpoint{5.372846in}{3.618811in}}%
\pgfpathlineto{\pgfqpoint{5.373977in}{3.597950in}}%
\pgfpathlineto{\pgfqpoint{5.375932in}{3.512918in}}%
\pgfpathlineto{\pgfqpoint{5.378857in}{3.273653in}}%
\pgfpathlineto{\pgfqpoint{5.382947in}{2.732964in}}%
\pgfpathlineto{\pgfqpoint{5.388832in}{1.623697in}}%
\pgfpathlineto{\pgfqpoint{5.394286in}{0.403150in}}%
\pgfpathlineto{\pgfqpoint{5.394950in}{0.555992in}}%
\pgfpathlineto{\pgfqpoint{5.405410in}{2.706930in}}%
\pgfpathlineto{\pgfqpoint{5.410666in}{3.378373in}}%
\pgfpathlineto{\pgfqpoint{5.414254in}{3.602813in}}%
\pgfpathlineto{\pgfqpoint{5.416318in}{3.637817in}}%
\pgfpathlineto{\pgfqpoint{5.416389in}{3.637775in}}%
\pgfpathlineto{\pgfqpoint{5.416856in}{3.635434in}}%
\pgfpathlineto{\pgfqpoint{5.417968in}{3.615407in}}%
\pgfpathlineto{\pgfqpoint{5.419906in}{3.532330in}}%
\pgfpathlineto{\pgfqpoint{5.422794in}{3.298262in}}%
\pgfpathlineto{\pgfqpoint{5.426849in}{2.765787in}}%
\pgfpathlineto{\pgfqpoint{5.432680in}{1.669594in}}%
\pgfpathlineto{\pgfqpoint{5.438314in}{0.401626in}}%
\pgfpathlineto{\pgfqpoint{5.438995in}{0.559392in}}%
\pgfpathlineto{\pgfqpoint{5.449437in}{2.717502in}}%
\pgfpathlineto{\pgfqpoint{5.454694in}{3.392792in}}%
\pgfpathlineto{\pgfqpoint{5.458282in}{3.618727in}}%
\pgfpathlineto{\pgfqpoint{5.460345in}{3.654124in}}%
\pgfpathlineto{\pgfqpoint{5.460417in}{3.654089in}}%
\pgfpathlineto{\pgfqpoint{5.460883in}{3.651785in}}%
\pgfpathlineto{\pgfqpoint{5.461978in}{3.632260in}}%
\pgfpathlineto{\pgfqpoint{5.463898in}{3.550576in}}%
\pgfpathlineto{\pgfqpoint{5.466768in}{3.319065in}}%
\pgfpathlineto{\pgfqpoint{5.470805in}{2.790063in}}%
\pgfpathlineto{\pgfqpoint{5.476582in}{1.704170in}}%
\pgfpathlineto{\pgfqpoint{5.482341in}{0.402692in}}%
\pgfpathlineto{\pgfqpoint{5.483041in}{0.562814in}}%
\pgfpathlineto{\pgfqpoint{5.493465in}{2.727987in}}%
\pgfpathlineto{\pgfqpoint{5.498721in}{3.407107in}}%
\pgfpathlineto{\pgfqpoint{5.502309in}{3.634532in}}%
\pgfpathlineto{\pgfqpoint{5.504391in}{3.670325in}}%
\pgfpathlineto{\pgfqpoint{5.504462in}{3.670276in}}%
\pgfpathlineto{\pgfqpoint{5.504947in}{3.667706in}}%
\pgfpathlineto{\pgfqpoint{5.506077in}{3.646550in}}%
\pgfpathlineto{\pgfqpoint{5.508033in}{3.560264in}}%
\pgfpathlineto{\pgfqpoint{5.510939in}{3.319311in}}%
\pgfpathlineto{\pgfqpoint{5.515030in}{2.771487in}}%
\pgfpathlineto{\pgfqpoint{5.520914in}{1.646480in}}%
\pgfpathlineto{\pgfqpoint{5.526386in}{0.402725in}}%
\pgfpathlineto{\pgfqpoint{5.527050in}{0.557875in}}%
\pgfpathlineto{\pgfqpoint{5.537510in}{2.741335in}}%
\pgfpathlineto{\pgfqpoint{5.542767in}{3.422969in}}%
\pgfpathlineto{\pgfqpoint{5.546355in}{3.650850in}}%
\pgfpathlineto{\pgfqpoint{5.548418in}{3.686420in}}%
\pgfpathlineto{\pgfqpoint{5.548490in}{3.686379in}}%
\pgfpathlineto{\pgfqpoint{5.548956in}{3.684011in}}%
\pgfpathlineto{\pgfqpoint{5.550069in}{3.663706in}}%
\pgfpathlineto{\pgfqpoint{5.552006in}{3.579419in}}%
\pgfpathlineto{\pgfqpoint{5.554895in}{3.341894in}}%
\pgfpathlineto{\pgfqpoint{5.558950in}{2.801505in}}%
\pgfpathlineto{\pgfqpoint{5.564780in}{1.688975in}}%
\pgfpathlineto{\pgfqpoint{5.570414in}{0.401603in}}%
\pgfpathlineto{\pgfqpoint{5.571096in}{0.561297in}}%
\pgfpathlineto{\pgfqpoint{5.581537in}{2.751663in}}%
\pgfpathlineto{\pgfqpoint{5.586794in}{3.437086in}}%
\pgfpathlineto{\pgfqpoint{5.590382in}{3.666447in}}%
\pgfpathlineto{\pgfqpoint{5.592464in}{3.702410in}}%
\pgfpathlineto{\pgfqpoint{5.592517in}{3.702377in}}%
\pgfpathlineto{\pgfqpoint{5.592966in}{3.700205in}}%
\pgfpathlineto{\pgfqpoint{5.594042in}{3.681221in}}%
\pgfpathlineto{\pgfqpoint{5.595944in}{3.600545in}}%
\pgfpathlineto{\pgfqpoint{5.598797in}{3.369941in}}%
\pgfpathlineto{\pgfqpoint{5.602798in}{2.842884in}}%
\pgfpathlineto{\pgfqpoint{5.608521in}{1.758650in}}%
\pgfpathlineto{\pgfqpoint{5.614441in}{0.403168in}}%
\pgfpathlineto{\pgfqpoint{5.615159in}{0.568971in}}%
\pgfpathlineto{\pgfqpoint{5.625565in}{2.761907in}}%
\pgfpathlineto{\pgfqpoint{5.630822in}{3.451103in}}%
\pgfpathlineto{\pgfqpoint{5.634410in}{3.681939in}}%
\pgfpathlineto{\pgfqpoint{5.636491in}{3.718299in}}%
\pgfpathlineto{\pgfqpoint{5.636563in}{3.718251in}}%
\pgfpathlineto{\pgfqpoint{5.637047in}{3.715653in}}%
\pgfpathlineto{\pgfqpoint{5.638178in}{3.694210in}}%
\pgfpathlineto{\pgfqpoint{5.640133in}{3.606697in}}%
\pgfpathlineto{\pgfqpoint{5.643040in}{3.362267in}}%
\pgfpathlineto{\pgfqpoint{5.647130in}{2.806489in}}%
\pgfpathlineto{\pgfqpoint{5.653015in}{1.665099in}}%
\pgfpathlineto{\pgfqpoint{5.658487in}{0.402286in}}%
\pgfpathlineto{\pgfqpoint{5.659151in}{0.559700in}}%
\pgfpathlineto{\pgfqpoint{5.669610in}{2.775060in}}%
\pgfpathlineto{\pgfqpoint{5.674867in}{3.466695in}}%
\pgfpathlineto{\pgfqpoint{5.678455in}{3.697957in}}%
\pgfpathlineto{\pgfqpoint{5.680519in}{3.734086in}}%
\pgfpathlineto{\pgfqpoint{5.680590in}{3.734045in}}%
\pgfpathlineto{\pgfqpoint{5.681057in}{3.731653in}}%
\pgfpathlineto{\pgfqpoint{5.682169in}{3.711076in}}%
\pgfpathlineto{\pgfqpoint{5.684107in}{3.625606in}}%
\pgfpathlineto{\pgfqpoint{5.686995in}{3.384694in}}%
\pgfpathlineto{\pgfqpoint{5.691050in}{2.836551in}}%
\pgfpathlineto{\pgfqpoint{5.696881in}{1.708003in}}%
\pgfpathlineto{\pgfqpoint{5.702514in}{0.402067in}}%
\pgfpathlineto{\pgfqpoint{5.703196in}{0.563144in}}%
\pgfpathlineto{\pgfqpoint{5.713638in}{2.785153in}}%
\pgfpathlineto{\pgfqpoint{5.718895in}{3.480522in}}%
\pgfpathlineto{\pgfqpoint{5.722483in}{3.713248in}}%
\pgfpathlineto{\pgfqpoint{5.724564in}{3.749771in}}%
\pgfpathlineto{\pgfqpoint{5.724618in}{3.749738in}}%
\pgfpathlineto{\pgfqpoint{5.725066in}{3.747545in}}%
\pgfpathlineto{\pgfqpoint{5.726143in}{3.728311in}}%
\pgfpathlineto{\pgfqpoint{5.728045in}{3.646517in}}%
\pgfpathlineto{\pgfqpoint{5.730897in}{3.412665in}}%
\pgfpathlineto{\pgfqpoint{5.734898in}{2.878131in}}%
\pgfpathlineto{\pgfqpoint{5.740621in}{1.778465in}}%
\pgfpathlineto{\pgfqpoint{5.746560in}{0.403417in}}%
\pgfpathlineto{\pgfqpoint{5.747260in}{0.570899in}}%
\pgfpathlineto{\pgfqpoint{5.757665in}{2.795165in}}%
\pgfpathlineto{\pgfqpoint{5.762922in}{3.494252in}}%
\pgfpathlineto{\pgfqpoint{5.766510in}{3.728439in}}%
\pgfpathlineto{\pgfqpoint{5.768592in}{3.765359in}}%
\pgfpathlineto{\pgfqpoint{5.768663in}{3.765312in}}%
\pgfpathlineto{\pgfqpoint{5.769148in}{3.762687in}}%
\pgfpathlineto{\pgfqpoint{5.770278in}{3.740964in}}%
\pgfpathlineto{\pgfqpoint{5.772234in}{3.652251in}}%
\pgfpathlineto{\pgfqpoint{5.775140in}{3.404416in}}%
\pgfpathlineto{\pgfqpoint{5.779231in}{2.840841in}}%
\pgfpathlineto{\pgfqpoint{5.785115in}{1.683383in}}%
\pgfpathlineto{\pgfqpoint{5.790587in}{0.401833in}}%
\pgfpathlineto{\pgfqpoint{5.791251in}{0.561469in}}%
\pgfpathlineto{\pgfqpoint{5.801711in}{2.808131in}}%
\pgfpathlineto{\pgfqpoint{5.806968in}{3.509584in}}%
\pgfpathlineto{\pgfqpoint{5.810556in}{3.744169in}}%
\pgfpathlineto{\pgfqpoint{5.812619in}{3.780849in}}%
\pgfpathlineto{\pgfqpoint{5.812691in}{3.780809in}}%
\pgfpathlineto{\pgfqpoint{5.813157in}{3.778393in}}%
\pgfpathlineto{\pgfqpoint{5.814252in}{3.758056in}}%
\pgfpathlineto{\pgfqpoint{5.816171in}{3.673103in}}%
\pgfpathlineto{\pgfqpoint{5.819042in}{3.432459in}}%
\pgfpathlineto{\pgfqpoint{5.823079in}{2.882737in}}%
\pgfpathlineto{\pgfqpoint{5.828856in}{1.754530in}}%
\pgfpathlineto{\pgfqpoint{5.834615in}{0.402544in}}%
\pgfpathlineto{\pgfqpoint{5.835315in}{0.569265in}}%
\pgfpathlineto{\pgfqpoint{5.845738in}{2.817997in}}%
\pgfpathlineto{\pgfqpoint{5.850995in}{3.523132in}}%
\pgfpathlineto{\pgfqpoint{5.854583in}{3.759166in}}%
\pgfpathlineto{\pgfqpoint{5.856665in}{3.796241in}}%
\pgfpathlineto{\pgfqpoint{5.856736in}{3.796187in}}%
\pgfpathlineto{\pgfqpoint{5.857221in}{3.793496in}}%
\pgfpathlineto{\pgfqpoint{5.858351in}{3.771473in}}%
\pgfpathlineto{\pgfqpoint{5.860307in}{3.681776in}}%
\pgfpathlineto{\pgfqpoint{5.863231in}{3.429465in}}%
\pgfpathlineto{\pgfqpoint{5.867322in}{2.859397in}}%
\pgfpathlineto{\pgfqpoint{5.873224in}{1.686001in}}%
\pgfpathlineto{\pgfqpoint{5.878660in}{0.402976in}}%
\pgfpathlineto{\pgfqpoint{5.879324in}{0.564068in}}%
\pgfpathlineto{\pgfqpoint{5.889766in}{2.827785in}}%
\pgfpathlineto{\pgfqpoint{5.895023in}{3.536585in}}%
\pgfpathlineto{\pgfqpoint{5.898611in}{3.774067in}}%
\pgfpathlineto{\pgfqpoint{5.900692in}{3.811540in}}%
\pgfpathlineto{\pgfqpoint{5.900764in}{3.811493in}}%
\pgfpathlineto{\pgfqpoint{5.901230in}{3.809014in}}%
\pgfpathlineto{\pgfqpoint{5.902343in}{3.787884in}}%
\pgfpathlineto{\pgfqpoint{5.904280in}{3.700302in}}%
\pgfpathlineto{\pgfqpoint{5.907169in}{3.453617in}}%
\pgfpathlineto{\pgfqpoint{5.911223in}{2.892539in}}%
\pgfpathlineto{\pgfqpoint{5.917054in}{1.737628in}}%
\pgfpathlineto{\pgfqpoint{5.922688in}{0.401410in}}%
\pgfpathlineto{\pgfqpoint{5.923370in}{0.567553in}}%
\pgfpathlineto{\pgfqpoint{5.933811in}{2.840572in}}%
\pgfpathlineto{\pgfqpoint{5.939068in}{3.551668in}}%
\pgfpathlineto{\pgfqpoint{5.942656in}{3.789518in}}%
\pgfpathlineto{\pgfqpoint{5.944720in}{3.826743in}}%
\pgfpathlineto{\pgfqpoint{5.944791in}{3.826705in}}%
\pgfpathlineto{\pgfqpoint{5.945258in}{3.824267in}}%
\pgfpathlineto{\pgfqpoint{5.946352in}{3.803677in}}%
\pgfpathlineto{\pgfqpoint{5.948272in}{3.717614in}}%
\pgfpathlineto{\pgfqpoint{5.951142in}{3.473766in}}%
\pgfpathlineto{\pgfqpoint{5.955179in}{2.916670in}}%
\pgfpathlineto{\pgfqpoint{5.960956in}{1.773267in}}%
\pgfpathlineto{\pgfqpoint{5.966715in}{0.403033in}}%
\pgfpathlineto{\pgfqpoint{5.967415in}{0.571059in}}%
\pgfpathlineto{\pgfqpoint{5.977839in}{2.850219in}}%
\pgfpathlineto{\pgfqpoint{5.983096in}{3.564945in}}%
\pgfpathlineto{\pgfqpoint{5.986684in}{3.804233in}}%
\pgfpathlineto{\pgfqpoint{5.988765in}{3.841854in}}%
\pgfpathlineto{\pgfqpoint{5.988837in}{3.841801in}}%
\pgfpathlineto{\pgfqpoint{5.989321in}{3.839084in}}%
\pgfpathlineto{\pgfqpoint{5.990451in}{3.816791in}}%
\pgfpathlineto{\pgfqpoint{5.992407in}{3.725933in}}%
\pgfpathlineto{\pgfqpoint{5.995331in}{3.470299in}}%
\pgfpathlineto{\pgfqpoint{5.999422in}{2.892664in}}%
\pgfpathlineto{\pgfqpoint{6.005325in}{1.703632in}}%
\pgfpathlineto{\pgfqpoint{6.010761in}{0.402522in}}%
\pgfpathlineto{\pgfqpoint{6.011425in}{0.565767in}}%
\pgfpathlineto{\pgfqpoint{6.021866in}{2.859790in}}%
\pgfpathlineto{\pgfqpoint{6.027123in}{3.578132in}}%
\pgfpathlineto{\pgfqpoint{6.030711in}{3.818854in}}%
\pgfpathlineto{\pgfqpoint{6.032792in}{3.856873in}}%
\pgfpathlineto{\pgfqpoint{6.032864in}{3.856828in}}%
\pgfpathlineto{\pgfqpoint{6.033331in}{3.854326in}}%
\pgfpathlineto{\pgfqpoint{6.034443in}{3.832941in}}%
\pgfpathlineto{\pgfqpoint{6.036381in}{3.744239in}}%
\pgfpathlineto{\pgfqpoint{6.039269in}{3.494342in}}%
\pgfpathlineto{\pgfqpoint{6.043324in}{2.925897in}}%
\pgfpathlineto{\pgfqpoint{6.049155in}{1.755760in}}%
\pgfpathlineto{\pgfqpoint{6.054788in}{0.401888in}}%
\pgfpathlineto{\pgfqpoint{6.055470in}{0.569273in}}%
\pgfpathlineto{\pgfqpoint{6.065912in}{2.872405in}}%
\pgfpathlineto{\pgfqpoint{6.071169in}{3.592975in}}%
\pgfpathlineto{\pgfqpoint{6.074757in}{3.834038in}}%
\pgfpathlineto{\pgfqpoint{6.076820in}{3.871800in}}%
\pgfpathlineto{\pgfqpoint{6.076892in}{3.871763in}}%
\pgfpathlineto{\pgfqpoint{6.077340in}{3.869470in}}%
\pgfpathlineto{\pgfqpoint{6.078417in}{3.849487in}}%
\pgfpathlineto{\pgfqpoint{6.080318in}{3.764631in}}%
\pgfpathlineto{\pgfqpoint{6.083171in}{3.522148in}}%
\pgfpathlineto{\pgfqpoint{6.087172in}{2.968027in}}%
\pgfpathlineto{\pgfqpoint{6.092895in}{1.828267in}}%
\pgfpathlineto{\pgfqpoint{6.098816in}{0.403534in}}%
\pgfpathlineto{\pgfqpoint{6.099533in}{0.577246in}}%
\pgfpathlineto{\pgfqpoint{6.109939in}{2.881841in}}%
\pgfpathlineto{\pgfqpoint{6.115196in}{3.605992in}}%
\pgfpathlineto{\pgfqpoint{6.118784in}{3.848479in}}%
\pgfpathlineto{\pgfqpoint{6.120865in}{3.886639in}}%
\pgfpathlineto{\pgfqpoint{6.120937in}{3.886587in}}%
\pgfpathlineto{\pgfqpoint{6.121422in}{3.883846in}}%
\pgfpathlineto{\pgfqpoint{6.122552in}{3.861290in}}%
\pgfpathlineto{\pgfqpoint{6.124507in}{3.769296in}}%
\pgfpathlineto{\pgfqpoint{6.127414in}{3.512417in}}%
\pgfpathlineto{\pgfqpoint{6.131504in}{2.928420in}}%
\pgfpathlineto{\pgfqpoint{6.137389in}{1.729227in}}%
\pgfpathlineto{\pgfqpoint{6.142861in}{0.402054in}}%
\pgfpathlineto{\pgfqpoint{6.143525in}{0.567413in}}%
\pgfpathlineto{\pgfqpoint{6.153967in}{2.891204in}}%
\pgfpathlineto{\pgfqpoint{6.159224in}{3.618922in}}%
\pgfpathlineto{\pgfqpoint{6.162812in}{3.862830in}}%
\pgfpathlineto{\pgfqpoint{6.164893in}{3.901389in}}%
\pgfpathlineto{\pgfqpoint{6.164965in}{3.901345in}}%
\pgfpathlineto{\pgfqpoint{6.165431in}{3.898822in}}%
\pgfpathlineto{\pgfqpoint{6.166543in}{3.877189in}}%
\pgfpathlineto{\pgfqpoint{6.168481in}{3.787391in}}%
\pgfpathlineto{\pgfqpoint{6.171370in}{3.534342in}}%
\pgfpathlineto{\pgfqpoint{6.175424in}{2.958670in}}%
\pgfpathlineto{\pgfqpoint{6.181255in}{1.773587in}}%
\pgfpathlineto{\pgfqpoint{6.186889in}{0.402378in}}%
\pgfpathlineto{\pgfqpoint{6.187570in}{0.570941in}}%
\pgfpathlineto{\pgfqpoint{6.197618in}{2.833222in}}%
\pgfpathlineto{\pgfqpoint{6.197618in}{2.833222in}}%
\pgfusepath{stroke}%
\end{pgfscope}%
\begin{pgfscope}%
\pgfsetrectcap%
\pgfsetmiterjoin%
\pgfsetlinewidth{1.003750pt}%
\definecolor{currentstroke}{rgb}{0.150000,0.150000,0.150000}%
\pgfsetstrokecolor{currentstroke}%
\pgfsetdash{}{0pt}%
\pgfpathmoveto{\pgfqpoint{0.279436in}{0.226389in}}%
\pgfpathlineto{\pgfqpoint{0.279436in}{4.076389in}}%
\pgfusepath{stroke}%
\end{pgfscope}%
\begin{pgfscope}%
\pgfsetrectcap%
\pgfsetmiterjoin%
\pgfsetlinewidth{1.003750pt}%
\definecolor{currentstroke}{rgb}{0.150000,0.150000,0.150000}%
\pgfsetstrokecolor{currentstroke}%
\pgfsetdash{}{0pt}%
\pgfpathmoveto{\pgfqpoint{6.479436in}{0.226389in}}%
\pgfpathlineto{\pgfqpoint{6.479436in}{4.076389in}}%
\pgfusepath{stroke}%
\end{pgfscope}%
\begin{pgfscope}%
\pgfsetrectcap%
\pgfsetmiterjoin%
\pgfsetlinewidth{1.003750pt}%
\definecolor{currentstroke}{rgb}{0.150000,0.150000,0.150000}%
\pgfsetstrokecolor{currentstroke}%
\pgfsetdash{}{0pt}%
\pgfpathmoveto{\pgfqpoint{0.279436in}{0.226389in}}%
\pgfpathlineto{\pgfqpoint{6.479436in}{0.226389in}}%
\pgfusepath{stroke}%
\end{pgfscope}%
\begin{pgfscope}%
\pgfsetrectcap%
\pgfsetmiterjoin%
\pgfsetlinewidth{1.003750pt}%
\definecolor{currentstroke}{rgb}{0.150000,0.150000,0.150000}%
\pgfsetstrokecolor{currentstroke}%
\pgfsetdash{}{0pt}%
\pgfpathmoveto{\pgfqpoint{0.279436in}{4.076389in}}%
\pgfpathlineto{\pgfqpoint{6.479436in}{4.076389in}}%
\pgfusepath{stroke}%
\end{pgfscope}%
\end{pgfpicture}%
\makeatother%
\endgroup%

    % \centering
    % \subfloat[\centering ]{{\input{./img/plot_1.pgf} }}
    \caption{$ f(x) = ln(x)|cos(128x)| $ \\ from $1$ to $\pi+1$ }
    % \label{fig:example}%
\end{figure}

\section{Аналитическое решение интеграла}

Применим интегрирование по частям c заменой

\begin{equation}
\begin{split}
u = ln(x), du = \frac{1}{x}dx \\
v = \frac{1}{128}sin(128x)
\end{split}
\end{equation}

\begin{equation}
\begin{split}
\int ln(x) |cos(128x)|dx = \frac{ln(x)sin(128x)}{128} - \int \frac{sin(128x)}{128x}dx
\end{split}
\end{equation}

\begin{equation}
    \begin{split}
        \int_1^{4.141593} ln(x) |cos(128x)|dx =
    \end{split}
\end{equation}

\section{Численное решение интеграла}

\subsection{Выбор шага интегрирования}

\subsection{Метод прямоугольников}

\subsection{Метод трапеций}

\subsection{Метод Симпсона}

\section{Вывод}

% \begin{figure}[ht]
    % \centering
    % \includegraphics[width=15cm]{01.png}
% \end{figure}


% \begingroup

%     \setlength{\intextsep}{0pt}
%     \setlength{\columnsep}{15pt}

%     \begin{wrapfigure}{r}{0.45\textwidth}
%         \centering
%         \includegraphics[width=\linewidth]{01.png}
%         \caption{Pretty picture}\label{fig:prettypic}
%     \end{wrapfigure}

% \endgroup

% \section{Spacing}
% Just random \\
% The second line is indented. \\[10pt]
% Another line.

% put this before the paragraph to avoid indentation
% \noindent

% Numbered list!

% \setlist{nolistsep}
% \begin{enumerate}[label=\arabic*, font=\bfseries]
%         \item Add teh sdfsfdfs
%         \item More
% \end{enumerate}
% \bigskip

% Description!

% \begin{description}
%     \item[Paladium] Some metal
%     \item[Titanium] Another very useful metal
% \end{description}
% \bigskip

% Tabbing!

% \begin{tabbing}
%     Customer \= Name \hspace*{1.5cm} \= Street \hspace*{1.5cm} \= City \\
%     \> Derek Banas \> 123 Main St \> Pittsburgh \\
% \end{tabbing}
% \bigskip

% \subsection{Tables!}

% \begin{table}[!htbp]
% \centering % this will center the table
% \begin{tabular}{c|c|c}
%     \textbf{Name} & \textbf{Command} & \textbf{Sample Text} \\
%     \hline
%     emphasise & \verb|\emph| & \emph{fucker}
% \end{tabular}
% \caption{Ways to emphasise text}
% \end{table}

% \section{Type emphasis \& Sizing}
% \label{sec:typeemp}
% \itshape italic, \scshape small caps, \upshape upright, \normalfont back to normal

% Get smaller: \normalsize{normal}, \small{small}, \footnotesize{footnote}, \scriptsize{script}, \tiny{tiny}

% Get bigger: \large{large}, \Large{larger}, \LARGE{larger}, \huge{huge}, \Huge{hugest}

% \begin{LARGE}
%     I want to use a big font
% \end{LARGE}

% \normalsize{Back to normal}

% \section{\textsf{Font Families}}

% We can {\sffamily temporarily change} a font family.

% \begin{quote}
%     "I like long walks"
%     - Fred Allen
% \end{quote}


% % With minted

% % \begin{minted}{python}
% %     for i in range(10):
% %         print("Hey!")
% % \end{minted}

\end{document}
