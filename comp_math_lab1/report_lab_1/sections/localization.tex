\documentclass[../main.tex]{subfiles}
\graphicspath{{\subfix{../images/}}}
\begin{document}

\section{Локализация}

% \begin{figure}[ht]
    % \centering
    % \includegraphics[width=15cm]{../images/eq_plot_1.png}
% \end{figure}


Требуется найти непересекающиеся отрезки $[a_i, b_i]$, каждому из которых принадлежит один и только один корень данного уравнения $x_i^*$. \\
Пусть $f(x) = a_0 x^n + a_1 x^{n - 1} + ... + a_{n - 1}x + a_n = 0 $, тогда требуется

\setlist{nolistsep}
\begin{enumerate}
    \item Определить число этих корней $N = N_+ + N_-$;
    \begin{itemize}
        \item Т. Декарта: число положительных корней равно числу перемен знаков в последовательности коэффициентов $a_0, a_1, ... , a_n$ или на четное число меньше.
        \item Все корни многочлена (включая комплексные) лежат в кольце
            $$ \frac{|a_n|}{|a_n| + B} <= |z| <= 1 + \frac{A}{|a_0|} $$
        где $ A = max\{|a_1|,|a_2|,...,|a_n|\}$ , $B = max\{|a_0|,|a_1|,...,|a_{n−1}|\} $
    \end{itemize}

    \item Определить отрезки, на которых лежат все действительные корни;
    \item Найти $N$ непересекающихся отрезков $[a_i,b_i]$, для которых $f(a_i) f(b_i) < 0$.
\end{enumerate}

\subsection{Графики функций и выбор отрезков}



\begin{figure}[ht]
    \centering
    \subfloat[\centering $ 3x + 4x^3 - 12 x^2 - 5 = 0 $]{{%% Creator: Matplotlib, PGF backend
%%
%% To include the figure in your LaTeX document, write
%%   \input{<filename>.pgf}
%%
%% Make sure the required packages are loaded in your preamble
%%   \usepackage{pgf}
%%
%% Figures using additional raster images can only be included by \input if
%% they are in the same directory as the main LaTeX file. For loading figures
%% from other directories you can use the `import` package
%%   \usepackage{import}
%%
%% and then include the figures with
%%   \import{<path to file>}{<filename>.pgf}
%%
%% Matplotlib used the following preamble
%%
\begingroup%
\makeatletter%
\begin{pgfpicture}%
\pgfpathrectangle{\pgfpointorigin}{\pgfqpoint{2.568749in}{1.766389in}}%
\pgfusepath{use as bounding box, clip}%
\begin{pgfscope}%
\pgfsetbuttcap%
\pgfsetmiterjoin%
\pgfsetlinewidth{0.000000pt}%
\definecolor{currentstroke}{rgb}{0.000000,0.000000,0.000000}%
\pgfsetstrokecolor{currentstroke}%
\pgfsetstrokeopacity{0.000000}%
\pgfsetdash{}{0pt}%
\pgfpathmoveto{\pgfqpoint{0.000000in}{0.000000in}}%
\pgfpathlineto{\pgfqpoint{2.568749in}{0.000000in}}%
\pgfpathlineto{\pgfqpoint{2.568749in}{1.766389in}}%
\pgfpathlineto{\pgfqpoint{0.000000in}{1.766389in}}%
\pgfpathclose%
\pgfusepath{}%
\end{pgfscope}%
\begin{pgfscope}%
\pgfsetbuttcap%
\pgfsetmiterjoin%
\pgfsetlinewidth{0.000000pt}%
\definecolor{currentstroke}{rgb}{0.000000,0.000000,0.000000}%
\pgfsetstrokecolor{currentstroke}%
\pgfsetstrokeopacity{0.000000}%
\pgfsetdash{}{0pt}%
\pgfpathmoveto{\pgfqpoint{0.243749in}{0.226389in}}%
\pgfpathlineto{\pgfqpoint{2.568749in}{0.226389in}}%
\pgfpathlineto{\pgfqpoint{2.568749in}{1.766389in}}%
\pgfpathlineto{\pgfqpoint{0.243749in}{1.766389in}}%
\pgfpathclose%
\pgfusepath{}%
\end{pgfscope}%
\begin{pgfscope}%
\pgfpathrectangle{\pgfqpoint{0.243749in}{0.226389in}}{\pgfqpoint{2.325000in}{1.540000in}}%
\pgfusepath{clip}%
\pgfsetroundcap%
\pgfsetroundjoin%
\pgfsetlinewidth{0.803000pt}%
\definecolor{currentstroke}{rgb}{0.800000,0.800000,0.800000}%
\pgfsetstrokecolor{currentstroke}%
\pgfsetdash{}{0pt}%
\pgfpathmoveto{\pgfqpoint{0.349431in}{0.226389in}}%
\pgfpathlineto{\pgfqpoint{0.349431in}{1.766389in}}%
\pgfusepath{stroke}%
\end{pgfscope}%
\begin{pgfscope}%
\definecolor{textcolor}{rgb}{0.501961,0.501961,0.501961}%
\pgfsetstrokecolor{textcolor}%
\pgfsetfillcolor{textcolor}%
\pgftext[x=0.349431in,y=0.111111in,,top]{\color{textcolor}\rmfamily\fontsize{8.800000}{10.560000}\selectfont \(\displaystyle {2.0}\)}%
\end{pgfscope}%
\begin{pgfscope}%
\pgfpathrectangle{\pgfqpoint{0.243749in}{0.226389in}}{\pgfqpoint{2.325000in}{1.540000in}}%
\pgfusepath{clip}%
\pgfsetroundcap%
\pgfsetroundjoin%
\pgfsetlinewidth{0.803000pt}%
\definecolor{currentstroke}{rgb}{0.800000,0.800000,0.800000}%
\pgfsetstrokecolor{currentstroke}%
\pgfsetdash{}{0pt}%
\pgfpathmoveto{\pgfqpoint{0.877843in}{0.226389in}}%
\pgfpathlineto{\pgfqpoint{0.877843in}{1.766389in}}%
\pgfusepath{stroke}%
\end{pgfscope}%
\begin{pgfscope}%
\definecolor{textcolor}{rgb}{0.501961,0.501961,0.501961}%
\pgfsetstrokecolor{textcolor}%
\pgfsetfillcolor{textcolor}%
\pgftext[x=0.877843in,y=0.111111in,,top]{\color{textcolor}\rmfamily\fontsize{8.800000}{10.560000}\selectfont \(\displaystyle {2.5}\)}%
\end{pgfscope}%
\begin{pgfscope}%
\pgfpathrectangle{\pgfqpoint{0.243749in}{0.226389in}}{\pgfqpoint{2.325000in}{1.540000in}}%
\pgfusepath{clip}%
\pgfsetroundcap%
\pgfsetroundjoin%
\pgfsetlinewidth{0.803000pt}%
\definecolor{currentstroke}{rgb}{0.800000,0.800000,0.800000}%
\pgfsetstrokecolor{currentstroke}%
\pgfsetdash{}{0pt}%
\pgfpathmoveto{\pgfqpoint{1.406254in}{0.226389in}}%
\pgfpathlineto{\pgfqpoint{1.406254in}{1.766389in}}%
\pgfusepath{stroke}%
\end{pgfscope}%
\begin{pgfscope}%
\definecolor{textcolor}{rgb}{0.501961,0.501961,0.501961}%
\pgfsetstrokecolor{textcolor}%
\pgfsetfillcolor{textcolor}%
\pgftext[x=1.406254in,y=0.111111in,,top]{\color{textcolor}\rmfamily\fontsize{8.800000}{10.560000}\selectfont \(\displaystyle {3.0}\)}%
\end{pgfscope}%
\begin{pgfscope}%
\pgfpathrectangle{\pgfqpoint{0.243749in}{0.226389in}}{\pgfqpoint{2.325000in}{1.540000in}}%
\pgfusepath{clip}%
\pgfsetroundcap%
\pgfsetroundjoin%
\pgfsetlinewidth{0.803000pt}%
\definecolor{currentstroke}{rgb}{0.800000,0.800000,0.800000}%
\pgfsetstrokecolor{currentstroke}%
\pgfsetdash{}{0pt}%
\pgfpathmoveto{\pgfqpoint{1.934666in}{0.226389in}}%
\pgfpathlineto{\pgfqpoint{1.934666in}{1.766389in}}%
\pgfusepath{stroke}%
\end{pgfscope}%
\begin{pgfscope}%
\definecolor{textcolor}{rgb}{0.501961,0.501961,0.501961}%
\pgfsetstrokecolor{textcolor}%
\pgfsetfillcolor{textcolor}%
\pgftext[x=1.934666in,y=0.111111in,,top]{\color{textcolor}\rmfamily\fontsize{8.800000}{10.560000}\selectfont \(\displaystyle {3.5}\)}%
\end{pgfscope}%
\begin{pgfscope}%
\pgfpathrectangle{\pgfqpoint{0.243749in}{0.226389in}}{\pgfqpoint{2.325000in}{1.540000in}}%
\pgfusepath{clip}%
\pgfsetroundcap%
\pgfsetroundjoin%
\pgfsetlinewidth{0.803000pt}%
\definecolor{currentstroke}{rgb}{0.800000,0.800000,0.800000}%
\pgfsetstrokecolor{currentstroke}%
\pgfsetdash{}{0pt}%
\pgfpathmoveto{\pgfqpoint{2.463078in}{0.226389in}}%
\pgfpathlineto{\pgfqpoint{2.463078in}{1.766389in}}%
\pgfusepath{stroke}%
\end{pgfscope}%
\begin{pgfscope}%
\definecolor{textcolor}{rgb}{0.501961,0.501961,0.501961}%
\pgfsetstrokecolor{textcolor}%
\pgfsetfillcolor{textcolor}%
\pgftext[x=2.463078in,y=0.111111in,,top]{\color{textcolor}\rmfamily\fontsize{8.800000}{10.560000}\selectfont \(\displaystyle {4.0}\)}%
\end{pgfscope}%
\begin{pgfscope}%
\pgfpathrectangle{\pgfqpoint{0.243749in}{0.226389in}}{\pgfqpoint{2.325000in}{1.540000in}}%
\pgfusepath{clip}%
\pgfsetroundcap%
\pgfsetroundjoin%
\pgfsetlinewidth{0.803000pt}%
\definecolor{currentstroke}{rgb}{0.800000,0.800000,0.800000}%
\pgfsetstrokecolor{currentstroke}%
\pgfsetdash{}{0pt}%
\pgfpathmoveto{\pgfqpoint{0.243749in}{0.540578in}}%
\pgfpathlineto{\pgfqpoint{2.568749in}{0.540578in}}%
\pgfusepath{stroke}%
\end{pgfscope}%
\begin{pgfscope}%
\definecolor{textcolor}{rgb}{0.501961,0.501961,0.501961}%
\pgfsetstrokecolor{textcolor}%
\pgfsetfillcolor{textcolor}%
\pgftext[x=0.064236in, y=0.497175in, left, base]{\color{textcolor}\rmfamily\fontsize{8.800000}{10.560000}\selectfont \(\displaystyle {0}\)}%
\end{pgfscope}%
\begin{pgfscope}%
\pgfpathrectangle{\pgfqpoint{0.243749in}{0.226389in}}{\pgfqpoint{2.325000in}{1.540000in}}%
\pgfusepath{clip}%
\pgfsetroundcap%
\pgfsetroundjoin%
\pgfsetlinewidth{0.803000pt}%
\definecolor{currentstroke}{rgb}{0.800000,0.800000,0.800000}%
\pgfsetstrokecolor{currentstroke}%
\pgfsetdash{}{0pt}%
\pgfpathmoveto{\pgfqpoint{0.243749in}{0.866163in}}%
\pgfpathlineto{\pgfqpoint{2.568749in}{0.866163in}}%
\pgfusepath{stroke}%
\end{pgfscope}%
\begin{pgfscope}%
\definecolor{textcolor}{rgb}{0.501961,0.501961,0.501961}%
\pgfsetstrokecolor{textcolor}%
\pgfsetfillcolor{textcolor}%
\pgftext[x=0.000000in, y=0.822760in, left, base]{\color{textcolor}\rmfamily\fontsize{8.800000}{10.560000}\selectfont \(\displaystyle {20}\)}%
\end{pgfscope}%
\begin{pgfscope}%
\pgfpathrectangle{\pgfqpoint{0.243749in}{0.226389in}}{\pgfqpoint{2.325000in}{1.540000in}}%
\pgfusepath{clip}%
\pgfsetroundcap%
\pgfsetroundjoin%
\pgfsetlinewidth{0.803000pt}%
\definecolor{currentstroke}{rgb}{0.800000,0.800000,0.800000}%
\pgfsetstrokecolor{currentstroke}%
\pgfsetdash{}{0pt}%
\pgfpathmoveto{\pgfqpoint{0.243749in}{1.191748in}}%
\pgfpathlineto{\pgfqpoint{2.568749in}{1.191748in}}%
\pgfusepath{stroke}%
\end{pgfscope}%
\begin{pgfscope}%
\definecolor{textcolor}{rgb}{0.501961,0.501961,0.501961}%
\pgfsetstrokecolor{textcolor}%
\pgfsetfillcolor{textcolor}%
\pgftext[x=0.000000in, y=1.148345in, left, base]{\color{textcolor}\rmfamily\fontsize{8.800000}{10.560000}\selectfont \(\displaystyle {40}\)}%
\end{pgfscope}%
\begin{pgfscope}%
\pgfpathrectangle{\pgfqpoint{0.243749in}{0.226389in}}{\pgfqpoint{2.325000in}{1.540000in}}%
\pgfusepath{clip}%
\pgfsetroundcap%
\pgfsetroundjoin%
\pgfsetlinewidth{0.803000pt}%
\definecolor{currentstroke}{rgb}{0.800000,0.800000,0.800000}%
\pgfsetstrokecolor{currentstroke}%
\pgfsetdash{}{0pt}%
\pgfpathmoveto{\pgfqpoint{0.243749in}{1.517333in}}%
\pgfpathlineto{\pgfqpoint{2.568749in}{1.517333in}}%
\pgfusepath{stroke}%
\end{pgfscope}%
\begin{pgfscope}%
\definecolor{textcolor}{rgb}{0.501961,0.501961,0.501961}%
\pgfsetstrokecolor{textcolor}%
\pgfsetfillcolor{textcolor}%
\pgftext[x=0.000000in, y=1.473930in, left, base]{\color{textcolor}\rmfamily\fontsize{8.800000}{10.560000}\selectfont \(\displaystyle {60}\)}%
\end{pgfscope}%
\begin{pgfscope}%
\pgfpathrectangle{\pgfqpoint{0.243749in}{0.226389in}}{\pgfqpoint{2.325000in}{1.540000in}}%
\pgfusepath{clip}%
\pgfsetroundcap%
\pgfsetroundjoin%
\pgfsetlinewidth{1.204500pt}%
\definecolor{currentstroke}{rgb}{0.007843,0.243137,1.000000}%
\pgfsetstrokecolor{currentstroke}%
\pgfsetdash{}{0pt}%
\pgfpathmoveto{\pgfqpoint{0.349431in}{0.296389in}}%
\pgfpathlineto{\pgfqpoint{0.428207in}{0.301142in}}%
\pgfpathlineto{\pgfqpoint{0.504351in}{0.307951in}}%
\pgfpathlineto{\pgfqpoint{0.578244in}{0.316781in}}%
\pgfpathlineto{\pgfqpoint{0.650161in}{0.327605in}}%
\pgfpathlineto{\pgfqpoint{0.720365in}{0.340412in}}%
\pgfpathlineto{\pgfqpoint{0.789059in}{0.355197in}}%
\pgfpathlineto{\pgfqpoint{0.856421in}{0.371965in}}%
\pgfpathlineto{\pgfqpoint{0.922610in}{0.390729in}}%
\pgfpathlineto{\pgfqpoint{0.987763in}{0.411506in}}%
\pgfpathlineto{\pgfqpoint{1.052007in}{0.434326in}}%
\pgfpathlineto{\pgfqpoint{1.115448in}{0.459218in}}%
\pgfpathlineto{\pgfqpoint{1.178192in}{0.486226in}}%
\pgfpathlineto{\pgfqpoint{1.240333in}{0.515395in}}%
\pgfpathlineto{\pgfqpoint{1.301957in}{0.546780in}}%
\pgfpathlineto{\pgfqpoint{1.363136in}{0.580437in}}%
\pgfpathlineto{\pgfqpoint{1.423946in}{0.616433in}}%
\pgfpathlineto{\pgfqpoint{1.484449in}{0.654836in}}%
\pgfpathlineto{\pgfqpoint{1.544709in}{0.695724in}}%
\pgfpathlineto{\pgfqpoint{1.604779in}{0.739178in}}%
\pgfpathlineto{\pgfqpoint{1.664711in}{0.785284in}}%
\pgfpathlineto{\pgfqpoint{1.724559in}{0.834139in}}%
\pgfpathlineto{\pgfqpoint{1.784365in}{0.885841in}}%
\pgfpathlineto{\pgfqpoint{1.844170in}{0.940492in}}%
\pgfpathlineto{\pgfqpoint{1.904008in}{0.998196in}}%
\pgfpathlineto{\pgfqpoint{1.963930in}{1.059082in}}%
\pgfpathlineto{\pgfqpoint{2.023957in}{1.123260in}}%
\pgfpathlineto{\pgfqpoint{2.084122in}{1.190853in}}%
\pgfpathlineto{\pgfqpoint{2.144456in}{1.261995in}}%
\pgfpathlineto{\pgfqpoint{2.204991in}{1.336828in}}%
\pgfpathlineto{\pgfqpoint{2.265748in}{1.415487in}}%
\pgfpathlineto{\pgfqpoint{2.326758in}{1.498130in}}%
\pgfpathlineto{\pgfqpoint{2.388033in}{1.584894in}}%
\pgfpathlineto{\pgfqpoint{2.449593in}{1.675936in}}%
\pgfpathlineto{\pgfqpoint{2.463067in}{1.696389in}}%
\pgfpathlineto{\pgfqpoint{2.463067in}{1.696389in}}%
\pgfusepath{stroke}%
\end{pgfscope}%
\begin{pgfscope}%
\pgfpathrectangle{\pgfqpoint{0.243749in}{0.226389in}}{\pgfqpoint{2.325000in}{1.540000in}}%
\pgfusepath{clip}%
\pgfsetbuttcap%
\pgfsetroundjoin%
\definecolor{currentfill}{rgb}{1.000000,0.486275,0.000000}%
\pgfsetfillcolor{currentfill}%
\pgfsetlinewidth{1.003750pt}%
\definecolor{currentstroke}{rgb}{1.000000,0.486275,0.000000}%
\pgfsetstrokecolor{currentstroke}%
\pgfsetdash{}{0pt}%
\pgfsys@defobject{currentmarker}{\pgfqpoint{-0.033333in}{-0.033333in}}{\pgfqpoint{0.033333in}{0.033333in}}{%
\pgfpathmoveto{\pgfqpoint{0.000000in}{-0.033333in}}%
\pgfpathcurveto{\pgfqpoint{0.008840in}{-0.033333in}}{\pgfqpoint{0.017319in}{-0.029821in}}{\pgfqpoint{0.023570in}{-0.023570in}}%
\pgfpathcurveto{\pgfqpoint{0.029821in}{-0.017319in}}{\pgfqpoint{0.033333in}{-0.008840in}}{\pgfqpoint{0.033333in}{0.000000in}}%
\pgfpathcurveto{\pgfqpoint{0.033333in}{0.008840in}}{\pgfqpoint{0.029821in}{0.017319in}}{\pgfqpoint{0.023570in}{0.023570in}}%
\pgfpathcurveto{\pgfqpoint{0.017319in}{0.029821in}}{\pgfqpoint{0.008840in}{0.033333in}}{\pgfqpoint{0.000000in}{0.033333in}}%
\pgfpathcurveto{\pgfqpoint{-0.008840in}{0.033333in}}{\pgfqpoint{-0.017319in}{0.029821in}}{\pgfqpoint{-0.023570in}{0.023570in}}%
\pgfpathcurveto{\pgfqpoint{-0.029821in}{0.017319in}}{\pgfqpoint{-0.033333in}{0.008840in}}{\pgfqpoint{-0.033333in}{0.000000in}}%
\pgfpathcurveto{\pgfqpoint{-0.033333in}{-0.008840in}}{\pgfqpoint{-0.029821in}{-0.017319in}}{\pgfqpoint{-0.023570in}{-0.023570in}}%
\pgfpathcurveto{\pgfqpoint{-0.017319in}{-0.029821in}}{\pgfqpoint{-0.008840in}{-0.033333in}}{\pgfqpoint{0.000000in}{-0.033333in}}%
\pgfpathclose%
\pgfusepath{stroke,fill}%
}%
\begin{pgfscope}%
\pgfsys@transformshift{1.290004in}{0.540578in}%
\pgfsys@useobject{currentmarker}{}%
\end{pgfscope}%
\end{pgfscope}%
\begin{pgfscope}%
\pgfsetrectcap%
\pgfsetmiterjoin%
\pgfsetlinewidth{1.003750pt}%
\definecolor{currentstroke}{rgb}{0.150000,0.150000,0.150000}%
\pgfsetstrokecolor{currentstroke}%
\pgfsetdash{}{0pt}%
\pgfpathmoveto{\pgfqpoint{0.243749in}{0.226389in}}%
\pgfpathlineto{\pgfqpoint{0.243749in}{1.766389in}}%
\pgfusepath{stroke}%
\end{pgfscope}%
\begin{pgfscope}%
\pgfsetrectcap%
\pgfsetmiterjoin%
\pgfsetlinewidth{1.003750pt}%
\definecolor{currentstroke}{rgb}{0.150000,0.150000,0.150000}%
\pgfsetstrokecolor{currentstroke}%
\pgfsetdash{}{0pt}%
\pgfpathmoveto{\pgfqpoint{2.568749in}{0.226389in}}%
\pgfpathlineto{\pgfqpoint{2.568749in}{1.766389in}}%
\pgfusepath{stroke}%
\end{pgfscope}%
\begin{pgfscope}%
\pgfsetrectcap%
\pgfsetmiterjoin%
\pgfsetlinewidth{1.003750pt}%
\definecolor{currentstroke}{rgb}{0.150000,0.150000,0.150000}%
\pgfsetstrokecolor{currentstroke}%
\pgfsetdash{}{0pt}%
\pgfpathmoveto{\pgfqpoint{0.243749in}{0.226389in}}%
\pgfpathlineto{\pgfqpoint{2.568749in}{0.226389in}}%
\pgfusepath{stroke}%
\end{pgfscope}%
\begin{pgfscope}%
\pgfsetrectcap%
\pgfsetmiterjoin%
\pgfsetlinewidth{1.003750pt}%
\definecolor{currentstroke}{rgb}{0.150000,0.150000,0.150000}%
\pgfsetstrokecolor{currentstroke}%
\pgfsetdash{}{0pt}%
\pgfpathmoveto{\pgfqpoint{0.243749in}{1.766389in}}%
\pgfpathlineto{\pgfqpoint{2.568749in}{1.766389in}}%
\pgfusepath{stroke}%
\end{pgfscope}%
\end{pgfpicture}%
\makeatother%
\endgroup%
 }}%
    \qquad
    \subfloat[\centering $ 2 \tan (x) - \frac{x}{2} + 1 = 0 $]{{%% Creator: Matplotlib, PGF backend
%%
%% To include the figure in your LaTeX document, write
%%   \input{<filename>.pgf}
%%
%% Make sure the required packages are loaded in your preamble
%%   \usepackage{pgf}
%%
%% Figures using additional raster images can only be included by \input if
%% they are in the same directory as the main LaTeX file. For loading figures
%% from other directories you can use the `import` package
%%   \usepackage{import}
%%
%% and then include the figures with
%%   \import{<path to file>}{<filename>.pgf}
%%
%% Matplotlib used the following preamble
%%
\begingroup%
\makeatletter%
\begin{pgfpicture}%
\pgfpathrectangle{\pgfpointorigin}{\pgfqpoint{2.704358in}{1.766389in}}%
\pgfusepath{use as bounding box, clip}%
\begin{pgfscope}%
\pgfsetbuttcap%
\pgfsetmiterjoin%
\pgfsetlinewidth{0.000000pt}%
\definecolor{currentstroke}{rgb}{0.000000,0.000000,0.000000}%
\pgfsetstrokecolor{currentstroke}%
\pgfsetstrokeopacity{0.000000}%
\pgfsetdash{}{0pt}%
\pgfpathmoveto{\pgfqpoint{0.000000in}{0.000000in}}%
\pgfpathlineto{\pgfqpoint{2.704358in}{0.000000in}}%
\pgfpathlineto{\pgfqpoint{2.704358in}{1.766389in}}%
\pgfpathlineto{\pgfqpoint{0.000000in}{1.766389in}}%
\pgfpathclose%
\pgfusepath{}%
\end{pgfscope}%
\begin{pgfscope}%
\pgfsetbuttcap%
\pgfsetmiterjoin%
\pgfsetlinewidth{0.000000pt}%
\definecolor{currentstroke}{rgb}{0.000000,0.000000,0.000000}%
\pgfsetstrokecolor{currentstroke}%
\pgfsetstrokeopacity{0.000000}%
\pgfsetdash{}{0pt}%
\pgfpathmoveto{\pgfqpoint{0.379358in}{0.226389in}}%
\pgfpathlineto{\pgfqpoint{2.704358in}{0.226389in}}%
\pgfpathlineto{\pgfqpoint{2.704358in}{1.766389in}}%
\pgfpathlineto{\pgfqpoint{0.379358in}{1.766389in}}%
\pgfpathclose%
\pgfusepath{}%
\end{pgfscope}%
\begin{pgfscope}%
\pgfpathrectangle{\pgfqpoint{0.379358in}{0.226389in}}{\pgfqpoint{2.325000in}{1.540000in}}%
\pgfusepath{clip}%
\pgfsetroundcap%
\pgfsetroundjoin%
\pgfsetlinewidth{0.803000pt}%
\definecolor{currentstroke}{rgb}{0.800000,0.800000,0.800000}%
\pgfsetstrokecolor{currentstroke}%
\pgfsetdash{}{0pt}%
\pgfpathmoveto{\pgfqpoint{0.837313in}{0.226389in}}%
\pgfpathlineto{\pgfqpoint{0.837313in}{1.766389in}}%
\pgfusepath{stroke}%
\end{pgfscope}%
\begin{pgfscope}%
\definecolor{textcolor}{rgb}{0.501961,0.501961,0.501961}%
\pgfsetstrokecolor{textcolor}%
\pgfsetfillcolor{textcolor}%
\pgftext[x=0.837313in,y=0.111111in,,top]{\color{textcolor}\rmfamily\fontsize{8.800000}{10.560000}\selectfont \(\displaystyle {\ensuremath{-}1}\)}%
\end{pgfscope}%
\begin{pgfscope}%
\pgfpathrectangle{\pgfqpoint{0.379358in}{0.226389in}}{\pgfqpoint{2.325000in}{1.540000in}}%
\pgfusepath{clip}%
\pgfsetroundcap%
\pgfsetroundjoin%
\pgfsetlinewidth{0.803000pt}%
\definecolor{currentstroke}{rgb}{0.800000,0.800000,0.800000}%
\pgfsetstrokecolor{currentstroke}%
\pgfsetdash{}{0pt}%
\pgfpathmoveto{\pgfqpoint{1.541861in}{0.226389in}}%
\pgfpathlineto{\pgfqpoint{1.541861in}{1.766389in}}%
\pgfusepath{stroke}%
\end{pgfscope}%
\begin{pgfscope}%
\definecolor{textcolor}{rgb}{0.501961,0.501961,0.501961}%
\pgfsetstrokecolor{textcolor}%
\pgfsetfillcolor{textcolor}%
\pgftext[x=1.541861in,y=0.111111in,,top]{\color{textcolor}\rmfamily\fontsize{8.800000}{10.560000}\selectfont \(\displaystyle {0}\)}%
\end{pgfscope}%
\begin{pgfscope}%
\pgfpathrectangle{\pgfqpoint{0.379358in}{0.226389in}}{\pgfqpoint{2.325000in}{1.540000in}}%
\pgfusepath{clip}%
\pgfsetroundcap%
\pgfsetroundjoin%
\pgfsetlinewidth{0.803000pt}%
\definecolor{currentstroke}{rgb}{0.800000,0.800000,0.800000}%
\pgfsetstrokecolor{currentstroke}%
\pgfsetdash{}{0pt}%
\pgfpathmoveto{\pgfqpoint{2.246409in}{0.226389in}}%
\pgfpathlineto{\pgfqpoint{2.246409in}{1.766389in}}%
\pgfusepath{stroke}%
\end{pgfscope}%
\begin{pgfscope}%
\definecolor{textcolor}{rgb}{0.501961,0.501961,0.501961}%
\pgfsetstrokecolor{textcolor}%
\pgfsetfillcolor{textcolor}%
\pgftext[x=2.246409in,y=0.111111in,,top]{\color{textcolor}\rmfamily\fontsize{8.800000}{10.560000}\selectfont \(\displaystyle {1}\)}%
\end{pgfscope}%
\begin{pgfscope}%
\pgfpathrectangle{\pgfqpoint{0.379358in}{0.226389in}}{\pgfqpoint{2.325000in}{1.540000in}}%
\pgfusepath{clip}%
\pgfsetroundcap%
\pgfsetroundjoin%
\pgfsetlinewidth{0.803000pt}%
\definecolor{currentstroke}{rgb}{0.800000,0.800000,0.800000}%
\pgfsetstrokecolor{currentstroke}%
\pgfsetdash{}{0pt}%
\pgfpathmoveto{\pgfqpoint{0.379358in}{0.281389in}}%
\pgfpathlineto{\pgfqpoint{2.704358in}{0.281389in}}%
\pgfusepath{stroke}%
\end{pgfscope}%
\begin{pgfscope}%
\definecolor{textcolor}{rgb}{0.501961,0.501961,0.501961}%
\pgfsetstrokecolor{textcolor}%
\pgfsetfillcolor{textcolor}%
\pgftext[x=0.000000in, y=0.237986in, left, base]{\color{textcolor}\rmfamily\fontsize{8.800000}{10.560000}\selectfont \(\displaystyle {\ensuremath{-}7.5}\)}%
\end{pgfscope}%
\begin{pgfscope}%
\pgfpathrectangle{\pgfqpoint{0.379358in}{0.226389in}}{\pgfqpoint{2.325000in}{1.540000in}}%
\pgfusepath{clip}%
\pgfsetroundcap%
\pgfsetroundjoin%
\pgfsetlinewidth{0.803000pt}%
\definecolor{currentstroke}{rgb}{0.800000,0.800000,0.800000}%
\pgfsetstrokecolor{currentstroke}%
\pgfsetdash{}{0pt}%
\pgfpathmoveto{\pgfqpoint{0.379358in}{0.556389in}}%
\pgfpathlineto{\pgfqpoint{2.704358in}{0.556389in}}%
\pgfusepath{stroke}%
\end{pgfscope}%
\begin{pgfscope}%
\definecolor{textcolor}{rgb}{0.501961,0.501961,0.501961}%
\pgfsetstrokecolor{textcolor}%
\pgfsetfillcolor{textcolor}%
\pgftext[x=0.000000in, y=0.512986in, left, base]{\color{textcolor}\rmfamily\fontsize{8.800000}{10.560000}\selectfont \(\displaystyle {\ensuremath{-}5.0}\)}%
\end{pgfscope}%
\begin{pgfscope}%
\pgfpathrectangle{\pgfqpoint{0.379358in}{0.226389in}}{\pgfqpoint{2.325000in}{1.540000in}}%
\pgfusepath{clip}%
\pgfsetroundcap%
\pgfsetroundjoin%
\pgfsetlinewidth{0.803000pt}%
\definecolor{currentstroke}{rgb}{0.800000,0.800000,0.800000}%
\pgfsetstrokecolor{currentstroke}%
\pgfsetdash{}{0pt}%
\pgfpathmoveto{\pgfqpoint{0.379358in}{0.831389in}}%
\pgfpathlineto{\pgfqpoint{2.704358in}{0.831389in}}%
\pgfusepath{stroke}%
\end{pgfscope}%
\begin{pgfscope}%
\definecolor{textcolor}{rgb}{0.501961,0.501961,0.501961}%
\pgfsetstrokecolor{textcolor}%
\pgfsetfillcolor{textcolor}%
\pgftext[x=0.000000in, y=0.787986in, left, base]{\color{textcolor}\rmfamily\fontsize{8.800000}{10.560000}\selectfont \(\displaystyle {\ensuremath{-}2.5}\)}%
\end{pgfscope}%
\begin{pgfscope}%
\pgfpathrectangle{\pgfqpoint{0.379358in}{0.226389in}}{\pgfqpoint{2.325000in}{1.540000in}}%
\pgfusepath{clip}%
\pgfsetroundcap%
\pgfsetroundjoin%
\pgfsetlinewidth{0.803000pt}%
\definecolor{currentstroke}{rgb}{0.800000,0.800000,0.800000}%
\pgfsetstrokecolor{currentstroke}%
\pgfsetdash{}{0pt}%
\pgfpathmoveto{\pgfqpoint{0.379358in}{1.106389in}}%
\pgfpathlineto{\pgfqpoint{2.704358in}{1.106389in}}%
\pgfusepath{stroke}%
\end{pgfscope}%
\begin{pgfscope}%
\definecolor{textcolor}{rgb}{0.501961,0.501961,0.501961}%
\pgfsetstrokecolor{textcolor}%
\pgfsetfillcolor{textcolor}%
\pgftext[x=0.099922in, y=1.062986in, left, base]{\color{textcolor}\rmfamily\fontsize{8.800000}{10.560000}\selectfont \(\displaystyle {0.0}\)}%
\end{pgfscope}%
\begin{pgfscope}%
\pgfpathrectangle{\pgfqpoint{0.379358in}{0.226389in}}{\pgfqpoint{2.325000in}{1.540000in}}%
\pgfusepath{clip}%
\pgfsetroundcap%
\pgfsetroundjoin%
\pgfsetlinewidth{0.803000pt}%
\definecolor{currentstroke}{rgb}{0.800000,0.800000,0.800000}%
\pgfsetstrokecolor{currentstroke}%
\pgfsetdash{}{0pt}%
\pgfpathmoveto{\pgfqpoint{0.379358in}{1.381389in}}%
\pgfpathlineto{\pgfqpoint{2.704358in}{1.381389in}}%
\pgfusepath{stroke}%
\end{pgfscope}%
\begin{pgfscope}%
\definecolor{textcolor}{rgb}{0.501961,0.501961,0.501961}%
\pgfsetstrokecolor{textcolor}%
\pgfsetfillcolor{textcolor}%
\pgftext[x=0.099922in, y=1.337986in, left, base]{\color{textcolor}\rmfamily\fontsize{8.800000}{10.560000}\selectfont \(\displaystyle {2.5}\)}%
\end{pgfscope}%
\begin{pgfscope}%
\pgfpathrectangle{\pgfqpoint{0.379358in}{0.226389in}}{\pgfqpoint{2.325000in}{1.540000in}}%
\pgfusepath{clip}%
\pgfsetroundcap%
\pgfsetroundjoin%
\pgfsetlinewidth{0.803000pt}%
\definecolor{currentstroke}{rgb}{0.800000,0.800000,0.800000}%
\pgfsetstrokecolor{currentstroke}%
\pgfsetdash{}{0pt}%
\pgfpathmoveto{\pgfqpoint{0.379358in}{1.656389in}}%
\pgfpathlineto{\pgfqpoint{2.704358in}{1.656389in}}%
\pgfusepath{stroke}%
\end{pgfscope}%
\begin{pgfscope}%
\definecolor{textcolor}{rgb}{0.501961,0.501961,0.501961}%
\pgfsetstrokecolor{textcolor}%
\pgfsetfillcolor{textcolor}%
\pgftext[x=0.099922in, y=1.612986in, left, base]{\color{textcolor}\rmfamily\fontsize{8.800000}{10.560000}\selectfont \(\displaystyle {5.0}\)}%
\end{pgfscope}%
\begin{pgfscope}%
\pgfpathrectangle{\pgfqpoint{0.379358in}{0.226389in}}{\pgfqpoint{2.325000in}{1.540000in}}%
\pgfusepath{clip}%
\pgfsetroundcap%
\pgfsetroundjoin%
\pgfsetlinewidth{1.204500pt}%
\definecolor{currentstroke}{rgb}{0.007843,0.243137,1.000000}%
\pgfsetstrokecolor{currentstroke}%
\pgfsetdash{}{0pt}%
\pgfpathmoveto{\pgfqpoint{0.577345in}{0.216389in}}%
\pgfpathlineto{\pgfqpoint{0.590560in}{0.309447in}}%
\pgfpathlineto{\pgfqpoint{0.604383in}{0.391299in}}%
\pgfpathlineto{\pgfqpoint{0.618826in}{0.463723in}}%
\pgfpathlineto{\pgfqpoint{0.633904in}{0.528162in}}%
\pgfpathlineto{\pgfqpoint{0.649643in}{0.585832in}}%
\pgfpathlineto{\pgfqpoint{0.666073in}{0.637714in}}%
\pgfpathlineto{\pgfqpoint{0.683229in}{0.684625in}}%
\pgfpathlineto{\pgfqpoint{0.701160in}{0.727261in}}%
\pgfpathlineto{\pgfqpoint{0.719929in}{0.766217in}}%
\pgfpathlineto{\pgfqpoint{0.739607in}{0.801984in}}%
\pgfpathlineto{\pgfqpoint{0.760285in}{0.834992in}}%
\pgfpathlineto{\pgfqpoint{0.782077in}{0.865613in}}%
\pgfpathlineto{\pgfqpoint{0.805116in}{0.894166in}}%
\pgfpathlineto{\pgfqpoint{0.829563in}{0.920930in}}%
\pgfpathlineto{\pgfqpoint{0.855604in}{0.946139in}}%
\pgfpathlineto{\pgfqpoint{0.883468in}{0.970013in}}%
\pgfpathlineto{\pgfqpoint{0.913426in}{0.992739in}}%
\pgfpathlineto{\pgfqpoint{0.945793in}{1.014483in}}%
\pgfpathlineto{\pgfqpoint{0.980964in}{1.035411in}}%
\pgfpathlineto{\pgfqpoint{1.019404in}{1.055668in}}%
\pgfpathlineto{\pgfqpoint{1.061712in}{1.075419in}}%
\pgfpathlineto{\pgfqpoint{1.108656in}{1.094844in}}%
\pgfpathlineto{\pgfqpoint{1.161286in}{1.114173in}}%
\pgfpathlineto{\pgfqpoint{1.221130in}{1.133732in}}%
\pgfpathlineto{\pgfqpoint{1.290739in}{1.154079in}}%
\pgfpathlineto{\pgfqpoint{1.375299in}{1.176390in}}%
\pgfpathlineto{\pgfqpoint{1.491761in}{1.204629in}}%
\pgfpathlineto{\pgfqpoint{1.732491in}{1.262529in}}%
\pgfpathlineto{\pgfqpoint{1.817783in}{1.285701in}}%
\pgfpathlineto{\pgfqpoint{1.886928in}{1.306732in}}%
\pgfpathlineto{\pgfqpoint{1.945962in}{1.326964in}}%
\pgfpathlineto{\pgfqpoint{1.997633in}{1.346981in}}%
\pgfpathlineto{\pgfqpoint{2.043563in}{1.367121in}}%
\pgfpathlineto{\pgfqpoint{2.084835in}{1.387615in}}%
\pgfpathlineto{\pgfqpoint{2.122233in}{1.408639in}}%
\pgfpathlineto{\pgfqpoint{2.156368in}{1.430356in}}%
\pgfpathlineto{\pgfqpoint{2.187720in}{1.452919in}}%
\pgfpathlineto{\pgfqpoint{2.216677in}{1.476482in}}%
\pgfpathlineto{\pgfqpoint{2.243570in}{1.501222in}}%
\pgfpathlineto{\pgfqpoint{2.268659in}{1.527321in}}%
\pgfpathlineto{\pgfqpoint{2.292177in}{1.554996in}}%
\pgfpathlineto{\pgfqpoint{2.314306in}{1.584482in}}%
\pgfpathlineto{\pgfqpoint{2.335210in}{1.616056in}}%
\pgfpathlineto{\pgfqpoint{2.355022in}{1.650036in}}%
\pgfpathlineto{\pgfqpoint{2.373848in}{1.686776in}}%
\pgfpathlineto{\pgfqpoint{2.391778in}{1.726696in}}%
\pgfpathlineto{\pgfqpoint{2.408885in}{1.770268in}}%
\pgfpathlineto{\pgfqpoint{2.411109in}{1.776389in}}%
\pgfpathlineto{\pgfqpoint{2.411109in}{1.776389in}}%
\pgfusepath{stroke}%
\end{pgfscope}%
\begin{pgfscope}%
\pgfpathrectangle{\pgfqpoint{0.379358in}{0.226389in}}{\pgfqpoint{2.325000in}{1.540000in}}%
\pgfusepath{clip}%
\pgfsetbuttcap%
\pgfsetroundjoin%
\definecolor{currentfill}{rgb}{1.000000,0.486275,0.000000}%
\pgfsetfillcolor{currentfill}%
\pgfsetlinewidth{1.003750pt}%
\definecolor{currentstroke}{rgb}{1.000000,0.486275,0.000000}%
\pgfsetstrokecolor{currentstroke}%
\pgfsetdash{}{0pt}%
\pgfsys@defobject{currentmarker}{\pgfqpoint{-0.033333in}{-0.033333in}}{\pgfqpoint{0.033333in}{0.033333in}}{%
\pgfpathmoveto{\pgfqpoint{0.000000in}{-0.033333in}}%
\pgfpathcurveto{\pgfqpoint{0.008840in}{-0.033333in}}{\pgfqpoint{0.017319in}{-0.029821in}}{\pgfqpoint{0.023570in}{-0.023570in}}%
\pgfpathcurveto{\pgfqpoint{0.029821in}{-0.017319in}}{\pgfqpoint{0.033333in}{-0.008840in}}{\pgfqpoint{0.033333in}{0.000000in}}%
\pgfpathcurveto{\pgfqpoint{0.033333in}{0.008840in}}{\pgfqpoint{0.029821in}{0.017319in}}{\pgfqpoint{0.023570in}{0.023570in}}%
\pgfpathcurveto{\pgfqpoint{0.017319in}{0.029821in}}{\pgfqpoint{0.008840in}{0.033333in}}{\pgfqpoint{0.000000in}{0.033333in}}%
\pgfpathcurveto{\pgfqpoint{-0.008840in}{0.033333in}}{\pgfqpoint{-0.017319in}{0.029821in}}{\pgfqpoint{-0.023570in}{0.023570in}}%
\pgfpathcurveto{\pgfqpoint{-0.029821in}{0.017319in}}{\pgfqpoint{-0.033333in}{0.008840in}}{\pgfqpoint{-0.033333in}{0.000000in}}%
\pgfpathcurveto{\pgfqpoint{-0.033333in}{-0.008840in}}{\pgfqpoint{-0.029821in}{-0.017319in}}{\pgfqpoint{-0.023570in}{-0.023570in}}%
\pgfpathcurveto{\pgfqpoint{-0.017319in}{-0.029821in}}{\pgfqpoint{-0.008840in}{-0.033333in}}{\pgfqpoint{0.000000in}{-0.033333in}}%
\pgfpathclose%
\pgfusepath{stroke,fill}%
}%
\begin{pgfscope}%
\pgfsys@transformshift{1.139353in}{1.106389in}%
\pgfsys@useobject{currentmarker}{}%
\end{pgfscope}%
\end{pgfscope}%
\begin{pgfscope}%
\pgfsetrectcap%
\pgfsetmiterjoin%
\pgfsetlinewidth{1.003750pt}%
\definecolor{currentstroke}{rgb}{0.150000,0.150000,0.150000}%
\pgfsetstrokecolor{currentstroke}%
\pgfsetdash{}{0pt}%
\pgfpathmoveto{\pgfqpoint{0.379358in}{0.226389in}}%
\pgfpathlineto{\pgfqpoint{0.379358in}{1.766389in}}%
\pgfusepath{stroke}%
\end{pgfscope}%
\begin{pgfscope}%
\pgfsetrectcap%
\pgfsetmiterjoin%
\pgfsetlinewidth{1.003750pt}%
\definecolor{currentstroke}{rgb}{0.150000,0.150000,0.150000}%
\pgfsetstrokecolor{currentstroke}%
\pgfsetdash{}{0pt}%
\pgfpathmoveto{\pgfqpoint{2.704358in}{0.226389in}}%
\pgfpathlineto{\pgfqpoint{2.704358in}{1.766389in}}%
\pgfusepath{stroke}%
\end{pgfscope}%
\begin{pgfscope}%
\pgfsetrectcap%
\pgfsetmiterjoin%
\pgfsetlinewidth{1.003750pt}%
\definecolor{currentstroke}{rgb}{0.150000,0.150000,0.150000}%
\pgfsetstrokecolor{currentstroke}%
\pgfsetdash{}{0pt}%
\pgfpathmoveto{\pgfqpoint{0.379358in}{0.226389in}}%
\pgfpathlineto{\pgfqpoint{2.704358in}{0.226389in}}%
\pgfusepath{stroke}%
\end{pgfscope}%
\begin{pgfscope}%
\pgfsetrectcap%
\pgfsetmiterjoin%
\pgfsetlinewidth{1.003750pt}%
\definecolor{currentstroke}{rgb}{0.150000,0.150000,0.150000}%
\pgfsetstrokecolor{currentstroke}%
\pgfsetdash{}{0pt}%
\pgfpathmoveto{\pgfqpoint{0.379358in}{1.766389in}}%
\pgfpathlineto{\pgfqpoint{2.704358in}{1.766389in}}%
\pgfusepath{stroke}%
\end{pgfscope}%
\end{pgfpicture}%
\makeatother%
\endgroup%
 }}%
    \caption{Графики}%
    \label{fig:example}%
\end{figure}

Первая функция возрастает во всей области определения. Корень находится в сегменте $[2.5, 3.5]$. Вторая функция периодическая. Один из корней находится в сегменте $ [-1, 0] $.

\end{document}
